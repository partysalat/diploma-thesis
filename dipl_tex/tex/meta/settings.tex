%BEGIN COLORS
% Light gray
\definecolor{Lgray}{HTML}{CCCCCC}
% Dark gray
\definecolor{Dgray}{HTML}{727272}
% Almost black, for links and stuff
\definecolor{Ddgray}{HTML}{4C4C4C}
%END

%BEGIN KOMA
% No font changing for headers
\setkomafont{sectioning}{\normalcolor\bfseries}
\KOMAoptions{DIV=last}

% enable head and foot
\pagestyle{scrheadings}
\clearscrheadings
 
% head and foot for chapter beginning
\renewcommand*{\chapterpagestyle}{scrheadings}	
 
\ohead{\headmark}
%\setlength{\headheight}{21mm} 				% height of head
%\setheadwidth[0pt]{textwithmarginpar} % head above text
\setheadsepline[text]{0.4pt} 					% 
 
\ofoot{\pagemark}
% \lofoot{\theauthor}
% \rofoot{\pagemark}
% \lefoot{\pagemark}

\setkomafont{caption}{\small}

% normal font for head and foot
\setkomafont{pageheadfoot}{\small}
\setkomafont{footnote}{\small\linespread{1}\selectfont}
%END

%BEGIN MATH

% better definition for :=
\mathtoolsset{centercolon}

%space around equations (with glue)
\AtBeginDocument{%
% DEFAULT:
%  \abovedisplayskip=12pt plus 3pt minus 9pt
%  \abovedisplayshortskip=0pt plus 3pt
%  \belowdisplayskip=12pt plus 3pt minus 9pt
%  \belowdisplayshortskip=7pt plus 3pt minus 4pt
% HALF:
%  \abovedisplayskip=6pt plus 2pt minus 4pt
%  \abovedisplayshortskip=0pt plus 3pt
%  \belowdisplayskip=6pt plus 2pt minus 4pt
%  \belowdisplayshortskip=4pt plus 2pt minus 2pt
}

%END

%BEGIN THEOREMS
% break after title
\theoremstyle{break}
\theorembodyfont{\normalfont}
%syntax: \newtheorem{<inner_name>}[<common_counter>]{<printed_name>}[<refence_counter>]

\newtheorem{theorem}{Theorem}[chapter]
% \newtheorem{satz}[theorem]{Satz}
\newtheorem{proposition}[theorem]{Proposition}
\newtheorem{lemma}[theorem]{Lemma}
\newtheorem{definition}[theorem]{Definition}
\newtheorem{corollary}[theorem]{Corollary}
% \newtheorem{korollar}[theorem]{Korollar}

\theoremindent0.7cm
\theoremsymbol{\ensuremath{\square}}
\theoremstyle{plain}
\newtheorem*{proof}{Proof}
% \newtheorem*{beweis}{Beweis}

\theoremsymbol{}
\theoremindent0cm
\newtheorem{example}[theorem]{Example}
\renewtheorem*{example*}{Example}
% \newtheorem{beispiel}[theorem]{Beispiel}
% \renewtheorem*{beispiel*}{Beispiel}

\newtheorem*{remark}{Remark}
% \newtheorem*{bemerkung}{Bemerkung}
%END

%BEGIN LISTINGS
\makeatletter
%%inline listings in description items; it's a modified version of \lstinline
\newcommand\lstinlineitem[1][]{%
  \setbox0=\hbox\bgroup % \lstinline has \leavevmode\bgroup
  \aftergroup\finish@off@lstinlineitem % do something after building the box
  \def\lst@boxpos{b}%
  \lsthk@PreSet\lstset{flexiblecolumns,#1}%
  \lsthk@TextStyle
  \@ifnextchar\bgroup{\afterassignment\lst@InlineG \let\@let@token}%
  \lstinline@}
\def\finish@off@lstinlineitem{\item[\usebox0]} % output the \item
\makeatother


\lstloadlanguages{C++}
\lstset{
	numbers=left, 
	numberstyle=\tiny, 
	basicstyle=\ttfamily,
	morecomment=[l][commentstyle]{//},
	commentstyle={\tiny\itshape},
	% highlight between @..@
	moredelim=**[is][\color{Dgray}]{@}{@},	% highlight between @..@
	aboveskip=\smallskipamount,
	belowskip=\smallskipamount,
	mathescape=true,
	breaklines,
	captionpos=b,
}
\usepackage{scrhack}									% ignores warning from listings
%END

%BEGIN HYPERREF
\hypersetup{
  colorlinks=true,
  linkcolor=black,
  urlcolor=black,
  citecolor=black,
  pdftitle=Integration und Adjungierung Lipschitzstetiger ODEs mit Anwendung in der Datenassimilierung,
  pdfauthor=Ben Lenser,
}
%END

%BEGIN TIKZ

%compat
\pgfplotsset{compat=1.9}


% output folder
\tikzsetexternalprefix{img/tikz/}
\tikzexternalize

% make externalized images subject to draft option
\pgfkeys{/pgf/images/include external/.code=\includegraphics{#1}}

% define table read from image folder
% \let\OLDpgfplotstableread\pgfplotstableread
% \renewcommand{\pgfplotstableread}[1]{\OLDpgfplotstableread{img/#1}}

%global options for pgfplots
\pgfplotsset{legend cell align=left}
%END

%BEGIN DEVELOPEMENT
%filter dest warnings
%END

