\chapter{Source}\label{sec:source}
\section[Public Methods]{The \texorpdfstring{\src{public}}{Public} Methods}
\lstset{
		aboveskip=0pt,
		belowskip=0pt,
		columns=flexible,
	}
\begin{lstlisting}[numbers=none]
PlanC(Vector &c, Vector &b, Matrix &Z, Matrix &L, Matrix &J,Matrix &Y, Vector &xo)
\end{lstlisting}
This constructor takes all necessary information to build an abs-normal form without further calculations.
\begin{lstlisting}[numbers=none]
PlanC(size_t n, size_t m, size_t s)
\end{lstlisting}
This constructor only takes the dimensions of the expected abs-normal form. This is used to internally in the ADOL-C constructor.
\begin{lstlisting}[numbers=none]
PlanC(const PlanC &other)
\end{lstlisting}
This is the copy constructor that copies all internal data.
\begin{lstlisting}[numbers=none]
PlanC(size_t n,size_t m,const Vector &x,simple_function func)
\end{lstlisting}
This is the main constructor for users. See \ref{sec:implementation} for details.
\begin{lstlisting}[numbers=none]
~PlanC()
\end{lstlisting}
This is the destructor.
\begin{lstlisting}[numbers=none]
void insertRow(size_t row, double* in)
\end{lstlisting}
This method provides the interface for ADOL-C. It fills the extended Jacobian \src{Big} row wise. 
\begin{lstlisting}[numbers=none]
void factorize()
\end{lstlisting}
This performs an LU-factorization of the lower left part ($J$) of the extended Jacobian.
\clearpage
\begin{lstlisting}[numbers=none]
is_factorized()
\end{lstlisting}
This is a getter that returns if the LU-factorization has been performed.
\begin{lstlisting}[numbers=none]
PlanC linear_combination(double alpha, double beta)
\end{lstlisting}
Calling this method creates a new \src{PlanC} object that represents the linear combination $\alpha x + \beta F$. In section  \ref{sec:implementation} this method is described in more detail. 
\begin{lstlisting}[numbers=none]
Vector integrate(const Vector v)
\end{lstlisting}
This method calculates the exact value of $\int_{0}^{-\frac12} \pl_{\rx}F(t\cdot v)\d t$.
\begin{lstlisting}[numbers=none]
Vector integrate_simple(Vector v, const double h)
\end{lstlisting}
This function integrates $\int_{0}^{-\frac12} \pl_{\rx}F(t\cdot v)\d t$ assuming that $\pl_{\rx} F(t\cdot v)$ is linear w.r.t. $t$.
\begin{lstlisting}[numbers=none]
Vector eval(const Vector &x)
\end{lstlisting}
This method evaluates the piecewise linearization $\pl_{\rx} F(\cdot)$ at point \src{x}.
\begin{lstlisting}[numbers=none]
Vector eval_F(const Vector &x)
\end{lstlisting}
Evaluates the underlying functions $F$ at point \src{x}. If there is no underlaying function, this call is identical to \src{eval(x)}.
\begin{lstlisting}[numbers=none]
Vector findRoot(double TOL)
\end{lstlisting}
Solves the fix point iteration $z_+ = S |z| + \hat c$.
\begin{lstlisting}[numbers=none]
Vector solve(const Vector &y,bool verbose, double TOL)
\end{lstlisting}
This is a method to solve the equation $\pl_{\rx} F(x) = y$ to a tolerance \src{TOL} with the unfolded Newton. See equation \eqref{eq:unfolded_newton} for details.
\begin{lstlisting}[numbers=none]
Matrix gen_jac(const Vector &x, const Vector &v)
\end{lstlisting}
This builds the generalized Jacobian in the direction of $v$. The equation \src{gen_jac(x,v) == jac(x+v)} (abusing notation) holds if \src{jac(x+v)} is well defined.
\begin{lstlisting}[numbers=none]
Matrix jac(const Vector &x)
\end{lstlisting}
This builds the Jacobian at point \src{x}. This is only viable for non-critical \src{x} (meaning all components of the signature vector $\sigma$ are nonzero).
\begin{lstlisting}[numbers=none]
Matrix solve_ode(const double N, const Vector &x0, const double t0, const double t_end, bool verbose, double TOL)
\end{lstlisting}
Solves the ordinary differential equation $\dot x = F(x)$. See algorithm \ref{alg:solveODE} for details.
\begin{lstlisting}[numbers=none]
Matrix solve_ode_simple(double N, const Vector &x0, double t_0,double t_end, bool verbose = false)
\end{lstlisting}
Solves the ODE $\dot x = F(x)$ with the explicit midpoint rule $\cx - \hx = h F(\cx + \tfrac12 h F(\cx))$.
\begin{lstlisting}[numbers=none]
friend ostream& operator <<(ostream &os, const PlanC &p)
\end{lstlisting}
This overrides the out stream operator \src{<<}. All relevant components are printed.
% \begin{description}
% \lstinlineitem{PlanC(Vector &c, Vector &b, Matrix &Z, Matrix &L, Matrix &J,Matrix &Y, Vector &xo)} This constructor takes all necessary information to build an abs-normal form without further calculations.
% \lstinlineitem{PlanC(size_t n, size_t m, size_t s)} This constructor only takes the dimensions of the expected abs-normal form. This is used to internally in the ADOL-C constructor.
% \lstinlineitem{PlanC(const PlanC &other)} This is the copy constructor that copies all internal data.
% \lstinlineitem{PlanC(size_t n,size_t m,const Vector &x,simple_function func)} This is the main constructor for users. See \ref{sec:implementation} for details
% \lstinlineitem{~PlanC()} This is the destructor.
% \lstinlineitem{void insertRow(size_t row, double* in)} This method provides the interface for ADOL-C. It fills the extended Jacobian \src{Big} row wise. 
% \lstinlineitem{void factorize()} This performs an LU-factorization of the lower left part ($J$) of the extended Jacobian.
% \lstinlineitem{is_factorized()} This is a getter that returns if the LU-factorization has been performed.
% \lstinlineitem{PlanC linear_combination(double alpha, double beta)} Calling this method creates a new \src{PlanC} object that represents the linear combination $\alpha x + \beta F$. In section  \ref{sec:implementation} this method is described in more detail. 
% \lstinlineitem{Vector integrate(const Vector v)} This method calculates the exact value of $\int_{0}^{-\frac12} \pl_{\rx}F(t\cdot v)\d t$ 
% \lstinlineitem{Vector integrate_simple(Vector v, const double h)} This function integrates $\int_{0}^{-\frac12} \pl_{\rx}F(t\cdot v)\d t$ assuming that $\pl_{\rx} F(t\cdot v)$ is linear w.r.t. $t$.
% \lstinlineitem{Vector eval(const Vector &x)} This method evaluates the piecewise linearization $\pl_{\rx} F(\cdot)$ at point \src{x}.
% \lstinlineitem{Vector eval_F(const Vector &x)} Evaluates the underlaying functions $F$ at point \src{x}. If there is no underlaying function, this call is identical to \src{eval(x)}.
% \lstinlineitem{Vector findRoot(double TOL)} Solves the fix point iteration $z_+ = S |z| + \hat c$.
% \lstinlineitem{Vector solve(const Vector &y,bool verbose, double TOL)} This is \, a method to solve the equation $\pl_{\rx} F(x) = y$ to a tolerance \src{TOL} with a Newton variant. See equation \eqref{eq:plannewton} for details.
% \lstinlineitem{Matrix gen_jac(const Vector &x, const Vector &v)} This builds the generalized Jacobian in the direction of $v$. The equation \src{gen_jac(x,v) == jac(x+v)} (abusing notation) holds if \src{jac(x+v)} is well defined. 
% \lstinlineitem{Matrix jac(const Vector &x)} This builds the Jacobian at point \src{x}. This is only viable for non-critical \src{x} (meaning all components of the signature vector $\sigma$ are nonzero).
% \lstinlineitem{Matrix solveODE(const double N, const Vector &x0, const double t0, const double t_end, bool verbose, double TOL)} Solves the ordinary differential equation $\dot x = F(x)$. See algorithm \ref{alg:solveODE} for details.
% \lstinlineitem{Matrix solve_ode_simple(double N, const Vector &x0, double t_0,double t_end, bool verbose = false)} Solves the ODE $\dot x = F(x)$ with the explicit midpoint rule $\cx - \hx = h F(\cx + \tfrac12 h F(\cx))$.
% \lstinlineitem{friend ostream& operator <<(ostream &os, const PlanC &p)} This overrides the outstream operator \src{<<}. All relevant components are printed.
% \end{description}
\section[Private Fields]{The \texorpdfstring{\src{private}}{Private} Fields}
\begin{description}
\lstinlineitem{const int MAX_IT} The maximal number of iterations in the inner loop of \src{SolveODE}.
\lstinlineitem{size_t n,m,s} The dimensions of the linearization. \src{n} is the size of the input of $F$, \src{m} the size of the output and \src{s} is the number of switching variables.
\lstinlineitem{bool factorized} The flag that shows if the LU-factorization of $J$ has been performed.
\lstinlineitem{Vector c,b} The vector components of the abs-normal form.
\lstinlineitem{Matrix Z,L,J,Y} The matrix components of the abs-normal form.
\lstinlineitem{Matrix Big} The extended Jacobian consisting of \src{Z,L,J,Y}. 
\lstinlineitem{Vector idx} The permutation vector belonging to the LU-factorized $J$. 
\lstinlineitem{SubMatrix S} The Schur-complement of the abs-normal form. This is only available if \src{factorized==true}. This is a virtual matrix that accesses the memory of \src{Big} (the upper right block matrix of it).
\lstinlineitem{SubMatrix LUJ} The lower left block of \src{Big} that is representing the LU-factorization of $J$. 
\lstinlineitem{simple_function F} The function pointer to the underlying function for the piecewise linearization. This may be \src{std::nullptr} if the linearization is not dependent on an external function.
\lstinlineitem{Vector xo} The base point of the linearization. If no base point is given, this point is initialized with $0_n$.
\lstinlineitem{Vector fxo} The function evaluated at \src{xo}. According to equation \eqref{eq:plf} this is the same for the linearization and the underlying function.
\end{description}
\section[Private Methods]{The \texorpdfstring{\src{private}}{Private} Methods}
\begin{description}
\lstinlineitem{void updateLinearization(const Vector &x)} Recalculates the piecewise linearization at point \src{x}. If there is no underlying function, this only changes \src{xo} and \src{fxo}. Otherwise a new linearization is calculated with ADOL-C.
\lstinlineitem{Vector getCHat(const Vector &y)} This calculates $\hat c = c - Z J^{-1}b$. It triggers a \src{factorize()}.
\lstinlineitem{Vector calculate_z(const Vector &x)} solves $z = c + Zx + L|z|$ for $z$ using the strict triangular structure of $L$.
\lstinlineitem{Vector calculate_x(Vector z, Vector y)} solves $y = b + Jx + Y|z|$ for $x$. This triggers a \src{factorize()} and uses the LU-factorization of $J$. 
\lstinlineitem{Vector eval(Vector z, Vector x)} evaluates the piecewise linearization using the already calculated value for \src{z}.
\end{description}