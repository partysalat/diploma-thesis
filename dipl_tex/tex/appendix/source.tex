\chapter{Quellcode Dokumentation}\label{sec:source}
Die Dokumentation ist in Teilen aus der Diplomarbeit von Paul Boeck aus \cite{boeck14} übernommen und mit eigenen Methoden erweitert worden.
\section[Public Methoden]{\texorpdfstring{\src{public}}{Public} Methoden}
\lstset{
		aboveskip=0pt,
		belowskip=0pt,
		columns=flexible,
	}
\begin{lstlisting}[numbers=none]
PlanC(vec &c, vec &b, mat &Z, mat &L, mat &J,mat &Y, vec &xo)
\end{lstlisting}
Mit diesem Konstruktor kann eine Abs-Normal Form ohne AD erstellt werden.
\begin{lstlisting}[numbers=none]
PlanC(size_t n, size_t m, size_t s)
\end{lstlisting}
Dieser Konstruktor wird intern benutzt und benötigt nur die Dimensionen als auch die Anzahl der switching Variablen.
\begin{lstlisting}[numbers=none]
PlanC(const PlanC &other)
\end{lstlisting}
Erstellt einen Klon eines gegebenen Plan-C Objektes
\begin{lstlisting}[numbers=none]
PlanC(size_t n,size_t m,const vec &x,simple_function func)
\end{lstlisting}
Dies ist der am meisten benutzte Konstruktor, er erstellt aus einer gegebenen Funktion eine Abs-Normal Form und rechnet mit ihr. Siehe \ref{sec:impl:planc} für Details
\begin{lstlisting}[numbers=none]
~PlanC()
\end{lstlisting}
Dies ist der Destruktor.
\begin{lstlisting}[numbers=none]
void insertRow(size_t row, double* in)
\end{lstlisting}
Diese Methode nimmt den Pointer auf ein Datenarray, welcher von ADOL-C erstellt wurde und kopiert die Daten zeilenweise in das interne Datenmodell.
\begin{lstlisting}[numbers=none]
void factorize()
\end{lstlisting}
Führt eine LU Faktorisierung der Matrix $J$ der Abs-Normal Form durch
% \clearpage
\begin{lstlisting}[numbers=none]
is_factorized()
\end{lstlisting}
Gibt ein \texttt{boolean} zurück, welcher beschreibt, ob die LU Faktorisierung für die aktuelle Abs-Normal Form bereits durchgeführt wurde.
\begin{lstlisting}[numbers=none]
PlanC linear_combination(double alpha, double beta)
\end{lstlisting}
Erstellt ein neues geshiftetes Plan-C Objekt, welches $\alpha x + \beta F$ beschreibt. Sie wird bei der Berechnung der ODE in Algorithmus \ref{alg:solveOde} verwendet.
% Calling this method creates a new \src{PlanC} object that represents the linear combination $\alpha x + \beta F$. In section  \ref{sec:implementation} this method is described in more detail. 
\begin{lstlisting}[numbers=none]
vec integrate(const vec v)
\end{lstlisting}
Implementation der Prozedur \ref{alg:integratePL}
\begin{lstlisting}[numbers=none]
vec eval(const vec &x)
\end{lstlisting}
Wertet die gegenwärtige Abs-Normal Form an Stelle \texttt{x} aus
\begin{lstlisting}[numbers=none]
vec eval_F(const vec &x)
\end{lstlisting}
Wertet die übergebene Funktion $F$ an der Stelle \texttt{x} aus, falls diese nicht definiert ist, wird stattdessen \texttt{eval(x)} ausgeführt.
\begin{lstlisting}[numbers=none]
vec solve(const vec &y,bool verbose, double TOL,int type=0)
\end{lstlisting}
Diese Methode wird benutzt, um Gleichungen der Art $F(\xo)+\Delta F(\xo,\Delta x) = y$ mit Toleranz \src{TOL} zu lösen. Dabei kann der Algorithmentyp \texttt{type} mit übergeben werden, wobei \text{type}=0 lösen auf dem CPL, \text{type}=1 lösen auf dem UPL und \text{type}=2 lösen mittels des stückweise linearen Newton bedeutet. Standardmäßig wird \texttt{type}=0 gesetzt.    
% This is a method to solve the equation $\pl_{\rx} F(x) = y$ to a tolerance \src{TOL} with the unfolded Newton. See equation \eqref{eq:unfolded_newton} for details.
\begin{lstlisting}[numbers=none]
mat gen_jac(const vec &x, const vec &dx, bool disableKinkPrediction=false)
\end{lstlisting}
Berechnet den verallgemeinerten Gradienten in Richtung \src{dx} an der Stelle \src x. Versucht intern den Gradienten an der Stelle \src{x +0.5 tau dx} zu bestimmen und berechnet dafür den kritischen Multiplikator \src{tau}. Falls in Richtung \src{dx} ein Valley tracing mode ist, wird der Gradient über polynomial escape bestimmt. Lässt sich mittels der Flag \src{disableKinkPrediction} abschalten, wobei intern \src{tau = 1} gesetzt wird.
% This builds the generalized Jacobian in the direction of $v$. The equation \src{gen_jac(x,v) == jac(x+v)} (abusing notation) holds if \src{jac(x+v)} is well defined.
\begin{lstlisting}[numbers=none]
mat solve_ode(const double N, const vec &x0, const double t0, const double t_end, bool verbose, double TOL)
\end{lstlisting}
Läst die gewöhnliche Differentialgleichung  $\dot x = F(x)$ mittels stückweiser Linearisierung. Implementierung des Algorithmus \ref{alg:solveOde}
% Solves the ordinary differential equation $\dot x = F(x)$. See algorithm \ref{alg:solveODE} for details.
\begin{lstlisting}[numbers=none]
mat solve_ode_midpoint_implicit(double N, const vec &x0, double t_0,double t_end, bool verbose = false)
\end{lstlisting}
Löst die ODE  $\dot x = F(x)$ mit der impliziten Mittelpunktsregel
% Solves the ODE $\dot x = F(x)$ with the explicit midpoint rule $\cx - \hx = h F(\cx + \tfrac12 h F(\cx))$.
\begin{lstlisting}[numbers=none]
mat solve_ode_midpoint_explicit(double N, const vec &x0, double t_0,double t_end, bool verbose = false)
\end{lstlisting}
Löst die ODE  $\dot x = F(x)$ mit der expliziten Mittelpunktsregel
\begin{lstlisting}[numbers=none]
mat projState2ObsLinear(const vec &tState, const vec &tObs,const mat &xState)
\end{lstlisting}
Projeziert die Werte \src{xState} des Zustandsraumes mit den Stützstellen \src{tState} in den Raum der Observierungen mit den Stützstellen \src{tObs} mittels linearer Interpolation.
\begin{lstlisting}[numbers=none]
mat projObs2StateLinear(const vec &tState, const vec &tObs,const mat &xObs)
\end{lstlisting}
Projeziert die Werte \src{xObs} des Observierungsraumes mit den Stützstellen \src{tObs} in den Zustandsraum mit den Stützstellen \src{tState} mittels linearer Interpolation.
\begin{lstlisting}[numbers=none]
mat projState2ObsSpline(const vec &tState, const vec &tObs,const mat &xState)
\end{lstlisting}
Projeziert die Werte \src{xState} des Zustandsraumes mit den Stützstellen \src{tState} in den Raum der Observierungen mit den Stützstellen \src{tObs} mittels Spline Interpolation.
\begin{lstlisting}[numbers=none]
mat projObs2StateSpline(const vec &tState, const vec &tObs,const mat &xObs)
\end{lstlisting}
Projeziert die Werte \src{xObs} des Observierungsraumes mit den Stützstellen \src{tObs} in den Zustandsraum mit den Stützstellen \src{tState} mittels Spline Interpolation.
%
\begin{lstlisting}[numbers=none]
void calc_kink_partial(double tCheck, double tHat, vec xCheck,  xHat, const vec &d, mat &A, mat &B)
\end{lstlisting}
Berechnet den gewichteten Gradienten nach Prozedur \ref{alg:kinkPartials} im Intervall \src{xCheck} bis \src{xHat} und speichert die Werte in \src{A} und \src{B}.
\begin{lstlisting}[numbers=none]
double cost_functional_data_assimilation(const vec &tState, const vec &x0, const vec &tObs, const mat &xObs, size_t type, bool verbose=false, mat sol=mat(0,0))
\end{lstlisting}
Berechnet das Kostenfunktional \eqref{eq:costfunctional} mittels Trapezregel wie im Abschnitt \ref{sec:implementation:costfunctional} beschrieben. \src{tState} ist dabei die Zeitdiskretisierung des Zustandsraumes, \src{x0} der Auswertungspunkt, \src{tObs} die Zeitdiskretisierung des Observierungsraumes, \src{xObs} die Observierungen, \src{type} gibt die ODE Methode an (0:PL,1:Implizit,2:Explizit), verbose gibt zusätzliche Informationen in der Konsole aus. Optional kann die ODE Lösung \src{sol} mit übergeben werden, sodass diese nicht noch einmal berechnet werden muss.  
\begin{lstlisting}[numbers=none]
vec jac_data_assimilation_pl(const vec &tState, const vec &x0, const vec &tObs, const mat &xObs, vec direction, double TOL, bool verbose = false, mat sol=mat(0,0))
\end{lstlisting}
Berechnet den Gradienten des Kostenfunktionals nach Prozedur \ref{alg:jacDataAssimilation}. Für Parameter siehe \src{cost_functional_data_assimilation}.
\begin{lstlisting}[numbers=none]
vec jac_data_assimilation_implicit_midpoint(const vec &tState, const vec &x0, const vec &tObs, const mat &xObs, vec direction, double TOL, bool verbose = false, mat sol=mat(0,0))
\end{lstlisting}
Analog zu \src{jac_data_assimilation_pl} nur mit impliziter Mittelpunktsregel berechnet.
\begin{lstlisting}[numbers=none]
vec jac_data_assimilation_explicit_midpoint(const vec &tState, const vec &x0, const vec &tObs, const mat &xObs, vec direction, double TOL, bool verbose = false, mat sol=mat(0,0))
\end{lstlisting}
Analog zu \src{jac_data_assimilation_pl} nur mit expliziter Mittelpunktsregel berechnet.
\begin{lstlisting}[numbers=none]
vec solve_data_assimilation_bfgs(const vec &tState, const vec &x0, const vec &tObs, const mat &xObs, double TOL, mat &iterationSteps, bool verbose = false, int type=0)
\end{lstlisting}
Führt die Optimierung des Kostenfunkionals mittels BFGS Algorithmus (siehe \ref{alg:bfgs}) aus. Die Variable \src{iterations} kann mit übergeben werden in welcher die Iterationsschritte gespeichert werden. Mit \src{type} kann wieder entschieden werden, welche Methode der Optimierung zu Grunde liegt.
\begin{lstlisting}[numbers=none]
void useLUFactorization()
\end{lstlisting}
Benutzt die LU Faktorisierung zum Lösen linearer Gleichungssysteme
\begin{lstlisting}[numbers=none]
void useNewtonSolver(size_t type)
\end{lstlisting}
Setzt den Typ des Newton Solvers.\src{(0:OPL=default, 1:CPL, 2:UPL)}
\begin{lstlisting}[numbers=none]
void useLUFactorization(bool flag=true)
\end{lstlisting}
Wenn auf \src{flag=true}, wird versucht mit der LU Faktorisierung zu rechnen. 
\begin{lstlisting}[numbers=none]
void out()
\end{lstlisting}
Gibt das \src{Plan-C} Objekte in der Konsole aus.
\section[Private Variablen]{\texorpdfstring{\src{private}}{Private} Variablen}
\begin{description}
\lstinlineitem{const int MAX_IT} Die Maximale Anzahl innerer Schleifen für \src{solve_ode}.
\lstinlineitem{size_t n,m,s} Die Dimension der Linearisierung. \src{n} ist dabei de Größe des Definitionsberechs von $F$, \src{m} die Größe des Wertebereichs und \src{s} die Anzahl der switching Variablen.
\lstinlineitem{bool factorized} Flag die zeigt, ob die aktuelle Abs-Normal form LU-faktorisiert wurde
\lstinlineitem{vec c,b} Die Vektor Komponenten der Abs-Normal Form
\lstinlineitem{mat Z,L,J,Y} Die Matrix Komponenten der Abs-Normal Form. 
\lstinlineitem{vec idx} Permutationsvektor der zur LU-Faktorisierung von $J$gehört.
\lstinlineitem{mat S} Das Schur Komplement der Abs-Normal Form die bei der Methode \src{factorize()} abgespeichert wird
\lstinlineitem{mat LUJ} Repräsentiert die LU-Faktorisierung von $J$
\lstinlineitem{simple_function F} Der Function pointer auf die übergebene Funktion
\lstinlineitem{vec xo} Entwicklungspunkt der aktuellen Abs-Normal Form, falls nicht übergeben wird er auf $0_n$ gesetzt.
\lstinlineitem{vec fxo} Der am Punkt \src{xo} ausgewertete Funktionswert. An diesem Punkt ist die Auswertung der Linearisierung und der Funktion identisch.
\lstinlineitem{size_t newtonSolverType} Speichert den gewünschten Newton Solver.
\lstinlineitem{bool useLUFactorizationFlag} Speichert, ob die LU Faktorisierung benutzt werden soll

\end{description}
\section[Private Methoden]{\texorpdfstring{\src{private}}{Private} Methoden}
\begin{description}
\lstinlineitem{void updateLinearization(const vec &x)} Erstellt am übergebenen Punkt eine neue Linearisierung der übergebenen Funktion. Falls keine Funktion übergeben wurde wird nur \src{xo} und \src{fxo} angepasst.
\lstinlineitem{vec getCHat(const vec &y)} Berechnet $\hat c = c - Z J^{-1}b$. Wird im Falle von \src{factorize()} aufgerufen.
\lstinlineitem{vec calculate_z(const vec &x)} Berechnet $z = c + Zx + L|z|$.
\lstinlineitem{vec calculate_x(vec z, vec y)} Löst $y = b + Jx + Y|z|$ nach $x$. Dies ruft \src{factorize()} auf und benutzt die LU Faktorisierung von $J$.(falls \src{useLUFactorization=true} ) 
\lstinlineitem{vec eval(vec z, vec x)} Wertet die Abs-Normal Form am Punkt \src{x} mit bereits berechnetem \src{z} aus. 
\end{description}