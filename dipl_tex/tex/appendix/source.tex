\chapter{Quellcode Dokumentation}\label{sec:source}
Die Dokumentation ist in Teilen aus der Diplomarbeit von Paul Boeck aus \cite{boeck14} übernommen und mit eigenen Methoden erweitert worden.
\section[Public Methods]{\texorpdfstring{\src{public}}{Public} Methoden}
\lstset{
		aboveskip=0pt,
		belowskip=0pt,
		columns=flexible,
	}
\begin{lstlisting}[numbers=none]
PlanC(vec &c, vec &b, mat &Z, mat &L, mat &J,mat &Y, vec &xo)
\end{lstlisting}
Mit diesem Konstruktor kann eine Abs-Normal Form ohne AD erstellt werden.
\begin{lstlisting}[numbers=none]
PlanC(size_t n, size_t m, size_t s)
\end{lstlisting}
Dieser Konstruktor wird intern benutzt und benötigt nur die Dimensionen als auch die Anzahl der switching Variablen.
\begin{lstlisting}[numbers=none]
PlanC(const PlanC &other)
\end{lstlisting}
Erstellt einen Klon eines gegebenen Plan-C Objektes
\begin{lstlisting}[numbers=none]
PlanC(size_t n,size_t m,const vec &x,simple_function func)
\end{lstlisting}
Dies ist der am meisten benutzte Konstruktor, er erstellt aus einer gegebenen Funktion eine Abs-Normal Form und rechnet mit ihr. Siehe \ref{sec:impl:planc} für Details
\begin{lstlisting}[numbers=none]
~PlanC()
\end{lstlisting}
Dies ist der Destruktor.
\begin{lstlisting}[numbers=none]
void insertRow(size_t row, double* in)
\end{lstlisting}
Diese Methode nimmt den Pointer auf ein Datenarray, welcher von ADOL-C erstellt wurde und kopiert die Daten zeilenweise in das interne Datenmodell.
\begin{lstlisting}[numbers=none]
void factorize()
\end{lstlisting}
Führt eine LU Faktorisierung der Matrix $J$ der Abs-Normal Form durch
% \clearpage
\begin{lstlisting}[numbers=none]
is_factorized()
\end{lstlisting}
Gibt ein \texttt{boolean} zurück, welcher beschreibt, ob die LU Faktorisierung für die aktuelle Abs-Normal Form bereits durchgeführt wurde.
\begin{lstlisting}[numbers=none]
PlanC linear_combination(double alpha, double beta)
\end{lstlisting}
Erstellt ein neues geshiftetes Plan-C Objekt, welches $\alpha x + \beta F$ beschreibt. Sie wird bei der Berechnung der ODE in Algorithmus \ref{alg:solveOde} verwendet.
% Calling this method creates a new \src{PlanC} object that represents the linear combination $\alpha x + \beta F$. In section  \ref{sec:implementation} this method is described in more detail. 
\begin{lstlisting}[numbers=none]
vec integrate(const vec v)
\end{lstlisting}
Implementation der Prozedur \ref{alg:integratePL}
\begin{lstlisting}[numbers=none]
vec eval(const vec &x)
\end{lstlisting}
Wertet die gegenwärtige Abs-Normal Form an Stelle \texttt{x} aus
\begin{lstlisting}[numbers=none]
vec eval_F(const vec &x)
\end{lstlisting}
Wertet die übergebene Funktion $F$ an der Stelle \texttt{x} aus, falls diese nicht definiert ist, wird stattdessen \texttt{eval(x)} ausgeführt.
\begin{lstlisting}[numbers=none]
vec solve(const vec &y,bool verbose, double TOL,int type=0)
\end{lstlisting}
Diese Methode wird benutzt, um Gleichungen der Art $F(\xo)+\Delta F(\xo,\Delta x) = y$ mit Toleranz \src{TOL} zu lösen. Dabei kann der Algorithmentyp \texttt{type} mit übergeben werden, wobei \text{type}=0 lösen auf dem CPL, \text{type}=1 lösen auf dem UPL und \text{type}=2 lösen mittels des stückweise linearen Newton bedeutet. Standardmäßig wird \texttt{type}=0 gesetzt.    
% This is a method to solve the equation $\pl_{\rx} F(x) = y$ to a tolerance \src{TOL} with the unfolded Newton. See equation \eqref{eq:unfolded_newton} for details.
\begin{lstlisting}[numbers=none]
mat gen_jac(const vec &x, const vec &dx, bool disableKinkPrediction=false)
\end{lstlisting}
Berechnet den verallgemeinerten Gradienten in Richtung \src{dx} an der Stelle \src x. Versucht intern den Gradienten an der Stelle \src{x +0.5 tau dx} zu bestimmen und berechnet dafür den kritischen Multiplikator \src{tau}. Falls in Richtung \src{dx} ein Valley tracing mode ist, wird der Gradient über polynomial escape bestimmt. Lässt sich mittels der Flag \src{disableKinkPrediction} abschalten, wobei intern \src{tau = 1} gesetzt wird.
% This builds the generalized Jacobian in the direction of $v$. The equation \src{gen_jac(x,v) == jac(x+v)} (abusing notation) holds if \src{jac(x+v)} is well defined.
\begin{lstlisting}[numbers=none]
mat solve_ode(const double N, const vec &x0, const double t0, const double t_end, bool verbose, double TOL)
\end{lstlisting}
Läst die gewöhnliche Differentialgleichung  $\dot x = F(x)$ mittels stückweiser Linearisierung. Implementierung des Algorithmus \ref{alg:solveOde}
% Solves the ordinary differential equation $\dot x = F(x)$. See algorithm \ref{alg:solveODE} for details.
\begin{lstlisting}[numbers=none]
mat solve_ode_midpoint_implicit(double N, const vec &x0, double t_0,double t_end, bool verbose = false)
\end{lstlisting}
Löst die ODE  $\dot x = F(x)$ mit der impliziten Mittelpunktsregel
% Solves the ODE $\dot x = F(x)$ with the explicit midpoint rule $\cx - \hx = h F(\cx + \tfrac12 h F(\cx))$.
\begin{lstlisting}[numbers=none]
mat solve_ode_midpoint_explicit(double N, const vec &x0, double t_0,double t_end, bool verbose = false)
\end{lstlisting}
Löst die ODE  $\dot x = F(x)$ mit der expliziten Mittelpunktsregel
\begin{lstlisting}[numbers=none]
mat projState2ObsLinear(const vec &tState, const vec &tObs,const mat &xState)
\end{lstlisting}
Projeziert die Werte \src{xState} des Zustandsraumes mit den Stützstellen \src{tState} in den Raum der Observierungen mit den Stützstellen \src{tObs} mittels linearer Interpolation.
\begin{lstlisting}[numbers=none]
mat projObs2StateLinear(const vec &tState, const vec &tObs,const mat &xObs)
\end{lstlisting}
Projeziert die Werte \src{xObs} des Observierungsraumes mit den Stützstellen \src{tObs} in den Zustandsraum mit den Stützstellen \src{tState} mittels linearer Interpolation.
\begin{lstlisting}[numbers=none]
mat projState2ObsSpline(const vec &tState, const vec &tObs,const mat &xState)
\end{lstlisting}
Projeziert die Werte \src{xState} des Zustandsraumes mit den Stützstellen \src{tState} in den Raum der Observierungen mit den Stützstellen \src{tObs} mittels Spline Interpolation.
\begin{lstlisting}[numbers=none]
mat projObs2StateSpline(const vec &tState, const vec &tObs,const mat &xObs)
\end{lstlisting}
Projeziert die Werte \src{xObs} des Observierungsraumes mit den Stützstellen \src{tObs} in den Zustandsraum mit den Stützstellen \src{tState} mittels Spline Interpolation.
to be continued...
\begin{lstlisting}[numbers=none]
void out()
\end{lstlisting}
Macht eine Konsolenausgabe des \src{Plan-C} Objektes
\section[Private Fields]{The \texorpdfstring{\src{private}}{Private} Fields}
\begin{description}
\lstinlineitem{const int MAX_IT} The maximal number of iterations in the inner loop of \src{SolveODE}.
\lstinlineitem{size_t n,m,s} The dimensions of the linearization. \src{n} is the size of the input of $F$, \src{m} the size of the output and \src{s} is the number of switching variables.
\lstinlineitem{bool factorized} The flag that shows if the LU-factorization of $J$ has been performed.
\lstinlineitem{vec c,b} The vector components of the abs-normal form.
\lstinlineitem{mat Z,L,J,Y} The matrix components of the abs-normal form.
\lstinlineitem{mat Big} The extended Jacobian consisting of \src{Z,L,J,Y}. 
\lstinlineitem{vec idx} The permutation vector belonging to the LU-factorized $J$. 
\lstinlineitem{mat S} The Schur-complement of the abs-normal form. This is only available if \src{factorized==true}. This is a virtual matrix that accesses the memory of \src{Big} (the upper right block matrix of it).
\lstinlineitem{mat LUJ} The lower left block of \src{Big} that is representing the LU-factorization of $J$. 
\lstinlineitem{simple_function F} The function pointer to the underlying function for the piecewise linearization. This may be \src{std::nullptr} if the linearization is not dependent on an external function.
\lstinlineitem{vec xo} The base point of the linearization. If no base point is given, this point is initialized with $0_n$.
\lstinlineitem{vec fxo} The function evaluated at \src{xo}. According to equation \eqref{eq:plf} this is the same for the linearization and the underlying function.
\end{description}
\section[Private Methods]{The \texorpdfstring{\src{private}}{Private} Methods}
\begin{description}
\lstinlineitem{void updateLinearization(const vec &x)} Recalculates the piecewise linearization at point \src{x}. If there is no underlying function, this only changes \src{xo} and \src{fxo}. Otherwise a new linearization is calculated with ADOL-C.
\lstinlineitem{vec getCHat(const vec &y)} This calculates $\hat c = c - Z J^{-1}b$. It triggers a \src{factorize()}.
\lstinlineitem{vec calculate_z(const vec &x)} solves $z = c + Zx + L|z|$ for $z$ using the strict triangular structure of $L$.
\lstinlineitem{vec calculate_x(vec z, vec y)} solves $y = b + Jx + Y|z|$ for $x$. This triggers a \src{factorize()} and uses the LU-factorization of $J$. 
\lstinlineitem{vec eval(vec z, vec x)} evaluates the piecewise linearization using the already calculated value for \src{z}.
\end{description}