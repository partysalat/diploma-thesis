\chapter{Theoreme}
% http://www.math.wisc.edu/~rharron/teaching/MAT203/LeibnizRule.pdf
\section{Verallgemeinerte Leibniz Integral Regel}
\begin{theorem}[Leibniz Integral Regel]
\label{thm:leibnizIntegralRule}
Sei $f$ eine Funktion auf einem rechteckigen Gebiet $R=[a,b]\times [c,d]$ und $\frac{\partial f}{\partial y}(x,y)$ ist stetig auf $R$. Dann gilt
\begin{equation}
 \frac{d}{dy}\int_a^b f(x,y)dx = \int_a^b \frac{\partial f}{\partial y}(x,y)dx
\end{equation}
\begin{proof}
 Wir wissen aus dem Satz von Fubini, dass gilt
 \begin{equation}
 \label{eq:leibnizFubiniIntgralRule}
  \frac{d}{dy}\left( \int_c^y \int_a^b \frac{\partial f}{\partial z}(x,z) dx dz\right) = 
  \frac{d}{dy}\left( \int_a^b \int_c^y \frac{\partial f}{\partial z}(x,z) dz dx\right) 
 \end{equation}
Aus dem Hauptsatz der Differential- und Integralrechnung 
\[
 \frac{d}{dt}\left( \int_a^t f(x)dx\right) = f(t)
\]
ergibt sich für die linke Seite von \eqref{eq:leibnizFubiniIntgralRule}, dass
\[
 \frac{d}{dy}\left( \int_c^y \int_a^b \frac{\partial f}{\partial z}(x,z) dx dz\right)= \int_a^b\frac{\partial f}{\partial y}(x,y)
\]
Unter Nutzung einer anderen Version des Fundamentalsatzes der Analysis $\int_a^bF'(x) =F(a) - F(b)$ ergibt sich für die rechte Seite
\[
\begin{aligned}
  \frac{d}{dy}\left( \int_a^b \int_c^y \frac{\partial f}{\partial z}(x,z) dz dx\right) &= \frac{d}{dy} \left(\int_a^b f(x,y) - f(x,c)dx\right)\\
  &= \frac{d}{dy} \left(\int_a^b f(x,y)dx\right)
\end{aligned}
 \]
da $f(x,c)$ unabhängig von $y$ ist und damit die Ableitung nach $y$ verschwindet.
\end{proof}
\end{theorem}

\begin{theorem}[Verallgemeinerte Leibniz Integral Regel]
 Seien die Voraussetzungen wie in Theorem \ref{thm:leibnizIntegralRule} gegeben. Dann gilt
 \begin{equation}
 \label{eq:genLeibnizIntRule}
  \frac{d}{dy}\int_{g_1(y)}^{g_2(y)}f(x,y) dx =\int_{g_1(y)}^{g_2(y)} \frac{\partial f}{\partial y}(x,y) dx + g_2'(y)f(g_2(y),y) - g_1'(y)f(g_1(y),y)
 \end{equation}
 \begin{proof}
  Sei $u = g_2(y)$, $v = g_1(y)$ und $w=y$. Damit folgt für 
  \[
  \begin{aligned}
       F(y) &=  \int_{g_1(y)}^{g_2(y)}f(x,y) dx\\
	    &=  \int_{v}^{u}f(x,w) dx\\
	    &= G(u,v,w)
  \end{aligned}
  \]
und damit mit der Kettenregel
\[
 \frac{dF}{dy} = \frac{\partial G}{\partial u}\frac{\partial u}{\partial y} + \frac{\partial G}{\partial v}\frac{\partial v}{\partial y}+ \frac{\partial G}{\partial w}\frac{\partial w}{\partial y}
\]
da $u, v$ und $w$ von $y$ abhängig sind.
Nun gilt für den ersten Term 
\[
  \frac{\partial G}{\partial u} = \frac{\partial}{\partial u} \int_v^u f(x,w)dx = f(u,w)
\]
mit der Leibnizregel \eqref{thm:leibnizIntegralRule}. Durch Rücksubstitution erhalten wir
\[
 u = g_2(y) \text{ und } \frac{du}{dy} = g_2'(y)
\]
und damit 
\[
  \frac{\partial G}{\partial u}\frac{\partial u}{\partial y}  = f(g_2(y),y)g_2'(y)
\]
Analog folgt der mittlere Term, jedoch entsteht ein negatives Vorzeichen, da
\[
\begin{aligned}
  \frac{\partial G}{\partial v} &= \frac{\partial}{\partial v} \int_v^u f(x,w)dx \\
				&= -\frac{\partial}{\partial v} \int_u^v f(x,w)dx \\
				&= -f(v,w) \\
\end{aligned}
\]
Der letzte Term folgt aus
\[
\begin{aligned}
 \frac{\partial G}{\partial w} &= \frac{\partial}{\partial w} \int_v^u f(x,w)dx  \\
			      &= \int_{v}^{u} \frac{\partial f}{\partial w}(x,w)dx \\
			      &= \int_{v}^{u} \frac{\partial f}{\partial y}(x,y)dx \\
\end{aligned}
 \]
und damit die Behauptung.
 \end{proof}


\end{theorem}
\section{Bouligandableitung und Inkrementfunktion}\label{sec:bouligandAndIncrement}
 Die Bouligandableitung $F'(x;\cdot)$ \eqref{eq:bouligandDerivative} ist positiv homogen, da für $x\in \R^n, d\in \R^n,$ und $\alpha>0$ gilt
 \[
 \begin{aligned}
  F'(x;\alpha d) &=  \lim_{\substack{s\to 0 \\ s>0}} \frac{F(x+s\alpha d)-F(x)}{s} &,s = \frac{\tilde s}{\alpha} \\
			 &=  \lim_{\substack{\sfrac{\tilde s}{\alpha}\to 0 \\ \sfrac{\tilde s}{\alpha}>0}} \alpha \frac{F(x+\tilde s d)-F(x)}{\tilde s} \\
			 &= \alpha \lim_{\substack{\tilde s\to 0 \\ \tilde s>0}} \frac{F(x+\tilde s d)-F(x)}{\tilde s} \\
			 &= \alpha  F'(x, d) 
 \end{aligned}
 \]
Für die Inkrementfunktion $\Delta F(x,\cdot)$ gilt dies nicht mit der Regel \ref{eq:absAdRule}, da 
\[
\begin{aligned}
 \Delta v_i &= \abs(\vo_j+ \alpha \Delta v_j) - \vo_i \\
 &\neq \alpha (\abs(\vo_j+ \Delta v_j) - \vo_i) \\
 \end{aligned}
\]
\section{Ableitung des Integranten des Kostenfunktionals}\label{sec:adjIntegrandCostfunctional}
Die Ableitung von 
\[
H(x) = \|Cx -\xobs\|^2
\]
ergibt sich zu
\[
 \nabla_x H(x) = C^\tr(Cx -\xobs)
\]
da 
\[
\begin{aligned}
H(x) &= \|Cx -\xobs\|^2\\
     &= (Cx -\xobs)^\tr(Cx -\xobs)\\
     &= ((Cx)^\tr -\xobs^\tr)(Cx -\xobs)\\
     &= ((Cx)^\tr Cx -x^\tr C^\tr \xobs - \xobs^\tr Cx + y^\tr y)\\
\end{aligned}
\]
Da für ein Skalar $a^\tr = a$ und $x^\tr C^\tr \xobs$ skalar ist, gilt 
\[
 x^\tr C^\tr \xobs =(x^\tr C^\tr \xobs)^\tr =  \xobs^\tr Cx
\]
und es folgt
\[
\begin{aligned}
H(x) &= (Cx)^\tr Cx -2 x^\tr C^\tr \xobs + y^\tr y\\
\end{aligned}
\]

Mit der Produktregel für den Ausdruck
\[
 \nabla_x ((Cx)^\tr Cx) = C^\tr Cx + C^\tr Cx = 2\cdot C^\tr Cx 
 \]
folgt dann 
\[
\begin{aligned}
 \nabla_x H &= 2\cdot C^\tr Cx -2\cdot C^\tr \xobs\\
	    &= 2\cdot C^\tr (Cx - \xobs)
\end{aligned}
\]
\qed

% \section{Verallgemeinerter MinMod Flux Limiter}
% Der verallgemeinerte MinMod lautet
% \[
%  \Phi_{mg} = \text{minmod}\left( \theta \frac{u_i-u_{i-1}}{\Delta x}, \frac{u_{i+1}-u_{i-1}}{2\Delta x},\theta \frac{u_{i+1}-u_{i}}{\Delta x}\right) ~ \Theta \in [1,2]
% \]
% mit 
% \[
%  \text{minmod}(z_1,z_2,\ldots) = 
%  \begin{cases}
%   \min_j & \text{falls }z_j>0~ \forall j\\
%   \max_j & \text{falls }z_j<0~\forall j\\
%   0 & \text{sonst}
%  \end{cases}
% \]
% 
% \begin{theorem}
%  Der Verallgemeinerte minmod Flux Limiter lässt sich darstellen als:
%  \[
%   \Phi_{mg} = \min(\max(z_1,z_2,z_3),0) + \max(\min(z_1,z_2,z_3),0)
%  \]
% mit $z_1 = \theta \frac{u_i-u_{i-1}}{\Delta x}$, $z_2=\frac{u_{i+1}-u_{i-1}}{2\Delta x}$ und $z_3 = \theta \frac{u_{i+1}-u_{i}}{\Delta x}$
% \end{theorem}
