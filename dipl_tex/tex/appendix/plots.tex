\chapter{Plots}
 In Abschnitt \ref{ch:experiments} wurden möglichst aussagekräftige Plots dargestellt. Jene, die weniger sichtbare Ergebnisse lieferten sind zur besseren Vergleichbarkeit in diesem Anhang untergebracht.
% \section{Rolling Stones}


% \section{LC-Diode}

\section{Shallow Water Equation}\label{sec:sweU2}
\subsection{Anfangswert}
\begin{figure}[H]
\footnotesize
\centering
\begin{minipage}[b]{0.49\linewidth}
% \begin{minipage}[t][3cm][t]{5cm}
\input{img/swe_initial_values2.tikz}
\caption*{(a) $h$}
\end{minipage}
\begin{minipage}[b]{0.49\linewidth}
% \begin{minipage}[t][3cm][t]{5cm}
\documentclass{standalone}
\IfStandalone{
	\usepackage{pgfplots,pgfplotstable}
	\usetikzlibrary{external}
	\newcommand{\fromRoot}[1]{../#1}
}{%
}
\begin{document}
\tikzsetnextfilename{swe_initial_values_hu2}
\begin{tikzpicture}
\begin{axis}[
	width=\linewidth,
% 	grid=both,
	xlabel=$x$,
	ylabel=$hu$,
	legend entries ={
% 	$U_0^{(1)}$,
	$U_0^{(2)}$,
% 	$x_0^{(3)}$,
% 	$x_0^{(4)}$
	},
	legend style={at={(0.5,1.0)},anchor=south},
	legend columns=2
]
   
%       \addplot[mark=none,green,very thin] table[x index=0,y index=6] {\fromRoot img/data/swe_initial_values.dat};%expl midpoint
%       \addplot[mark=none,red,very thin] table[x index=0,y index=7] {\fromRoot img/data/swe_initial_values.dat};%expl midpoint
%       \addplot[mark=none,cyan,very thin] table[x index=0,y index=8] {\fromRoot img/data/swe_initial_values.dat};%expl midpoint
    \addplot[mark=none,blue,very thin] table[x index=0,y index=5] {\fromRoot img/data/swe_initial_values.dat};%expl midpoint
 \end{axis}
 	%tfewfe
\end{tikzpicture}
\end{document}
\caption*{(b) $hu$}
\end{minipage}
\caption{SWE Anfangswerte}
\label{fig:sweInitialValues2}
\end{figure}
Der benutzte Anfangswert $U_0 = (h_0,(hu)_0)$ für dieses Beispiel ist in Abbildung \ref{fig:sweInitialValues2} ersichtlich. Genauer ergibt er sich zu
\[
\begin{aligned}
%  h_0^{(1)} \sim n\cdot\mathcal N(0.5\cdot L,L) + 1 &&(hu)_0^{(1)}\sim n\cdot\mathcal N(0.5\cdot L,L) && \text{(Normalverteilt)}\\
 h_0^{(2)}(x)= \sin\left( \frac{2\pi k\cdot x}{L}\right) + 1.5 &&(hu)_0^{(2)}(x) = \cos\left( \frac{2\pi k\cdot x}{L}\right) + 1  && \text{(Schwingung)}
\end{aligned}
\]
Mit $k\in \R$ ist die Periode der Schwingung definiert. 
\subsection{Lösen der ODE}

\begin{figure}[H]
\centering
\input{img_appendix/swe_convergence1.tikz}
\caption{Konvergenz SWE im Intervall $I = [0,50], L=100, \Delta x=10$ mit $U_0^{(2)},k=2,\theta=1$}
\end{figure}

\begin{figure}[H]
\centering
\documentclass{standalone}
\IfStandalone{
	\usepackage{pgfplots,pgfplotstable}
	\usetikzlibrary{external}
	\newcommand{\fromRoot}[1]{../#1}
}{%
}
\begin{document}
\tikzsetnextfilename{swe_convergence1_romberg}
\begin{tikzpicture}
\begin{loglogaxis}[
	width=10cm,
% 	grid=both,
	xlabel=Anzahl der Freiheitsgrade $N$,
	ylabel=Fehler in zum Zeitpunkt $T$,
% 	ymin=1E-13,
%  	ymax=4E-7,
	legend entries ={
% 	Expl. MP,
	IMP,
	GIMP, 
	Romberg GIMP,
	Romberg IMP
	},
% 	legend style={at={(0.5,1.0)},anchor=south},
% 	legend columns=2
]
%   	\addplot[mark=none,red,very thin] table[x index=0,y index=2] {\fromRoot img_appendix/data/swe_convergence1.dat};%expl midpoint
 	\addplot[mark=none,green,very thin] table[x index=0,y index=3] {\fromRoot img_appendix/data/swe_convergence1.dat};%impl midpoint
	\addplot[mark=none,blue,very thin] table[x index=0,y index=1] {\fromRoot img_appendix/data/swe_convergence1.dat};%gen midpoint
	\addplot[mark=none,lime,very thin] table[x index=0,y index=5] {\fromRoot img_appendix/data/swe_convergence1.dat};%gen midpoint impl
	\addplot[mark=none,cyan,very thin] table[x index=0,y index=4] {\fromRoot img_appendix/data/swe_convergence1.dat};%gen midpoint rom
	\addplot[mark=none,very thin,gray, yshift=-20pt] 
		table[y={create col/linear regression={x=0,y=4}}] {\fromRoot img_appendix/data/swe_convergence1.dat}
		  coordinate [pos=0.5] (A)
		  coordinate [pos=0.6] (B)
		;
	% save the slope parameter:
	\pgfmathparse{-\pgfplotstableregressiona}	
	\pgfmathsetmacro{\slope}{\pgfmathresult}
	
	% draw the opposite and adjacent sides
	% of the triangle
	\draw[very thin,gray] (B) -| (A)
	node [pos=0.2,anchor=north]
	{\pgfmathprintnumber{\slope}};
	%deufe
\end{loglogaxis}
\end{tikzpicture}
\end{document}
\caption{Konvergenz SWE im Intervall $I = [0,50], L=100, \Delta x=10$ mit $U_0^{(2)},k=2,\theta=1$}
\end{figure}



\begin{figure}[H]
\footnotesize
\begin{minipage}[b]{0.49\linewidth}
% \begin{minipage}[t][3cm][t]{5cm}
\centering
% \input{img_appendix/swe_error_over_time1.tikz}
\input{img/swe_error_over_time.tikz}
\caption*{(a) Am Zeitpunk $t$}
\end{minipage}
%  \quad 
\begin{minipage}[b]{0.49\linewidth}
% \begin{minipage}[t][3cm][t]{5cm}
\centering
% \documentclass{standalone}
\IfStandalone{
	\usepackage{pgfplots,pgfplotstable}
	\usetikzlibrary{external}
	\newcommand{\fromRoot}[1]{../#1}
}{%
}
\begin{document}
\tikzsetnextfilename{swe_error_over_time1_all}
\begin{tikzpicture}
\begin{axis}[
	width=\linewidth,
	xlabel=Zeitpunkt $t$,
	ylabel=Fehler zum Zeitpunkt $t$,
	legend entries ={Expl. MP,
	IMP,
	GIMP 
	},
% 	legend style={at={(0,1)},anchor=north west}
	legend style={at={(1,1.0)},anchor=south east},
	legend columns=2
]
\addplot[mark=none,red,very thin] table[x index=0,y index=6] {\fromRoot img_appendix/data/swe_error_over_time_new.dat};
\addplot[mark=none,green,very thin] table[x index=0,y index=5] {\fromRoot img_appendix/data/swe_error_over_time_new.dat};
\addplot[mark=none,blue,very thin] table[x index=0,y index=4] {\fromRoot img_appendix/data/swe_error_over_time_new.dat};
\end{axis}
\end{tikzpicture}
\end{document}

\documentclass{standalone}
\IfStandalone{
	\usepackage{pgfplots,pgfplotstable}
	\usetikzlibrary{external}
	\newcommand{\fromRoot}[1]{../#1}
}{%
}
\begin{document}
\tikzsetnextfilename{swe_error_over_time_all}
\begin{tikzpicture}
\begin{axis}[
	width=\linewidth,
	xlabel=Zeitpunkt $t$,
	ylabel=Fehler zum Zeitpunkt $t$,
% 	axis y line=right,
% 	every axis y label/.style={at={(current axis.right )},rotate=270},
% 	axis y line*=right,
	ylabel near ticks,
	yticklabel pos=right,	
	legend entries ={Expl. MP,
	IMP,
	GIMP 
	},
% 	legend style={at={(0,1)},anchor=north west}
	legend style={at={(0.5,1.0)},anchor=south},
	legend columns=2
]
\addplot[mark=none,red,very thin] table[x index=0,y index=6] {\fromRoot img/data/swe_error_over_time.dat};
\addplot[mark=none,green,very thin] table[x index=0,y index=5] {\fromRoot img/data/swe_error_over_time.dat};
\addplot[mark=none,blue,very thin] table[x index=0,y index=4] {\fromRoot img/data/swe_error_over_time.dat};

\end{axis}
\end{tikzpicture}
\end{document}
\caption*{(b) Summiert}
\end{minipage}
% \caption{SWE Fehler über Zeit mit $I=[0,300], L=100, \Delta x=10, h = 1$ und $U_0^{(1)},n=10,\theta=1$}
\caption{SWE Fehler über Zeit mit $I=[0,300], L=100, \Delta x=10, h = 1$ und $U_0^{(2)},k=2,\theta=1$}
\label{fig:sweErrorOverTime2}
\end{figure}
\subsection{Gradient und Optimierung}
\begin{figure}[H]
\footnotesize
\begin{minipage}[b]{0.49\linewidth}
% \begin{minipage}[t][3cm][t]{5cm}
\centering
\input{img/swe_adjoint_eq.tikz}
\caption*{(a) $\dot{\overline{h}}$}
\end{minipage}
%  \quad 
\begin{minipage}[b]{0.49\linewidth}
% \begin{minipage}[t][3cm][t]{5cm}
\centering
\documentclass{standalone}
\IfStandalone{
	\usepackage{pgfplots,pgfplotstable}
	\usetikzlibrary{external}
	\newcommand{\fromRoot}[1]{../#1}
}{%
}

\begin{document}
\tikzsetnextfilename{swe_adjoint_eq1}
\begin{tikzpicture}
\begin{axis}[
	width=\linewidth,
	xlabel=Zeit $t$,
	ylabel=$\dot{\overline{hu}}$,
	legend entries ={
	$\dot{\overline{hu}}$
	},
% 	legend style={at={(0.5,1.0)},anchor=south},
% 	legend columns=2
]
\foreach \ind in {1,2,...,11}{
  	\addplot[mark=none,red,very thin] table[x index=0,y index=\ind] {\fromRoot img/data/swe_adjoint_eq1.dat};%expl midpoint
 }
\end{axis}
\end{tikzpicture}
\end{document}
\caption*{(b) $\dot{\overline{hu}}$}
\end{minipage}
\caption{SWE Adjungierte Gleichung mit $L=100,\Delta x=10,h = 0.1$ und $U_0^{(2)},k=1,\theta=2,\xobs(0)=U_0^{(2)},k_{\text{obs}}=2$}
\label{fig:sweAdjointEqRHS2}
\end{figure}

\begin{figure}
\footnotesize
\begin{minipage}[b]{0.49\linewidth}
% \begin{minipage}[t][3cm][t]{5cm}
\centering
% \input{img_appendix/swe_convergence_adjoint_discrete_theta1.tikz}
\documentclass{standalone}
\IfStandalone{
	\usepackage{pgfplots,pgfplotstable}
	\usetikzlibrary{external}
	\newcommand{\fromRoot}[1]{../#1}
}{%
}
\begin{document}
\tikzsetnextfilename{swe_convergence_adjoint_discrete}
\begin{tikzpicture}
\begin{loglogaxis}[
	width=\linewidth,
	xlabel=Anzahl der Freiheitsgrade $N$,
	ylabel=Fehler zu $\nabla J$,
	legend entries ={Expl. MP,
	IMP,
	GIMP 
	},
% 	ymax=3E-5,
	legend style={at={(0.5,1.0)},anchor=south},
	legend columns=2
]
	\addplot[mark=none,red,very thin] table[x index=0,y index=3] {\fromRoot img/data/swe_convergence_adjoint_discrete.dat};%expl midpoint
 	\addplot[mark=none,green,very thin] table[x index=0,y index=2] {\fromRoot img/data/swe_convergence_adjoint_discrete.dat};%impl midpoint
	\addplot[mark=none,blue,very thin] table[x index=0,y index=1] {\fromRoot img/data/swe_convergence_adjoint_discrete.dat};%gen midpoint

	\addplot[mark=none,very thin,gray, yshift=-20pt] 
		table[y={create col/linear regression={x=0,y=1}}] {\fromRoot img/data/swe_convergence_adjoint_discrete.dat}
		  coordinate [pos=0.2] (A)
		  coordinate [pos=0.3] (B)
		;
	% save the slope parameter:
	\pgfmathparse{-\pgfplotstableregressiona}	
	\pgfmathsetmacro{\slope}{\pgfmathresult}
	
	% draw the opposite and adjacent sides
	% of the triangle
	\draw[very thin,gray] (B) -| (A)
	node [pos=0.2,anchor=north]
	{\pgfmathprintnumber{\slope}};
\end{loglogaxis}
\end{tikzpicture}
\end{document}
\caption*{(a) diskrete Observierung}
\end{minipage}
%  \quad 
\begin{minipage}[b]{0.49\linewidth}
% \begin{minipage}[t][3cm][t]{5cm}
\centering
% \input{img_appendix/swe_convergence_adjoint_smooth_theta1.tikz}
\input{img/swe_convergence_adjoint_smooth.tikz}
\caption*{(b) Glatte Observierung}
\end{minipage}
% \caption{SWE Konvergenz von $\nabla J$ mit $I=[0,50],L=50,\Delta x=10$, $x_0=U_0^{(1)},n=10$, $x_{\text{obs}} = U_0^{(1)}, n=20$ und $\theta=1$}
\caption{SWE Konvergenz von $\nabla J$ mit $I=[0,50],L=50,\Delta x=10$, $x_0=U_0^{(2)},k=0.5$, $x_{\text{obs}} = U_0^{(2)}, k=2$ und $\theta=1$}
% \end{figure}
% 
% \begin{figure}
% \footnotesize
\quad\\[0.3cm]
\begin{minipage}[b]{0.49\linewidth}
% \begin{minipage}[t][3cm][t]{5cm}
\centering
% \input{img_appendix/swe_convergence_adjoint_discrete_theta2.tikz}
\input{img/swe_convergence_adjoint_discrete1.tikz}
\caption*{(a) Diskrete Observierung}
\end{minipage}
%  \quad 
\begin{minipage}[b]{0.49\linewidth}
% \begin{minipage}[t][3cm][t]{5cm}
\centering
% \input{img_appendix/swe_convergence_adjoint_smooth_theta2.tikz}
\input{img/swe_convergence_adjoint_smooth1.tikz}
\caption*{(b) Glatte Observierung}
\end{minipage}
% \caption{SWE Konvergenz von $\nabla J$ mit $I=[0,50],L=50,\Delta x=10$, $x_0=U_0^{(1)},n=10$, $x_{\text{obs}} = U_0^{(1)}, n=20$ und $\theta=2$}
\caption{SWE Konvergenz von $\nabla J$ mit $I=[0,50],L=50,\Delta x=10$, $x_0=U_0^{(2)},k=0.5$, $x_{\text{obs}} = U_0^{(2)}, k=2$ und $\theta=2$}
\end{figure}


