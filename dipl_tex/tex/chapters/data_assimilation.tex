\chapter{Datenassimilation}
\section{Problemstellung}
Die 4D variationelle Datenassimilation ist eine Methode, die im Zusammenhang mit hyperbolischen partiellen Differentialgleichungen steht, wie z.B. der Wellengleichung. Ziel ist es durch Steuerungsparameter, wie etwa Anfangswerte zum Zeitpunkt $t_0$, den Abstand eines Modells gegenüber der in der Zeit gemessenen Observierungsparameter minimal werden zu lassen. 

Um diese Methode herzuleiten folgen wir der Herangehensweise von Talagrand in \cite{talagrand1987variational} und nehmen zuerst die Evolutionsgleichung
\begin{equation}
\label{eq:odemodel}
 \frac{dx}{dt} = F(x)\quad \text{mit} \quad x(0) = x_0 \quad \text{und}\quad F:\R^n\to \R^n
\end{equation}
  
welche eine Gewöhnliche Differentialgleichung beschreibt. $x$ ist dabei die zeitabhängige Zustandsvariable aus dem Hilbertraum $\R^n$, welche die Entwicklung des Systems zu einem Zeitpunkt $t\in \R$ beschreibt. 
$x_0 \in \R^n$ sei der Anfangswert, welcher gleichzeitig der Steuerungsparameter ist.
Die Observierungsparameter $\xobs$ sind diskrete Werte, welche Orts - und Zeitabhängig und über ein Intervall $[0,T]$ verteilt sind. Da $\xobs$ diskret ist, wird eine Projektion $C:\Xstate\to \Xobs$ benötigt, welche $x$ aus dem Zustandsraum $\Xstate$ in den Observierungsraum $\Xobs$ abbildet. Üblicherweise ist diese Funktion eine Auswertungsfunktion von $x$ an den Stellen $\tobs$.

Als Kostenfunktional dient die $L^2$ - Norm der Differenz der Lösung $x$ von \eqref{eq:odemodel} zu $\xobs$ über ein gegebenes Zeitintervall $[0,T]$
\begin{equation}
\label{eq:costfunctional}
 J(x_0) = \frac{1}{2}\int_0^T H(x(t),t)dt= \frac{1}{2}\int_0^T \|C\cdot x(V)-\xobs\|^2dt
\end{equation}
Die skalarwertige Funktion $J$ ist damit jene Funktion, die das Maß zwischen der Lösung $x$ und den Observierungsparametern $\xobs$ darstellt.
Man beachte, dass das Zielfunktional nur abhängig vom Anfangswert $x_0$ ist. Das Ziel besteht darin, $x_0 = x_0^*$ zu finden, sodass
\[
 J(x_0^*) = \inf_{x_0} J(x_0) 
\]
minimiert wird. Die Grundlage zur Minimierung des Kostenfunktionals ist zunächst die Lösungstheorie zu gewöhnlichen Differentialgleichungen.

\section{Gewöhnliche Differentialgleichungen}
Eine Gleichung der Form \eqref{eq:odemodel} wird eine \textit{Gewöhnliche Differentialgleichung erster Ordnung} genannt. 
In der nachfolgend betrachteten Theorie sowie in der Implementation der Verfahren in dieser Arbeit werden ausschließlich autonome Gewöhnliche Differentialgleichungen erster Ordnung betrachtet, da sich sämtliche Gewöhnliche Differentialgleichungen in diese Form umschreiben lassen.
Falls nämlich eine Differentialgleichung n-ter Ordnung der Form 
\[
 \frac{d^n x}{dt^n}= F\left(t;x,\frac{dx}{dt},\ldots,\frac{d^{n-1} x}{dt^{n-1}}\right)
\]
mit $F$ differenzierbar, gegeben ist, ist wohlbekannt, dass sie sich durch Hinzufügen zusätzlicher Variablen in ein Differentialgleichungssystem erster Ordnung überführen lässt (\cite[S. 105]{arnold2001grundbegriffe}), mittels
\[
 \dot x_1 = x_2 \quad ,\ldots,\quad \dot x_{n-1} = x_n, \quad  \dot x_n = F(t;x_1,\ldots,x_{n-1})
\]
Durch die Einführung einer weiteren Variablen $x_0 = t$ und Gleichung $\dot x_0=1$ lässt sich die gegebene ODE autonom, also die rechte Seite $F$ unabhängig von der Zeit $t$, schreiben.
\subsection{Grundlagen}
Die Existenz und Eindeutigkeit von Lösungen wird durch den Satz von Picard-Lindelöf beschrieben, der die Grundlage zur Lösungstheorie Gewöhnlicher Differentialgleichungen bildet. Dieser wird beispielsweise in \cite{teschl2012ordinary} bewiesen und lautet wie folgt 
\begin{theorem}[Picard-Lindelöf]
\label{thm:picard-lindeloeff}
 Sei $F\in C(U,\R^n)$, wobei $U$ eine offene Teilmenge von $\R^{n+1}$ ist und $(t_0,x_0)\in U$. Falls $F$ im zweiten Argument lokal Lipschitzstetig und gleichmäßig stetig im ersten Argument ist, dann existiert eine eindeutige lokale Lösung $\bar x(t)\in C^1(I)$ des Anfangswertproblems \eqref{eq:odemodel}, wobei $I$ ein Intervall um $t_0$ ist.
\end{theorem}
% 
% Ein Spezialfall des Modells \eqref{eq:odemodel}, welche wir im Folgenden benötigen werden, sind die Matrix Differentialgleichungen, welche die Form
% \[
%  \frac{d}{dt} X = A(t)X,\quad \frac{d}{dt}X(t) =[\frac{d}{dt}x_{ij}(t)] 
% \]
% besitzen, mit $A\in R^{n\times n}$

TODO: REST IN PLDATAASSIMILATION UMLAGERN? ALSO: FILIPPOV THEOREM ETC??
TODO: WO NUMERIK VON ODES?
\[
 \dot x \in F(t,x)
\]
\cite{filippov1971existence}
\begin{theorem}[Filippov]
 Sei U wie in \ref{thm:picard-lindeloeff} gegeben und $D\subset U$. Sei $F:D\rightrightarrows \R^n$ eine mengenwertige Abbildung. Für alle Punkte der Region $D$ sei $F(t,x)$ eine nichtleere beschränkte abgeschlossene Menge. Desweiteren sei $F(t,x)$ oberhalbstetig, also für alle $(t,x)\in D$ und alle $\varepsilon>0$ existiert ein $\delta = \delta(\varepsilon,t,x)>0$, sodass für alle Punkte $(t',x')$ der $\delta$- Umgebung von $(t,x)$ die Menge $F(t',x')$ in der $\varepsilon$-Umgebung der menge $F(t,x)$ liegt.
 Wenn nun $F(t,x)$ überall konvex ist, dann besitzt  
 \[
  \dot x \in F(t,x)
 \]
 für alle Anfangswerte $x(t_0)=x_0$ mit $(t_0,x_0)\in D$ mindestens eine Lösung.
\end{theorem}


\subsection{Numerik gewöhnlicher Differentialgleichungen}
Bei den Optimierungsmethoden, welche wir betrachten, benötigt man daher den Gradienten der zu minimierenden Funktion $J$.

\section{Der Gradient des Zielfunktionals}
\subsection{Adjungierte Operatoren}
Für höhere Dimensionen ist die Bestimmung dieses Gradienten jedoch auf klassischem Weg numerisch sehr komplex. Dazu würde man jede Komponente unseres Systems \eqref{eq:odemodel} stören, das gestörte Modell in jeder Komponente integrieren und das Kostenfunktional dieser ausrechnen. Aus der Störung würde sich der Gradient $\partial J/\partial x_0$ ergeben. Dies ist sehr ineffizient. Als effiziente Alternative hat sich die Anwendung adjungierter Gleichungen erwiesen. Mit ihrer Hilfe kann die numerische Komplexität der Berechnung des Gradienten auf ein weniges Vielfaches des integrierens der Evolutionsgleichung \eqref{eq:odemodel} vermindert werden.

Eine allgemeine Einführung zur Theorie von adjungierten Gleichungen wurde beispielsweise von Cacuci \cite{cacuci1981sensitivity} gegeben, während Talagrand in \cite{talagrand1987variational} adjungierte Operatoren auf Hilberträume einschränkt und mit diesen die Datenassimilation einführt.
Für die folgenden Betrachtungen sind Hilberträume von besonderer Wichtigkeit. Sämtliche Lösungstheorie sowohl bei gewöhnlichen als auch bei partiellen Differentialgleichungen basieren auf den speziellen Eigenschaften dieser Räume.

Ein Hilbertraum ist ein reller oder komplexer Vektorraum mit einem Skalarprodukt, der bzgl. der vom Skalarprodukt induzierten Norm vollständig ist. Da $\R$ vollständig ist, ist jeder endlichdimensionale Vektorraum versehen mit einem Skalarprodukt vollständig. In der numerischen Praxis sind damit sämtliche folgenden Schritte valide. 
Ein Ableitungsbegriff auf Hilberträumen ist die sogenannte Gâteaux Ableitung.
\begin{definition}[Gâteaux Ableitung]
 Seien $X$ und $Y$ Banachräume, $U \subset X$ offen und $F:X\to Y$. Das Gâteaux Differential $\gatdiff F(u;\psi)$ von F an $u\in U$ in Richtung $\psi\in X$ ist definiert als
 \[
  \gatdiff F(u;\psi) = \lim_{\tau \to 0} \frac{F(x+\tau \psi) - F(x)}{\tau} = \frac{d}{d\tau} F(u+\tau\psi)\biggr\rvert_{\tau = 0}\quad \forall \psi \in X
 \]
 falls der Grenzwert existiert. Existiert der Grenzwert für alle $\psi\in X$, dann heißt $F$ Gâteaux-Differenzierbar in $u$.
 Falls $X$ und $Y$ endlichdimensional sind, entspricht das Konzept der Gâteaux Ableitung der der Richtungsableitung.
\end{definition}
Die folgende Definition der Adjungierten Operatoren und Proposition von Hilberträumen sind besonders nützlich für nachfolgende Betrachtungen.
\begin{definition}[Adjungierte Operatoren]
Sei $\Gil$ ein anderer Hilbertraum mit Innenprodukt $\langle \cdot,\cdot \rangle_\Gil$ und $L$ ein stetiger linearer Operator $L:\mathcal{G}\to \Hil$. Dann existiert ein eindeutiger stetiger linearer Operator $L^*:\Hil\to\Gil$, sodass für alle $u\in \Gil$ und $v\in \Hil$ gilt
\begin{equation}
\label{eq:adjointInnerProduct}
\langle v,Lu\rangle_\Hil =  \langle L^*v,u \rangle_\Gil
\end{equation}
$L^*$ heißt der \textit{adjungierte Operator} von $L$ (vgl. \cite[Definition V.5.1]{werner2007funktionalanalysis}). Im Falle das $\Hil$ und $\Gil$ endlichdimensional und durch orthonormale Koordinaten beschrieben werden kann, ist bekannt, dass  $L^*$ gerade die transponierte Matrix von $L$ ist.
% , da $\langle x,L^*y\rangle = \langle Lx,y\rangle =(Lx)^\tr y  = x^\tr L^\tr y= \langle x,L^\tr y\rangle  $
\end{definition}
Die Gâteaux Ableitung einer differenzierbaren Funktion ergibt sich als Richtungsableitung (vgl. \cite[Example 9.2.4]{debnath2005hilbert})
\begin{proposition}[Gâteaux Ableitung differenzierbarer Funktionen]
\label{prop:adjoints}
Sei $\Hil$ ein Hilbertraum mit Skalarprodukt $\langle \cdot,\cdot \rangle_\Hil$ und $v:\Hil \to \R,~ v\mapsto J(v)$ eine einmal stetig differenzierbare skalare Funktion definiert auf $\Hil$. Das Gâteaux Differential $\gatdiff J$ von $J$ kann an jedem Punkt von $\Hil$ über das Gâteaux Differential $\gatdiff v$ von $v$ ausgedrückt werden als
 \begin{equation}
 \label{eq:diffInnerProduct}
  \gatdiff J(v;\gatdiff v) = \langle \nabla_v J, \gatdiff v\rangle_\Hil
 \end{equation}
 $\nabla_v J$ ist hierbei der eindeutige Gradient von $J$ in Richtung $v$ an der Stelle $x$, welche zur Übersichtlichkeit weggelassen wurde. Wenn $\Hil$ endlichdimensional und durch orthonormale Koordinaten $v_i$ beschrieben ist, dann sind die Komponenten von $\nabla_v J$ gerade der Vektor der partiellen Ableitung von $J$ nach $v_i$, also $\partial J/\partial v_i$.
\end{proposition}


Seien $\Hil,~\Gil$ und $J$ wie in Proposition \ref{prop:adjoints} eingeführt. Betrachtet man jetzt eine Funktion $u:\Gil \to \Hil, ~u\mapsto v = G(u)$, G differenzierbar, so folgt, dass sich $J(v) = J(G(u))$ zu einer zusammengesetzte Funktion von $u$ ergibt. Nun ist 
\begin{equation}
\label{eq:diffCompoundFunction}
\gatdiff v = \gatdiff(G(u)) = G'(u)\gatdiff u
\end{equation}
wobei $G':\Gil\to\Hil$ die Ableitung von $G$ bzgl. $u$ und ein linearer Operator ist. Wenn man nun \eqref{eq:diffCompoundFunction} in \eqref{eq:diffInnerProduct} einsetzt erhält man 
\begin{equation}
 \gatdiff J = \langle \nabla_v J,G'\gatdiff u\rangle_\Hil \overset{\eqref{eq:adjointInnerProduct}}{=}  \langle G'^* \nabla_v J,\gatdiff u\rangle_\Gil \overset{n.V.}{=} \langle \nabla_u J,\gatdiff u\rangle_\Gil 
\end{equation}
wobei $G'^*$ der adjungierte Operator von $G'$ beschreibt. Damit ist 
\begin{equation}
\label{eq:adjointEssential}
\nabla_u J = G'^*\nabla_v J
\end{equation} 
Die Gleichung \eqref{eq:adjointEssential} stellt die Basis für die Nutzung der adjungierte Operatoren für die Datenassimilation dar. Sie bietet ein besonders effizienten Weg $\nabla_u J$ numerisch zu berechnen. Angewandt auf unser Problem wäre $u\mapsto v =G(u) $ eine Integration eines numerischen Modells. Aus einem Integrierer der $G'^*w$ für ein gegebenes $w$ berechnen kann, kann nun aus $\nabla_v J$ mittels \eqref{eq:adjointEssential} einfach $\nabla_u J$ errechnet werden. 
% Wenn $J$ einfach gewählt wurde, lässt sich daraus $\nabla_v J$ ebenfalls einfach berechnen. 
Um $\nabla_u J$ zu erhalten, muss zuerst $v = G(u)$ berechnet werden, danach $\nabla_v J$ und schlussendlich wieder $\nabla_u J$ durch \eqref{eq:adjointEssential}.

\subsection{Berechnung des Gradienten}
Um vorherige Betrachtungen auf die Ableitung des Zielfunktionals \eqref{eq:costfunctional} anwenden zu können wird die Gâteaux Ableitung des Kostenfunktionals \eqref{eq:costfunctional} zu einem Anfangswert $v$ gebildet
\begin{equation}
\label{eq:gatcost}
\begin{aligned}
 \gatdiff J(v)  &= \lim_{\tau\to 0} \frac{1}{\tau}\int_0^T H(x+\tau\gatdiff x)-H(x) dt\\
	    &= \lim_{\tau\to 0} \frac{1}{\tau}\int_0^T \int_0^1 \frac{d}{ds}H(x+s\tau\gatdiff x) dt\\
	    &= \lim_{\tau\to 0} \frac{1}{\tau}\int_0^T \int_0^1 \nabla_x H(x+s\tau\gatdiff x)\cdot \gatdiff x dt\\
	    &= \int_0^T \nabla_x H(x)\cdot \gatdiff x dt\\
	    &= \int_0^T \langle \nabla_x H(x), \gatdiff x \rangle dt\\  
\end{aligned}
\end{equation}
wobei $\nabla_x H(t)$ der Gradient von $H(x,t)$ ausgewertet an der Stelle $(x(t),t)$ und $\gatdiff x$ die Gâteaux-Ableitung von $x$, also eine Störung, beschreibt. Diese ist die Lösung von
\begin{equation}
\label{eq:tlm}
\begin{aligned}
  \frac{d \gatdiff x}{dt} &= \lim_{\tau\to 0}\frac{F(x+\tau \gatdiff x)-F(x)}{\tau}\\
			 &= \frac{d F(x+\tau\gatdiff x)}{d\tau} \Bigg\rvert_{\tau=0}\\
			 &= F'(t)\gatdiff x
\end{aligned} 
\end{equation}

und wird als \textit{Tangent Linear Model} bezeichnet. $F'(t)$ ist der Operator, der durch differenzieren von $F(x,t)$ nach $x$ ausgewertet an $x(t)$ entsteht. Da die gewöhnliche Differentialgleichung \eqref{eq:tlm} homogen und linear ist, hängt ihre Lösung linear vom Anfangswert $v$ ab und es gilt
\begin{equation}
\label{eq:resolvent}
 \gatdiff x(t) = R(t,0)v
\end{equation}

$R(t,0)$ ein wohldefinierter linearer Operator. Dieser wird als \textit{State Transition Matrix} oder \textit{Resolvent} bezeichnet und wird ausführlich von Zadeh und Desoer in \cite[S. 339 ff.]{zadeh1976linear} beschrieben. Er besitzt die folgenden Eigenschaften
\begin{align}
  R(t,t) &= I \label{eq:resolventPropertiesA}\\
  \frac{\partial}{\partial t}R(t,t') &= F'(t)R(t,t') \text{ für alle } t,t>0\label{eq:resolventPropertiesB}
\end{align}
welche ebenfalls in \cite[S. 339 Theorem 4]{zadeh1976linear} eingeführt und bewiesen werden.
Nun kann das Differential des Kostenfunktionals \eqref{eq:gatcost} mittels des Resolventen \eqref{eq:resolvent} und deren adjungiertem Operator $R^*(t,0)$ umgeschrieben werden zu
\begin{equation*}
\begin{aligned}
 \gatdiff J &= \int_0^T \langle \nabla_x H(t), R(t,0)v\rangle dt \\
	    &= \int_0^T \langle R^*(t,0) \nabla_x H(t), v\rangle dt \\
	    &= \left\langle \int_0^T R^*(t,0) \nabla_x H(t) dt, v\right\rangle \\
\end{aligned}
\end{equation*}
und es ergibt sich wie in den Betrachtungen zu Gleichung \eqref{eq:adjointEssential} %TODO die Anfangswertbedinungen einheitlich bennen v -> \gatdiff u oder so
\begin{equation}
\label{eq:gradCostFunctional}
 \nabla_v J = \int_0^T R^*(t,0) \nabla_x H(t) dt
\end{equation}
Der adjungierte Operator $R^*(t,0)$ des Resolventen $R(t,0)$ wird durch die Lösung des adjungierten Tangent Linear Models von \eqref{eq:tlm} 
\begin{equation}
\label{eq:adjointtlm}
 -\frac{d \gatdiff'x}{dt} = F'^*(t)\gatdiff'x
\end{equation}
beschrieben, wobei $\gatdiff'x \in\Hil$ und $F'^*(t)$ die Adjungierte von $F'(t)$ ist. 
Sei $S(t,t')$ der Resolvent von \eqref{eq:adjointtlm} zwischen den Zeitpunkten $t$ und $t'$. Für zwei Lösungen $\gatdiff x$ und $\gatdiff' x$ des Tangent Linear Models \eqref{eq:tlm} und deren Adjungierte \eqref{eq:adjointtlm} ist das innere Produkt $\langle \gatdiff x,\gatdiff' x\rangle$ konstant, da das innere Produkt der Ableitungen nach $t$ verschwindet:
\begin{equation}
\begin{aligned}
 \frac{d}{dt}\langle \gatdiff x(t),\gatdiff' x(t)\rangle  &= \left\langle \frac{d \gatdiff x}{dt}(t) ,\gatdiff' x\right\rangle +\left\langle \gatdiff x,\frac{d\gatdiff ' x}{dt}(t)\right\rangle \\
 &= \langle F'(t)\gatdiff x(t),\gatdiff 'x(t)\rangle - \langle \gatdiff x(t),F'^*(t)\gatdiff' x(t)\rangle\\
 &= 0
\end{aligned}
\end{equation}
Daraus folgt unmittelbar, dass die Lösungen zweier Anfangswerte $y,y' \in \Hil$ die Gleichung
\begin{equation*}
\langle R(t',t)y,y'\rangle = \langle y,S(t,t')y'\rangle 
\end{equation*}
erfüllen, da die Lösung des Tangent Linear Models \eqref{eq:tlm} für die Anfangsbedingung $y$ am Zeitpunkt $t$ gerade den Wert $R(t',t)y$ zum Zeitpunkt $t'$ und die Lösung des adjungierten Tangent Linear Models \eqref{eq:adjointtlm} für die Anfangsbedingung $y'$ am Zeitpunkt $t'$ den Wert $S(t,t')y'$ zum Zeitpunkt $t$ annimmt. Da dies für alle $y,y'$ gilt, ist $S(t,t')$ der adjungierte Operator von $R(t',t)$.
Für den Gradienten des Kostenfunktionals \eqref{eq:gradCostFunctional} entsteht somit
\begin{equation}
\label{eq:gradCostFunctionalAdjoint}
 \nabla_v J = \int_0^T S(0,t) \nabla_x H(t) dt
\end{equation}
Um den Gradienten des Kostenfunktionals nun darstellen zu können benötigen wir die \textit{Inhomogene Adjungierte Gleichung}
\begin{equation}
\label{eq:inhAdjEquation}
-d\gatdiff'x(t) = F'^*(t)\gatdiff'x - \nabla_x H(t)
\end{equation}
welche durch die Lösung 
\begin{equation}
\label{eq:solutionInhAdjEquation}
 \gatdiff'x(t) = \int_t^{t_1} S(t,\tau)\nabla_x H(\tau)d\tau
\end{equation}
mit $\gatdiff x(t_1)=0$ beschrieben ist. Dass \eqref{eq:solutionInhAdjEquation} die Inhomomgene Adjungierte Gleichung \eqref{eq:inhAdjEquation} löst lässt sich mittels den Eigenschaften des Resolventen \eqref{eq:resolventPropertiesA} und \eqref{eq:resolventPropertiesB} zeigen. Durch Substituieren erhält man
\begin{equation*}
 \begin{aligned}
  \frac{d\gatdiff'x}{dt} &= \frac{d}{dt} \int_t^{t_1} S(t,\tau)\nabla_x H(\tau)d\tau \\
			 &= \frac{d}{dt} \int_t^{t_1} G(t,\tau)d\tau 
\end{aligned}
\end{equation*}
Durch ableiten nach $t$ und der verallgemeinerten Leibniz Integral Formel (\ref{eq:genLeibnizIntRule}) folgt
\begin{equation*}
 \begin{aligned}
 \frac{d}{dt} \int_t^{t_1} G(t,\tau)d\tau  
			 &\overset{\eqref{eq:genLeibnizIntRule}}= G(t,t_1)\cdot 0 - G(t,t) + \int_t^{t_1} \frac{d}{dt}G(t,\tau) d\tau \\
			 &= -S(t,t)\nabla_x H(t) + \int_t^{t_1} \frac{d}{dt}S(t,\tau)\nabla_x H(\tau) d\tau \\
			 &\overset{\eqref{eq:resolventPropertiesA},\eqref{eq:resolventPropertiesB}}= -\nabla_x H(t) + \int_t^{t_1} F'^*(t)S(t,\tau)\nabla_x H(\tau) d\tau \\
			  &= -\nabla_x H(t) +F'^*(t) \int_t^{t_1} S(t,\tau)\nabla_x H(\tau) d\tau \\
			 &= -\nabla_x H(t) +F'^*(t) \gatdiff 'x(t) \\
 \end{aligned}
\end{equation*}
Die umgekehrte Richtung wird fast analog bewiesen; es gilt
\begin{equation*}
 -\frac{d\gatdiff 'x}{dt} = F'^*(t) \gatdiff'x - \nabla_x H(t)
			 = \frac{d}{dt} \int_t^{t_1} S(t,\tau) \nabla_x H(\tau) d\tau
\end{equation*}
Integration über $[t,t_1]$ liefert
\[
 \gatdiff 'x(t) -  \gatdiff 'x(t_1)  = \int_t^{t_1} S(t,\tau) \nabla_x H(\tau) d\tau
\]
Mit $ \gatdiff 'x(t_1) = 0$ und mit \eqref{eq:gradCostFunctionalAdjoint} folgt, dass $\nabla_v J = \gatdiff'x(0)$. Das bedeutet, dass sich der Gradient des Zielfunktionals mithilfe einer Rückwärtsintegration des Adjungierten Modells berechnen lässt.
Genauer werden folgende Schritte benötigt, um den Gradienten des Zielfunktionals \eqref{eq:costfunctional} zu berechnen
\begin{figure}
 \begin{enumerate}
 \item Setze $v$ als Anfangswert des Modells \eqref{eq:odemodel} und löse dieses über das Intervall $[0,T]$, speichere berechnete Werte $x(t_i)$ ab\\
 ($0= t_0<~ t_1<\ldots<t_l=T$, $l\in \N$)
 \item Setze $\gatdiff 'x(T) = 0$, integriere die Inhomogene Adjungierte Gleichung \eqref{eq:inhAdjEquation} rückwärts in der Zeit von $T$ bis $0$, berechne zu den Zeitpunkten $t_i$ den Gradienten $F'^*(t_i)$ und $\nabla_x H(t_i)$ aus den gespeicherten Werten $x(t)$ aus Schritt 1.
 \item $\nabla_v J = \gatdiff 'x(t_0)$
\end{enumerate}
 \caption{Berechnung der Jacobimatrix des Kostenfunktionals $J$}
\end{figure}

% Übersetzt in Pseudocode lautet der Algorithmus
%  \begin{algorithm}[H]
%  \algrenewcommand{\algorithmiccomment}[1]{\hfill{\scriptsize #1}}
%  \caption{\texttt{PlanC::calc\_kink\_partials}}
%  \label{alg:kinkPartials}
%  \begin{algorithmic}[1]
%  \Function{jac\_data\_assimilation}{$f,$}
%  	\State $xCalc \gets solve_ode()$
%  	\State $\bar{A} \gets 0 $, $\hat{A} \gets 0$
% 	\State $d\gets \frac{d}{\|d\|}\cdot \|\xhat - \xcheck\|$ \Comment{Normalize direction}
%  \Until{$\hat{\tau} \geq 1$	}
%  \State \Return $[\bar{A}, \hat{A}]$;
%  \EndFunction
%  \end{algorithmic}
%  \end{algorithm}
% 
% TODO: ALGORITHMUS HIER?

\section{Optimierungsmethoden}

Nun, da ein numerischer Weg gefunden wurde, $\nabla_v J$ effizient zu berechnen, stellt die Optimierung des Zielfunktional die letzte Hürde dar, der wir uns annehmen müssen. Das Ausgangsproblem besteht darin, eine reellwertigen, differenzierbaren Funktion J
\begin{equation}
\label{eq:minProblem}
 \min_{x_0} J(x_0) 
\end{equation}
zu minimieren. Dies ist ein unrestringiertes Optimierungsproblem. 

\subsection{Grundlagen}
Im folgenden sei vorausgesetzt, dass für ein $\tilde x\in D$ die \textit{Niveaumenge}
 \begin{equation}
 \label{eq:niveauset}
  N(J,J(\tilde x)) = \left\{ x\in D \vert J(x)\leq J(\tilde x)\right\}
 \end{equation}

  

 von $J$ in $x\in \R^n$ kompakt sei. Unter dieser Voraussetzung existiert nach dem Satz von Weierstraß ein $\bar x \in N(J,J(\tilde x))$ mit 
 \[
  J(\bar x) \leq J(x) \forall x\in N(J,J(\tilde x))
 \]
 und damit ist für $x\in D\backslash N(J,J(\tilde x))$: $J(x)>J(\tilde x)> J(\bar x)$. (vgl. \cite[Satz 1.2.2]{alt2002nichtlineare})
 

Zur Charakerisierung von Lösungen von \eqref{eq:minProblem} ist folgender Satz über notwendige Bedingungen (vgl. \cite[vgl. Satz 3.1.1 ff.]{alt2002nichtlineare}) hilfreich:
\begin{theorem}[Notwendige Bedingung erster Ordnung] 
\label{thm:optnotbed}
 Die Zielfunktion $J$ sei in $x\in D$, $D\subset \R^n$ differenzierbar. Ist $\bar x$ lokales Minimum von \eqref{eq:minProblem}, dann gilt
 \[
  \nabla J(\bar x)^\tr d \geq 0 \quad \forall d\in \R^n
 \]
 Wenn $d\in \R^n$ ist $(-d)\in \R^n$ und damit $\nabla J(\bar x)^\tr (-d) =  - \nabla J(\bar x)^\tr d  \geq 0$. Es folgt also
 \[
  \nabla J(\bar x)  =0 \in \R^n
 \]
\end{theorem}
 Den Beweis findet man ebenfalls in \cite[vgl. Satz 3.1.1 ff]{alt2002nichtlineare}.$\proofend$

Dieses Resultat wird zur Konstruktion von Optimierungsverfahren, insbesondere Abstiegsverfahren, verwendet. Das folgende Kriterium für Abstiegsrichtungen ist grundlegend für die Konstruktion von Abstiegsverfahren

%Bevor wir uns dieser annehmen, werden grundsätzliche Fragen zur Wahl der Suchrichtung und Schrittweite betrachtet. 

% Unser Ziel besteht darin, motiviert durch das Resultat aus Theorem \ref{thm:optnotbed}, für  
% (\cite[vgl. Satz 4.1.1]{alt2002nichtlineare}) 
\begin{lemma}
\label{lem:optstepsize}
 Die Funktion $J:\R^n\to \R^n$ sei differenzierbar in $x$. Weiter sei $d\in \R^n$ mit 
 \[
  \nabla J(x)^\tr d<0
 \]
 Dann gibt es ein $\bar \sigma>0$ mit $J(x+\bar \sigma d)< J(x)$ für alle $\sigma \in (0,\bar \sigma)$
\end{lemma}
Den Beweis findet man u.a. in \cite[vgl. Satz 4.1.1]{alt2002nichtlineare}. Ein Vektor $d$ heißt Abstiegsrichtung von $J$ im Punkt $x$, wenn $\nabla J(x)^\tr d<0$. Beispielsweise wird im Verfahren des steilsten Abstiegs gerade $d=- \nabla J(x)$ als Abstiegsrichtung verwendet, da 
\[
\nabla J(x)^\tr d = \nabla J(x)^\tr (-\nabla J(x)) = - \|\nabla J(x)\|^2 <0                                                                                                                                                                                                                                                                        \]
Ein allgemeines Abstiegsverfahren mit Schrittweitensteuerung wird ebenfalls in \cite[S. 69, Verfahren 4.1.4]{alt2002nichtlineare} gegeben
\begin{figure}[H]
 \begin{enumerate}
  \item Wähle einen Startpunkt $x^{(0)}\in \R^n$ und setze $k:=0$
  \item Ist $\nabla f(x^{(k)})=0_n$: Stopp
  \item Berechne eine Abstiegsrichtung $d^{(k)}$ und eine Schrittweite $\sigma_k>0$, so dass
  \[
   f(x^{(k)} + \sigma_k d^{(k)} < f(x^{(k)}))
  \]
  ist und setze $x^{(k+1)} = x^{(k)}+\sigma_k d^{(k)}$
  \item Setze $k:=k+1$ und gehe zu 2.
 \end{enumerate}
\caption{Allgemeines Abstiegsverfahren mit Schrittweitensteuerung}
\label{alg:genSteepestDescent}
\end{figure}
\subsection{Schrittweitensteuerung}
Um unser Ziel, effiziente Schrittweiten zu bestimmen, näher zu kommen, müssen wir fordern, dass für die Folge $x^{(k)}$ mit jedem weiteren Schritt gegen einen stationären Punkt konvergiert, also dass  
\begin{equation}
\label{eq:optstepsizegradtozero}
 \nabla f(x^{(k)})\to 0 \quad \text{für }k\to \infty
\end{equation}
gilt. Um dies zu erreichen wird das Prinzip des hinreichenden Abstiegs formuliert (\cite[Def. 4.4.2]{alt2002nichtlineare}):
\begin{definition}[Prinzip des hinreichenden Abstiegs]
\label{def:sufficientdescent}
 Seien $x\in N(J,J(x^{(0)}))$ und $d\in \R^n$ mit $\nabla_u J(x)^\tr d<0$ gegeben. Eine Schrittweite $\sigma$ erfüllt das \textit{Prinzip des hinreichenden Abstiegs}, falls
 \begin{equation}
 \label{eq:armijo1}
  f(x+\sigma d) \leq f(x) + c_1\sigma \nabla f(x)^\tr d
 \end{equation}
 und 
 \begin{equation}
\label{eq:armijo2}
 \sigma \geq -c_2 \frac{\nabla J(x)^\tr d}{\|d\|^2}
 \end{equation}


mit von $x$ und $d$ unabhängigen Konstanten $c_1>0$, $c_2>0$ gilt. Eine Schrittweite $\sigma$ heißt \textit{effizient}, falls
\[
 f(x+\sigma d) \leq f(x) - c\left( \frac{\nabla J(x)^\tr d}{\|d\|}\right)^2
\]
mit einer von $x$ und $d$ unabhängigen Konstanten $c>0$ gilt.
\end{definition}
Die erste Bedingung stellt sicher, dass der neue Funktionswert $f(x+\sigma d)$ unterhalb der von $x$ in Abstiegsrichtung $d$ aufgespannten Geraden befindet. Die zweite Bedingung sichert, dass die Schrittweite $\sigma$ in Bezug zu $\nabla J(x)^\tr d$ nicht zu schnell gegen $0$ geht.
TODO: BILD EINFÜGEN
Falls die Voraussetzung \eqref{eq:niveauset} erfüllt ist und wir effiziente Schrittweiten benutzen ist 
\[
 \frac{\nabla f(x^{(k)})^\tr d^{(k)}}{\|d^{(k)}\|} \to 0 \quad \text{für }k\to \infty
\]
erfüllt. Andersherum ist für $\nabla f(x^{(k)})^\tr d^{(k)}<0$
\[
 0 \leftarrow \frac{\nabla f(x)^{(k)} d^{(k)}}{\|d\|} = \frac{\nabla f(x)^{(k)} d^{(k)}}{\|\nabla f(x^{(k)})\|\|d\|}\cdot \|\nabla f(x^{(k)})\| = \beta_k \|\nabla f(x^{(k)})\|
\]
für $\beta_k<c<0$ für eine Konstante $c$ erfüllt; damit folgt \eqref{eq:optstepsizegradtozero}. Suchrichtungen, für die diese Bedingung gelten, heißen \textit{gradientenbezogen} (\cite[Def. 4.4.4]{alt2002nichtlineare}).
\begin{definition}[Gradientenbezogene Suchrichtungen]
 Es seien $x\in N(f,f(x^{(0)}))$ und $d\in \R^n$. Die Richtung $d$ heißt \textit{gradientenbezogen in x}, wenn
 \[
  -\nabla f(x)^\tr d\geq c_3\|\nabla f(x)\| \|d\|
 \]
mit einer von $x$ und $d$ unabhängigen Konstante $c_3>0$.

Die Richtung $d$ heißt \textit{streng gradientenbezogen in $x$}, wenn zusätzlicher 
\[
 c_4\|\nabla f(x)\| \geq \|d\| \geq \frac{1}{c_4}\|\nabla f(x)\| 
\]

\end{definition}

Daraus lassen sich Konvergenzsätze herleiten (\cite[Satz 4.4.9]{alt2002nichtlineare})
\begin{theorem}[Konvergenz des allgemeinen Abstiegsverfahrens]
 Sei \eqref{eq:niveauset} erfüllt, die Suchrichtungen aus dem allgemeinen Abstiegsverfahren \ref{alg:genSteepestDescent} seien gradientenbezogen in $x^{(k)}$ und die Schrittweiten $\sigma_k$ seien effizient. Stoppt das Verfahren nicht nach endlich vielen Schritten, dann gilt $\nabla f(x^{(k)})\to 0$ für $k\to \infty$. Weiter hat die Folge $\{x^{(k)}\}$ mindestens einen Häufungspunkt und für jeden solchen Punkt $\tilde x$ gilt $\nabla f(\tilde x) = 0$
% TODO eindige Nullstelle oder mehrere Nullstellen?

Falls zusätzlich $d$ eine streng gradientenbezogene Suchrichtung in $x^{(k)}$, die Schrittweitenfolge $\sigma_k$ beschränkt, die Menge 
$$M = \{\tilde x\in N(f,f(x^{(0)}))|\nabla f(\tilde x)=0\}$$ endlich ist und das Verfahren nicht nach endlich vielen Schritten stoppt, dann konvergiert die Folge $\{x^{(k)}\}$ gegen eine Nullstelle von $\nabla f$.

 Hat $f$ genau einen stationären Punkt $\tilde x \in N(f,f(x^{(0)}))$ und das allgemeine Abstiegsverfahren nicht nach endlich vielen Schritten stoppt, konvergiert die Folge $\{x^{(k)}\}$ gegen $\tilde x$.
 \end{theorem}
 
Motiviert aus dem Prinzip des hinreichenden Abstiegs \ref{def:sufficientdescent} ergibt sich das Armijo Verfahren (vgl. \cite[Verfahren 4.5.4]{alt2002nichtlineare}), um eine effiziente Schrittweite $\sigma_A$ zu erhalten:
\begin{figure}[H]
 Gegeben seien Konstante $0<c_1<1$, $0<c_2$ und $0<\beta_1\leq \beta_2 <1$ unabhängig von $x$ und $d$.
 \begin{enumerate}
  \item Wähle eine Start-Schrittweite 
  \[
   \sigma_0 \geq -c_2 \frac{\nabla f(x)^\tr d}{\|d\|^2}
  \]
  und setze $j:=0$
  \item Ist die Bedingung \eqref{eq:armijo1} erfüllt, d.h. gilt
  \[
   f(x+\sigma_j d)\leq f(x)+ c_1\sigma_j \nabla f(x)^\tr d
  \]
  dann setze $\sigma_A = \sigma_j$ und stoppe das Verfahren
  \item Wähle $\sigma_{j+1}\in [\beta_1\sigma_j,\beta_2\sigma_j]$
  \item Setze $j:=j+1$ und gehe zu 2.
 \end{enumerate}
\caption{Armijo Verfahren zur Schrittweitensteuerung}
\end{figure}



\subsection{Abstiegsverfahren}

Für dieses Verfahren werden folgende Schritte ausgeführt
\begin{figure}[H]
\begin{enumerate}
 \item Wähle Anfangspunkt $x^{(0)}$ und setze $k:=0$
 \item Überprüfe $\nabla J(x^{(k)}) == 0$?: Stopp
 \item Setze als Suchrichtung $d = - \nabla J(x^{(k)})$, berechne effiziente Schrittweite $\alpha_k$ und setze $x^{(k+1)} = x^{(k)}+ \alpha_k d^{(k)}$
 \item Setze $k:= k+1$ und gehe zu 2.
\end{enumerate} 
 \caption{Methode des steilsten Abstiegs}
\end{figure}



\section{Zusammenfassung}