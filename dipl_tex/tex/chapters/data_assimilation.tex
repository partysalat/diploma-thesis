\chapter{Datenassimilation}
\section{Problemstellung}
Die 4D Datenassimilation ist eine Methode, die im Zusammenhang mit hyperbolischen partiellen Differentialgleichungen steht, wie z.B. der Wellengleichung. Ziel ist es durch Steuerungsparameter, wie etwa Anfangswerte zum Zeitpunkt $t_0$, den Abstand eines Modells gegenüber der in der Zeit gemessenen Observierungsparameter minimal werden zu lassen. 

Um dieses Methode herzuleiten nehmen wir zuerst ein Modell der Form
\begin{equation}\label{eq:odemodel}
 \xdot = F(t,x)\quad \text{mit} \quad x(t_0) = V \quad \text{und}\quad F:\R^n\to \R^m
\end{equation}
  
welche eine Gewöhnliche Differentialgleichung beschreibt (vorheriges Kapitel). $x$ sei die zeitabhängige Zustandsvariable, welche die Entwicklung des Systems zu einem Zeitpunkt $t\in \R$ beschreibt. 
$V \in \R^m$ ist der Anfangswert, über den wir steuern wollen.
Die Observierungsparameter $\xobs$ sind diskrete Werte, welche Orts - und Zeitabhängig sind. Da $\xobs$ diskret ist wird eine Projektion $C:\Xstate\to \Xobs$ benötigt, welche $x$ aus dem Zustandsraum $\Xstate$ in den Observierungsraum $\Xobs$ abbildet. 
% Beispielsweise kann diese Funktion durch eine Interpolation der gemessenen Werte $\xobs$ dargestellt werden.
Als Kostenfunktional dient die $L^2$ - Norm der Differenz der Lösung $x$ von (\ref{eq:odemodel}) zu $\xobs$ über ein gegebenes Zeitinterval $[0,T]$
\begin{equation}
 J(x_0) = \frac{1}{2}\int_0^T \|C\cdot x(V)-\xobs\|^2dt
\end{equation}