% \documentclass[../../diplom_data_assimilation.tex]{subfiles}
%use the following makro for all include paths (graphics, data files, other tex files,...)
% \makeatletter\@ifundefined{fromRoot}{\newcommand{\fromRoot}[1]{../../#1}}{}\makeatother
% \begin{document}
\chapter{Datenassimilierung mittels stückweiser Linearisierung}
\section{Probleme durch Unglattheiten}
\subsection{Exclusion of Zenon}
\subsection{Adjoint Inclusion}

% \section{Verallgemeinerter Gradient}

% Da die Jacobimatrix an Kinks nicht eindeutig sein muss, wird ein Vektor $d$ mit angegeben, welcher uns die Richtung der Ableitung vorgibt. Diese ist dann wiederum eindeutig bestimmt.
% Um dies mit unseren eben geführten Betrachtungen zu vereinen berechnen wir also den Gradienten $J_\sigma$ nicht an der gegebenen Stelle $x_0$, sondern gehen ein Stück in das durch $d$ und $\Sigma$ vorgegebene Polyeder und berechnen dort die Ableitung an Stelle $\xo_0 = x_0+\frac{1}{2}\tau\cdot d$. Aufgrund der Linearität der Polyeder stimmen die Ableitung an der Stelle $x_0$ und jene an Stelle $\xo_0$ überein. Der Algorithmus ergibt sich sofort zu
% [TODOBild auf Linker Seite]


%% GENERALIZED ADJOINT MIDPOINT RULE
\section{Adjungierte verallgemeinerte Mittelpunktsregel}

Wie wir in (??) beobachtet haben, müssen wir das Adjungierte Modell rückwärts integrieren. Die Grundidee besteht darin, dass wir auf das Adjungierte Modell (??) der Datenassimilierung ebenfalls wieder die verallgemeinerte Mittelpunktsregel (??) anwenden. Daher, dass wir das Integral in Teilintervalle zerlegen können, berechnen wir die kritischen Multiplikatoren, springen von Kink zu Kink und berechnen auf diesen Teilintervalen den Gradienten unserer rechten Seite $F$.
Sei $\tau$ der kritische Multiplikator zum nächsten Kink. $x_n$ und $x_a$ sind die jeweiligen iterierten Werte mit $\xo$ als deren Mittelpunkt und $\diff \xo$ als Differenz. Dann ergibt sich sofort 
\begin{align*}
\xo &= \frac{\xhat+\xcheck}{2}\eq \xcheck = 2\xo - \xhat\\
\diff \xo &= \xo_n - \xo_a \eq \xo_n = \diff \xo + \xo_a\\
\xhat - \xcheck &= 2\xhat - 2\xo\\
\diff \tau_j&=\tau_j-\tau_{j-1}
\end{align*}
Unsere adjungierte Differentialgleichung mit adjungierter Variable $\xadj$ ergibt sich zu
\begin{equation}
\dot{\xadj} = x - x_{obs} - \frac{\partial F(x,d)}{\partial x}^\tr \cdot \xadj
\label{eq:adjModel}
\end{equation}

Wenn wir nun die verallgemeinerte Mittelpunktsregel (??) auf \ref{eq:adjModel} anwenden, folgt
\begin{align*}
\xadj_n - \xadj_a &= h\cdot \int_{-0.5}^{0.5}\xo-\rxobs - \frac{\partial F(\xo,d)}{\partial x}^\tr \cdot \xadj\\
									&= h\cdot \left[\xo-\rxobs - \int_{-0.5}^{0.5} \frac{\partial F(\xo ,d)}{\partial x}^\tr \cdot \xadj dt\right]\\
\end{align*}
Angenommen wir haben $l \in \mathbb{N}$ Kinks zwischen $x_n$ und $x_{a}$, wobei $-0.5 = \tau_0 <\tau_1 <\ldots < \tau_l=0.5$, dann können wir unser Integral aufteilen in
\begin{align*}
\xadj_n - \xadj_a &= h\cdot \left[ \xo-\rxobs - \sum_{i=1}^l \int_{\tau_{i-1}}^{\tau_{i}}\frac{\partial F(\xo+\tau_{i-1}d,d)}{\partial x}^\tr \cdot \xadj dt\right]\\
									&= h\cdot \left[\xo -\rxobs - \sum_{i=1}^l \underbrace{\frac{\partial F(\xo+\tau_{i-1}d,d)}{\partial x}^\tr }_{A_i^\tr} \cdot \int_{\tau_{i-1}}^{\tau_{i}} \xadj dt\right]\\
\end{align*}

Nun wenden wir die stückweise Linearisierung auf $\xadj$ an. Es entsteht
\begin{align*}
\xadj_n - \xadj_a &= h\cdot (\xo -\rxobs - \sum_{i=1}^l A_i^\tr \cdot \int_{\tau_{i-1}}^{\tau_{i}} \rxadj + t\diff \xadj dt)\\
									&= h\cdot \left[\xo -\rxobs - \sum_{i=1}^l A_i^\tr \cdot \left[t\rxadj + \frac{t^2}{2}\diff \xadj \right]_{\tau_{i-1}}^{\tau_i} \right]\\
									&= h\cdot \left[\xo -\rxobs - \sum_{i=1}^l A_i^\tr \cdot \left( (\tau_i - \tau_{i-1})\cdot \rxadj + \frac{1}{2}\diff \xadj \cdot(\tau_i^2-\tau_{i-1}^2) \right)\right]\\
									&= h\cdot \left[\xo -\rxobs - \sum_{i=1}^l A_i^\tr \cdot \left(\diff \tau_i\cdot \rxadj +  \diff \tau_i \frac{1}{2}\cdot(\tau_i+\tau_{i-1})\cdot \diff \xadj \right)\right]\\
									&= h\cdot \left[\xo -\rxobs - \sum_{i=1}^l A_i^\tr \cdot \left( \diff \tau_i\cdot \rxadj +  \diff \tau_i \rtau_i \diff \xadj \right)\right]\\
\end{align*}
Mit $\rxadj =\frac{1}{2}(\xadj_n + \xadj_a) $ und $\diff \xadj =\xadj_n - \xadj_a $ ergibt sich
\begin{align*}
\xadj_n - \xadj_a &= h\cdot \left[\xo -\rxobs - \sum_{i=1}^l A_i^\tr \cdot \left( \diff \tau_i\cdot \frac{\xadj_n + \xadj_a}{2} +  \diff \tau_i \rtau_i (\xadj_n - \xadj_a) \right)\right]\\
 &= h\cdot \left[\xo -\rxobs - \sum_{i=1}^l A_i^\tr \cdot \left( \left(\frac{1}{2} \diff \tau_i +\diff \tau_i \rtau_i\right) \cdot \xadj_n  +  \left(\frac{1}{2}\diff \tau_i-\diff \tau_i \rtau_i\right) \cdot \xadj_a \right)\right]\\
 &= h\cdot \left[\xo -\rxobs -  \left( \sum_{i=1}^l A_i^\tr \left(\frac{1}{2} \diff \tau_i +\diff \tau_i \rtau_i\right) \cdot \xadj_n  + \sum_{i=1}^l A_i^\tr  \left(\frac{1}{2}\diff \tau_i-\diff \tau_i \rtau_i\right) \cdot \xadj_a \right)\right])\\
\end{align*}
Durch Umsortierung erhalten wir
\begin{align*}
&& \left[I +h\sum_{i=1}^l A_i^\tr \left(\frac{1}{2} \diff \tau_i +\diff \tau_i \rtau_i\right) \right]\xadj_n &= 
\begin{aligned}[t]
&\left[I - h\sum_{i=1}^l A_i^\tr  \left(\frac{1}{2}\diff \tau_i-\diff \tau_i \rtau_i\right)\right]\xadj_a \\
& +h\cdot (\xo -\rxobs)
\end{aligned}\\
\iff&& \left[I +\frac{h}{2} \bar{A}^\tr +h\hat{A}^\tr\right]\xadj_n &= \left[I - \frac{h}{2}\bar{A}^\tr + h\hat{A}^\tr\right]\xadj_a  +h\cdot (\xo -\rxobs)\\
\end{align*}
mit $\bar{A}^\tr = \sum_{i=1}^l A_i^\tr \diff \tau_i $ und $\hat{A}^\tr = \sum_{i=1}^l A_i^\tr \diff \tau_i \rtau_i$.
Als Algorithmus ergibt sich folglich
 \begin{algorithm}[H]
 \algrenewcommand{\algorithmiccomment}[1]{\hfill{\scriptsize #1}}
 \caption{\texttt{PlanC::calc\_kink\_partials}}
 \label{alg:kinkPartials}
 \begin{algorithmic}[1]
 \Function{calc\_kink\_partials}{$\cx,\hx,d$}
 	\State $\hat{\tau} \gets 0$, $x_{kink} \gets \cx$
 	\State $\bar{A} \gets 0 $, $\hat{A} \gets 0$
	\State $d\gets \frac{d}{\|d\|}\cdot \|\xhat - \xcheck\|$ \Comment{Normalize direction}
 	\Repeat
 	  \State $\check{\tau} \gets \hat{\tau}, ~ x_{kink} \gets x_{kink} +\check{\tau}d$
 	  \State $\hat{\tau} \gets \Call{critMult}{ x_{kink},d}$ 		 \Comment{Berechne kritischen Multiplikator bis zum nächsten Kink}
		\If{$\hat{\tau}>1$} $\hat{\tau}\gets 1$ \EndIf 
 		\State $\rx \gets x_{kink}+0.5\cdot \hat{\tau} d$	\Comment{Berechne Mittelpunkt zwischen den Kinks}
 	  \State $\frac{\partial F(\rx)}{\partial x} \gets $ gen\_jac($\rx,d$) \Comment{Berechne $\partial F$ aus der Abs-Normalform}
 
 % 		\State $\rx \gets \Call{Solve}{\rx_\mathrm{new}}$
 	  \State $\bar{A} \gets \bar{A} +  \frac{\partial F(\rx)}{\partial x} \cdot (\hat{\tau} - \check{\tau})$ 
 		\State $\hat{A} \gets \hat{A} +  \frac{\partial F(\rx)}{\partial x} \cdot  \frac{1}{2}(\check{\tau} + \hat{\tau})-0.5$ \Comment{Verschiebe $\tau$ um $-0.5$}
 		
 \Until{$\hat{\tau} \geq 1$	}
 \State \Return $[\bar{A}, \hat{A}]$;
 \EndFunction
 \end{algorithmic}
 \end{algorithm}
\begin{algorithm}[H]
 \algrenewcommand{\algorithmiccomment}[1]{\hfill{\scriptsize #1}}
 \caption{\texttt{PlanC::jac\_data\_assimilation}}
 \label{alg:jacDataAssimilation}
 \begin{algorithmic}[1]
 \Require $x_{0},t_0,T, h,x_{Obs}, TOL$
 \State $N = \ceilS{\frac{t_0 - T}{h}}$
 \State $\hat{\bar{x}} \gets 0$ \Comment{Setze Anfangswert}
 \State $x \gets  \Call{solveODE}{x_0,t_0, T,h, TOL};$\Comment{Löse ODE in Vorwärtsrichtung}
 \For{$k\gets$ N-1 to $1$} \Comment{Zeitschritt rückwärts}
 	%\State $\rx \gets \cx - \frac h2 F(\cx)$ \Comment{initialization by half Euler}
 	\State $\rx \gets 0.5(x_k + x_{k-1})$ \Comment{Berechne Mittelpunkt}
 	\State $\pl_{\rx} F \gets \Call{Update}{}$ \Comment{Berechne neue Linearisierung am Mittelpunkt $\rx$}
 	\State $d \gets x_{k-1}-x_k$\Comment{Berechne neue Richtung}
 	\State $[\bar{A},\hat{A}] \gets \Call{calc\_kink\_partials}{x_k,x_{k-1},d}$  \Comment{Berechne $\partial F$ zwischen jedem Kink}
 	%\Until{$\|\hx - \cx - r - h F(\rx)\|$} < TOL
 	\State $\check{\bar{x}} \gets \Call{Solve}{( I-\frac{h}{2}\bar{A}^\tr + h \hat{A}^\tr)\check{\bar{x}}  = (I+\frac{h}{2}\bar{A}^\tr + h\hat{A}^\tr)\hat{\bar{x}}- h(\rx-\rx_{Obs})}$
 \EndFor
 \State \Return $-\check{\bar{x}}$
 \end{algorithmic}
 \end{algorithm}
Bemerkt sei, dass sich die Formel für $l=1$, wenn sich also kein Kink zwischen $x_n$ und $x_a$ befindet, zur bekannten impliziten Mittelpunktsregel vereinfacht. Für diesen Fall gilt $\diff \tau =1,\rtau = 0$ und damit
\begin{align*}
&& \left[I +h\sum_{i=1}^1 A_i^\tr \left(\frac{1}{2} \diff \tau_i +\diff \tau_i \rtau_i\right) \right]\xadj_n &=\begin{aligned}[t]
   & \left[I - h\sum_{i=1}^1 A_i^\tr  \left(\frac{1}{2}\diff \tau_i-\diff \tau_i \rtau_i\right)\right]\xadj_a  \\
	&+	h\cdot (\xo -\rxobs) 
       \end{aligned} \\
\iff &&  \left[I +h A_1^\tr \left(\frac{1}{2} \diff \tau_1 +\diff \tau_1 \rtau_1\right) \right]\xadj_n &= 
  \begin{aligned}[t]	
&\left[I - h A_1^\tr  \left(\frac{1}{2}\diff \tau_1-\diff \tau_1 \rtau_1\right)\right]\xadj_a  \\
&+h\cdot (\xo -\rxobs)
  \end{aligned} \\
\iff &&  \left[I +h A_1^\tr \left(\frac{1}{2} \cdot 1+0\right) \right]\xadj_n &= \begin{aligned}[t]
&\left[I - h A_1^\tr  \left(\frac{1}{2} \cdot 1-0\right)\right]\xadj_a \\
&+h\cdot (\xo -\rxobs)                                                                                  
\end{aligned}\\
\iff &&  \left[ I +\frac{h}{2} \frac{\partial F(\xo ,d)}{\partial x}^\tr \right]\xadj_n &= \left[ I - \frac{h}{2}  \frac{\partial F(\xo ,d)}{\partial x}^\tr  \right]\xadj_a  +h\cdot (\xo -\rxobs)\\
\iff &&  \xadj_n - \xadj_a &=h\cdot \left[\xo -\rxobs - \frac{\partial F(\xo ,d)}{\partial x}^\tr  \frac{\xadj_n + \xadj_a}{2}\right]\\
\end{align*}
vgl. [Satz wo normale implizite midpointrule aufgeführt wird]
 

 
\section{Optimierung}
*  Stepsize 
 
% \end{document}