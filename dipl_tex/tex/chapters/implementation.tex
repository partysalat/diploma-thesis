\chapter{Implementierung}
\section{Armadillo und OPENBlas}
Die Implementierung der theoretischen Ergebnisse der letzten Kapitel wurde in C++ programmiert mittels \textit{Armadillo}. 
Armadillo ist eine Template basierte C++ -Matrix-Vektor Bibliothek, welche ähnliche Syntax wie Matlab zulässt. Neben Matrix-Matrix oder Matrix-Vektor Multiplikationen liefert es eigene LGS-Löser mit, bietet einfache Möglichkeiten Teilmatrizen zu extrahieren und unterstützt teilweise Sparse Matrizen. Armadillo ist unter der Webseite \url{http://arma.sourceforge.net} erreichbar und steht unter der \textit{Mozilla Public License 2.0} (\url{https://www.mozilla.org/MPL/2.0/}) als Open Source Projekt zur Verfügung.

Desweiteren unterstützt Armadillo Multi Threaded Operationen durch die Abstrahierung des OpenBlas (\url{http://www.openblas.net/}) Interfaces, wodurch nahtlos Armadillos eigene Routinen durch die von OpenBlas ersetzt werden und trotzdem der Vorteil der einfachen Nutzbarkeit beibehalten bleibt. OpenBlas ist eine auf Parallelität optimierte Version der bekannten Linearen Algebra Bibliothek BLAS (Basic Linear Algebra Subprograms, \url{http://www.netlib.org/blas}). Es steht unter der BSD Lizenz (\url{http://www.linfo.org/bsdlicense.html}) zur Nutzung bereit.

\section{ADOL-C}
ADOL-C ist ebenfalls ein Open Source Paket zur exakten automatischen Differentiation von C/C++ Programmen im Rahmen der Rechengenauigkeit. Im Mittelpunkt stehen dabei die \textit{aktiven Variablen}, welche die Initialisierungsvariablen darstellen. Diese werden mit einem eigenen Typ \texttt{adouble} initialisiert, welche eine Erweiterung der \texttt{double} Klasse darstellt. Mit diesen Werten wird die gewünschte Funktion, wie im Falle von \texttt{double}, zwischen den Aufrufen \texttt{trace\_on(tag)} und \texttt{trace\_off(tag)} ausgeführt und durch \textit{operator overloading} im gleichen Zuge die Ableitung bestimmt. Intern erzeugt ADOL-C ein Tape, welches den computational graph der Funktionsauswertung erstellt und auf das durch \texttt{tag} referenziert wird. Nach dieser Auswertung bietet ADOL-C verschiedene Methoden dieses Tape auszuwerten.

ADOL-C wurde von Andreas Griewank erdacht und wird zurzeit in der Forschungsgruppe um Andrea Walther an der Universität Paderborn weiterentwickelt. Weitere Informationen zur Verwendung lassen sich in der Dokumentation \cite{walther2012getting} nachvollziehen. ADOL-C ist unter der \textit{Eclipse Public License 1.0} oder der \textit{GNU General Public License 2.0} in Projekten einsetzbar.

\section{Plan-C}
Sämtliche Versuche wurden mit Plan-C erstellt, einer C++ Bibliothek, welche im Rahmen der Diplomarbeit von Paul Boeck in \cite{boeck14} erarbeitet und dokumentiert wurde. Da im Rahmen der vorliegenden Arbeit rechenintensivere Beispiele betrachtet werden, wurde diese Bibliothek geforkt und Armadillo als Matrix - Vektor Bibliothek eingesetzt, um die allgemeine Performance zu steigern; dies bedeutet, die in Boecks Arbeit als Verbesserungsvorschlag gegebene Benutzung von Sparse Matrizen einzubauen und mehrerer Prozessorkerne durch Parallelisierung zu unterstützen.  Desweiteren wurde sie durch diverse Funktionen erweitert, damit sie die Datenassimilierung mittels stückweiser Linearisierung unterstützt. Eine kurze Dokumentation darüber soll im Folgenden dargelegt werden.

\subsection{Abs-Normal Form}
Plan-C ist eine C++ Klasse, die ein Objekt zur Verfügung stellt welches eine Abstraktion einer Funktion darstellt. Intern wird der Reverse Mode von \texttt{ADOL-C} benutzt, um zu einer gegebene Funktion seine Abs-Normal Form an einem gegebenen Entwicklungspunkt aufzustellen und mit ihr zu rechnen.

Um ein Plan-C Objekt zu erstellen kann man entweder direkt die Abs-Normal Form mit den Matrizen aus Gleichung  \eqref{eq:absNormalForm} 
\begin{lstlisting}
 PlanC(vec c,vec b,mat Z, mat L, mat J, mat Y, vec xo);
\end{lstlisting}
oder eine Funktion \texttt{func} übergeben
\begin{lstlisting}
 PlanC(size_t n,size_t m,vec xo, simple_function func, short tag);
\end{lstlisting}
\texttt{vec} und \texttt{mat} sind die Armadillo internen Klassen für Vektoren bzw. Matrizen, \texttt{n} und \texttt{m} sind die Dimensionen des Definitions-/Werteberech von \texttt{func} und \texttt{xo} der Entwicklungspunkt. \texttt{tag} gibt an, welches Tape von \texttt{ADOL-C} benutzt werden soll. 
Die Funktion \texttt{func} muss in der Form 
\begin{lstlisting}
 void (simple_function*) (short tag, double* px, double* py);
\end{lstlisting}
initialisiert werden, damit \texttt{Plan-C} diese verarbeiten kann. \texttt{tag} ist wieder die Nummer des benutzten Tapes,  \texttt{px} ist ein Pointer auf die Eingabeparameter und \texttt{py} beschreibt die Rückgabewerte. 
Beispielsweise ist eine Funktion gegeben als
\begin{lstlisting}[caption=Beispiel einer simple\_function,label=lst:simpleFunc]
 void func(short tag, double* px, double* py){
 //aktiviere Abs Normal Form Berechnung
  enableMinMaxUsingAbs();
  //beginne Taping auf tag
  trace_on(tag);
  size_t n = 2;
  size_t m = 2;
  x = new adouble[n];
  y = new adouble[m];
  for(size_t i=0;i<n;i++){
    //markiere x[i] als aktive Variable
    x[i] = <<= px[i]		
  }
  
  //Fuehre Berechnungen durch
  y[0] = fabs(x[0]+x[1]);
  y[1] = x[0] + fabs(x[0]-x[1]);
  
  for(size_t i=0;i<m;i++){
    //markiere y[i] als Resultat
    y[i] = <<= px[i]		
  }
  delete[] x, delete[] y;
  // stoppe taping
  trace_off(tag); 
 }
\end{lstlisting}

Die Methoden \texttt{enableMinMaxUsingAbs}, \texttt{trace\_on} und \texttt{trace\_off} sind \texttt{ADOL-C} interne Funktionen, wobei \texttt{enableMinMaxUsingAbs} erst im Rahmen der Entwicklung von \texttt{Plan-C} zu \texttt{ADOL-C} hinzugefügt wurde. Alle arithmetischen Operatoren der \texttt{std::cmath} Klasse wurden von \texttt{ADOL-C} überladen.

In \texttt{Plan-C} werden im Konstruktor folgende Methoden benutzt
\begin{description}
 \item[\texttt{get\_num\_switches}] gibt die Menge der switching Variablen zurück
 \item[\texttt{zos\_an\_forward}] Analysiert das Tape und behält Ableitungen erster Ordnung, berechnet \texttt{z}
 \item[\texttt{fos\_an\_reverse}] Die Matrizen \texttt{Z,L,Y} und \texttt{J} werden aus dem Tape errechnet und zeilenweise zurückgegeben
\end{description}
Mithilfe dieser drei Funktionen wird die Abs-Normal Form als Matrix-Repräsentation aufgestellt.
Desweiteren stellt das \texttt{Plan-C} Objekt weitere Funktionen zur Verfügung:
\begin{description}
 \item[\texttt{update\_linearization(vec xo)}] Berechnet die Abs-Normal Form an der Stelle \texttt{xo}
 \item[\texttt{eval(vec x)}] Wertet die Abs-Normal Form an Stelle \texttt{x} aus
 \item[\texttt{eval\_F(vec x)}] Wertet die Ausgangsfunktion an Stelle \texttt{x} aus 
\end{description}

\subsection{Verallgemeinerte Mittelpunktsregel}
Die Implementierung der verallgemeinerten Mittelpunktregel in C++ wird analog dem Pseudocode aus \ref{alg:genMidpointRule} programmiert. 
\begin{lstlisting}[caption=Verallgemeinerte Mittelpunktsregel,label=lst:genMidpointRule]
 mat PlanC::solve_ode(const vec &t, const vec &x0, bool verbose, double TOL) {
  size_t N = t.n_elem;
  double h = fabs(t(1)-t(0));
  //setze Anfangswert
  x.row(idx) = x0.t();
  ...
  //Initialisierung der Vektoren
  ...	
  while (idx < N - 1) {
    ++idx;
    //Initialschritt mit halben Euler
    xo = xc + h * 0.5 * eval_F(xc);		
    do {
      ... 
      ++cnt_inner_it;
      //erstellt ein Update der Abs-Normal Form an Punkt xo
      updateLinearization(xo);
      v = 2.0 * (xo - xc);
      r = h * (integrate(v) + integrate(-v) - fxo);
      PlanC tmp = scale_shift(2.0, -h);
      xo = tmp.solve(r + 2.0 * xc, false, 1E-14);
      res = 2.0 * xo - 2.0 * xc - r - h * eval_F(xo);
      resError = arma::norm(res);
      ...
    } while (resError > TOL && cnt_inner_it < MAX_IT);
    if (cnt_inner_it == MAX_IT) {...}//error
    xc = 2 * xo - xc;    
    x.row(idx) = xc.t();
  }
  return x;
}

\end{lstlisting}
Die Methode \texttt{integrate} ist die Implementierung der Prozedur \ref{alg:quad} und \texttt{scale\_shift} erstellt eine verschobenes \texttt{Plan-C} Objekt, welches für die stückweise lineare Newton Prozedur \ref{eq:unfoldedNewton} verwendet wird.
\subsection{Berechnung des Kostenfunktionals}
\subsection{Projektionsoperator}

\subsection{Adjungierte Mittelpunktsregel}


\subsection{Optimierung}