\chapter{Einleitung}
Viele in der Natur vorkommende Probleme in den Ingenieursdisziplinen sowie in der angewandten Mathematik, meist modelliert durch gewöhnliche oder partielle Differentialgleichungen, besitzen eine unglatte Natur. Die klassische Lösungstheorie setzt jedoch meist hinreichend glatte Eigenschaften der zu integrierenden Funktion zur Existenz der Lösung oder vollen Konvergenzordnung voraus. Zur Berechnung in der numerischen Praxis werden die Probleme oftmals glatt approximiert oder deren Unglattheiten gar nicht behandelt. Dadurch können deren numerische Lösungen in der Theorie auf lineare Konvergenzordnung zurückfallen. Dies ist auch der Fall, falls sich diese Modelle nur aus unglatten Konkatenationen, wie \textit{max}, \textit{min} und \textit{abs}, von glatten Funktion darstellen. Auch prinzipiell glatte Modelle erhalten beispielsweise durch Diskretisierungen unglatte Komponenten. 

Deswegen gab es in den letzten Jahren Bestrebungen, Unglattheiten numerisch besser behandeln zu können. In den Artikeln \cite{monster} und \cite{plan} von Andreas Griewank et. al. wurde daher ein neuer Ansatz für lipschitzstetige rechte Seiten, den der \textit{stückweisen Linearisierung} gewählt. Diese auf die Integration einiger ausgewählter Beispiele anzuwenden und mit ihr Datenassimilation zu betreiben und auszuwerten soll Ziel dieser Arbeit sein.

Dazu wird in Kapitel 2 eine Einführung in die Datenassimilation gegeben, welche eine Erläuterung zum Lösen von gewöhnlichen Differentialgleichungen, mathematische Grundlagen zur Berechnung des Gradienten des Kostenfunktionals und Optimierung des Letzteren beinhaltet.

Kapitel 3 widmet sich anschließend der Beschreibung des \textit{Automatischen Differenzierens}, der Darstellung der zu behandelnden Funktionsklasse und eine Beschreibung der Grundlagen der stückweisen Linearisierung.

In Kapitel 4 wird zuerst die von Griewank eingeführte verallgemeinerte implizite Mittelpunktsregel hergeleitet. Um diese Methode zu implementieren werden stückweise lineare Quadraturverfahren und linearer Gleichungssystemlöser eingeführt und mit deren Hilfe ein implementierbarer Algorithmus erstellt.

Die Berechnung des Gradienten des Kostenfunktionals zu verbessern ist Gegenstand von Kapitel 5, wo ebenfalls auf Schwierigkeiten bei der Berechnung des Gradienten der rechten Seite eingegangen wird. 

Während Kapitel 6 die Implementierung der Methoden in der \texttt{C++} Klasse \src{Plan-C} beschreibt sind in Kapitel 7 numerische Experimente dargelegt, mit denen die Implementierung und die numerischen Eigenschaften getestet und ausgewertet werden.

Schließlich ist in Kapitel 8 eine Zusammenfassung und ein Ausblick der Weiterentwicklung und Möglichkeiten der stückweisen Linearisierung auf das Lösen von gewöhnlichen Differentialgleichungen, der Datenassimilation und Optimierung dargestellt.