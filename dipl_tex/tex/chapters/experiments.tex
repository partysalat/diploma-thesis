
\chapter{Experimente}
\section{Rolling Stone}

\section{Shallow Water Equation}
\subsection{Problemstellung}
\subsection{Unglattheiten}
Fluxlimiter, Eigenwerte, Plot der Rechten Seite
\subsection{Ergebnisse}


Ein oft genutztes Beispiel, um Datenassimilierungsmethoden zu testen ist die sogenannte Shallow Water Equation (vgl. \cite{zou,navon}). Diese beschreibt die Bewegung einer Flüssigkeit in einem quadratischen Gebiet unter Berücksichtigung von Gravitationswellen. In kartesischen Koordinaten erhält man folgende Gleichungen
\begin{equation}
\begin{aligned}
 \frac{\partial u }{\partial t} = -u\frac{\partial u}{\partial x} -v\frac{\partial u}{\partial y} + fv - \frac{\partial \phi}{\partial x}\\
 \frac{\partial v }{\partial t} = -u\frac{\partial v}{\partial x} -v\frac{\partial v}{\partial y} - fu - \frac{\partial \phi}{\partial y}\\
 \frac{\partial \phi }{\partial t} = -\frac{\partial u\phi}{\partial x} -\frac{\partial v\phi}{\partial y} 
\end{aligned}
\label{eq:swe}
 \end{equation}
Hierbei sind $u$ und $v$ die zwei Komponenten des Fließgeschwindigkeitsfeldes in $s^{-1}$ und $\phi$ das Geopotential in $m^2 s^{-1}$; 
$f$ bezeichnet den Coriolisfaktor.
Die Anfangsbedingungen für die folgenden Experimente wurden aus \cite{grammeltvedt}(6.I) übernommen. Es handelt sich hierbei um eine westwärtgsgerichtete Strömung mit Nord - Süd Störungen verschiedener Wellenlängen und Amplituden entlang der Zonal - Achse der Strömung. Das anfängliche Höhenfeld wurde gewählt mittels Höhenfunktion
\begin{align*}
 h(x,y) = H_0 + H_1 \tanh \left( \frac{9(y-y_0)}{2D}\right) + H_2 \sech^2\left(\frac{9(y-y_0)}{D}\right) \sin\left(\frac{2\pi x}{L}\right)
\end{align*}
und Ableitungen
\begin{align*}
 \frac{\partial h}{\partial x}(x,y) &= h_2\sech^2\left( \frac{9(y-y_0)}{D}\right) \cos\left( \frac{2\pi x}{L} \right)\frac{2\pi}{L} \\
 \frac{\partial h}{\partial y}(x,y) &= h_1 \sech^2\left( \frac{9(y-y_0)}{2D}\right)\frac{9}{2D} -  h_2\frac{18}{D} \sin\left( \frac{2\pi x}{L}\right) \frac{\sinh(9(y-y_0)/D)}{\cosh^3(9(y-y_0)/D)}\\
\end{align*}
wobei $D$ die Breite, $L$ die Länge der betrachteten Fläche ist,  $h_0 := 2000m$, $h_1 := -220m$, $h_2 := 133m$ und $y_0 = D/2$. Als schlussendliche Anfangsbedingungen ergeben sich
\begin{align*}
 \phi_0(x,y) &= gh(x,y)\\
 u_0(x,y) &= -\frac{f}{g} \frac{\partial h}{\partial y}(x,y)\\
 v_0(x,y) &= \frac{f}{g} \frac{\partial h}{\partial x}(x,y)
\end{align*}
mit $f := 10^{-4} s^{-1}$ und $g:= 10 ms^{-1}$.
Damit wir die Datenassimilierungsmethode auf (\ref{eq:swe}) anwenden können, müssen wir die Gleichung in eine Form überführen, welche nur noch von der Zeit $t$ abhängt. Grammeltvedt (in \cite{grammeltvedt}) bietet dazu diverse Finite Differenzen Schematas für die Shallow Water Equation an, mit denen sich \ref{eq:swe} in die gewünschte Form überführen lässt. In unserem Fall nutzen wir Schema F. (\ref{eq:swe}) erhält somit die Form:
\begin{equation}
 \begin{aligned}
  \frac{\partial u}{\partial t} &= -(\bar{u}^x\bar{u}^x)_x - (\bar{u}^y\bar{v}^y)_y + u(\bar{u}^x_x+ \bar{v}^y_y) +fv - \bar{\phi}^x_x\\
  \frac{\partial v}{\partial t} &= -(\bar{v}^x\bar{u}^x)_x - (\bar{v}^y\bar{v}^y)_y + v(\bar{u}^x_x+ \bar{v}^y_y) -fv - \bar{\phi}^y_y\\
  \frac{\partial \phi}{\partial t} &= -(\bar{\phi}^x\bar{u}^x)_x - (\bar{\phi}^y\bar{v}^y)_y   
 \end{aligned}
 \label{eq:schemef}
\end{equation}
mit \[
\begin{aligned}
\overline{(uv)}^x_x & = \bar{u}^{2x} \bar{v}^x_x +\bar{v}^{2x} \bar{u}^x_x \\
(\bar{u}^z\bar{v}^z)_z &= \frac{1}{2} \left[ \overline{(uv)}^x_x + u \bar{v}^x_x + v \bar{u}^x_x \right] \\
 &= \frac{1}{2} \left[ \bar{u}^{2x}\bar{v}^x_x +  \bar{v}^{2x}\bar{u}^x_x  + u \bar{v}^x_x + v \bar{u}^x_x \right]\\
 \bar{u}^x_x &= \frac{1}{\Delta} \left[ \bar{u}^x(x_i+\frac{\Delta}{2}) -\bar{u}^x(x_i-\frac{\Delta}{2})  ) \right]\\
	    &=  \frac{1}{2\Delta} \left[ u(x_i+\Delta)+u(x_i)-u(x_i)-u(x_i-\Delta) \right]\\
	    &=  \frac{1}{2\Delta} \left[ u(x_i+\Delta)-u(x_i-\Delta) \right]\\
 \bar{u}^{2x}&= \frac{1}{2} \left[ u(x_i+\Delta) + u(x_i - \Delta)\right]
\end{aligned}
\]
Mit diesen Gleichungen können wir das Schema programmieren. Die Randbedingungen sind durch starre homogene Neumannbedingungen entlang der Nord- und Süd Grenzen und periodischen Randbedingungen bezüglich der Ost/West Grenzen gegeben. Seien $x_l,x_r,y_t,y_b$ die Koordinaten bzgl. der linken (l), rechten (r), oberen (t) und unteren (b) Ränder. Dann sind die Randbedingungen folgendermaßen definiert:
\begin{equation}
 \begin{aligned}
  u(x_l,y,t) = u(x_r,y,t)\\
  v(x_l,y,t) = v(x_r,y,t)\\
  \phi(x_l,y,t) = \phi(x_r,y,t)\\
  \frac{\partial u}{\partial y}(x,y_t,t) = 0 = \frac{\partial u}{\partial y}(x,y_b,t)\\ 
  v(x,y_t,t) = 0 = v(x,y_b,t)\\
  \frac{\partial \phi}{\partial y}(x,y_t,t) = 0 = \frac{\partial \phi}{\partial y}(x,y_b,t)\\ 
 \end{aligned}
\end{equation}
