\chapter{AD und Behandlung von Unglattheiten}
Sämtliche Betrachtungen, welche wir bis jetzt geführt haben, setzen voraus, dass die rechte Seite $F$ unseres Modells \eqref{eq:odemodel} hinreichend glatt definiert ist, damit wir sie anwenden können. In diesem Kapitel werden Lösungen geführt, welche es ermöglichen, vorherige Betrachtungen auf stückweise lineare
\footnote{In dieser Arbeit nutzen wir "linear" synonym zu "affin" oder "polyhedral", obwohl letztere eventuell präziser wären. Sie entsprechen den Begriffen bei Griewank \cite{monster} und Scholtes \cite{scholtes2012introduction}}
Funktionen
% , bzw. lokal Lipschitzstetige Funktionen $F$ 
anzuwenden. Desweiteren wird eine kurze Einführung in Automatischem Differenzieren gegeben, welches so erweitert wird, dass es unserem Modell entspricht.


% Eine Funktion $F:U\subseteq \R^n\to \R^m$ heißt Lipschitzstetig, wenn eine Konstante $L$ existiert, sodass für alle $x,y\in U$ gilt
% \[
%  \|F(y) - F(x)\| \leq \|y-x\|
% \]
% , wobei $L$ als \textit{Lipschitzkonstante} bezeichnet wird. 
% Eine Funktion $F$ wird \textit{lokal Lipschitzstetig} genannt, falls eine Umgebung $V\subseteq U$ zu einem $x\in U$ existiert, sodass $F$ auf eingeschränkt auf diese Umgebung $V$ Lipschitzstetig ist.

\section{Stückweise lineare Funktionen}
\subsection{Max Min Repräsentation}
Scholtes definiert in \cite[S.19]{scholtes2012introduction} eine stückweise affine Funktion $f:\R^n\to \R^m$ als eine stetige Funktion, zu der eine endliche Menge affiner Funktionen $f_i(x)=A^ix+b$, $i=1,\ldots,k$ existiert, sodass für jedes $x\in \R^n$ die Inklusion $f\in\lbrace f_1(x),\ldots, f_k(x)\rbrace $ gilt. Die affinen Funktionen $f_i$ werden \textit{Auswahlfunktionen} genannt. Scholtes bewies in \cite[Prop.2.2.2.]{scholtes2012introduction}, dass sich jede dieser stückweise affinen Funktionen als Verkettung der Funktionen max und min darstellen lässt
\begin{theorem}[Max-Min Repräsentation]
 Falls $f:\R^n\to \R^m$ eine stückweise affine Funktion mit affinen Auswahlfunktionen $f_1=a_1^\tr x+ b_1,\ldots,f_k=a_k^\tr x+ b_k$ ist, dann existiert eine endliche Anzahl von Indexmengen $M_1,\ldots,M_k\subseteq \lbrace 1,\ldots,k\rbrace$ sodass
 \begin{equation}
 \label{eq:maxMinRepresentiation}
  f(x) = \max_{1\leq i\leq l} \min_{j\in M_i} a_i^\tr x + b_i
 \end{equation}
 
\end{theorem}
In der Theorie hat diese Repräsentation viele Vorteile, in der Praxis stellt es sich jedoch als aufwändig heraus, eine Max-Min Repräsentation zu einer gegebenen stückweise affinen Funktion zu finden. Desweiteren ist sie offensichtlich nicht eindeutig. Falls eine Repräsentation gefunden wurde ist es numerisch sinnvoll, eine minimale Verschachtelungstiefe zu erreichen, d.h. möglichst wenig min/max Aufrufe miteinander zu verschachteln. Diese zu reduzieren gestaltet sich im Allgemeinen ebenfalls schwierig.

Eine andere Art der Darstellung, welche in der polyhedralen Natur der stückweise lineare Funktionen liegt, wird durch eine endliche Menge von Teilmengen $\Sigma\subset \R^n$ definiert, auf denen die stückweise affine Funktion mit ihren Auswahlfunktionen übereinstimmt. Es lässt sich zeigen, dass sich diese Menge in konvexe Polyeder zerlegen lässt, so dass ihre Auswahlfunktionen auf einem oder mehreren Polyeder oder Schnittmengen mehrerer Polyeder aktiv sind \cite[S.23 ff.]{scholtes2012introduction}.
Ein Polyeder ist eine Menge $P\subseteq \R^n$, falls eine $m\times n$ - Matrix $A$ und ein $m$-dimensionaler Vektor $b$ existiert, sodass $P = \lbrace x\in \R^n ~|~ Ax\leq b \rbrace$. 
Die Menge dieser konvexen Polyeder werden \textit{Polyhedrale Subdivision} von $\R^n$ zur Funktion $f$ genannt. Genauer ist $\Sigma$ eine Polyhedrale Subdivision von $P \subseteq \R^n$, falls
\begin{enumerate}
 \item Jedes Polyeder $\sigma \in \Sigma$ eine Teilmenge von $\R^n$ ist
 \item Die Dimension der Polyeder aus $\Sigma$ mit der Dimension von $\R^n$ übereinstimmt
 \item Die Vereinigung aller Polyeder von $\Sigma$ den Raum $\R^n$ überdeckt
 \item Jeweils zwei Polyeder von $\Sigma$ sind entweder disjunkt oder berühren sich nur an ihren Kanten.
\end{enumerate}
\begin{figure}[H]
\centering
\documentclass{standalone}
\usepackage{pgfplots,pgfplotstable}

\usetikzlibrary{external}

\begin{document}

\tikzsetnextfilename{polyhedral_subdevision}
\begin{tikzpicture}
\pgfplotstableread{
0.1	0.2	0.5
1.2	0.1	0.3
2	0	0.4
3	0	0.5
0	1	0.7
1.1	0.9	0.6
2.3	0.7	0.7
3	1	0.7
0.3	1.8	1.0
1.2	2.2	0.7
1.8	2.3	0.6
2.7	1.7	0.9
}\datatable
\begin{axis}[
  height=0.4\textheight,
  view={15}{15},	%view={azimuth}{elevation}
%   xlabel=x,
%   ylabel=y,
  xtick={0,0.5,1,1.5,2,2.5,3},
  ytick={0,0.5,1,1.5,2},
  xticklabels=\empty,
  yticklabels=\empty,
  zticklabels=\empty
  3d box=none,
  grid=major,
  grid style={color=black!10},
  zmin = 0,
  xmin = -0.2, xmax = 3.2,
  ymin = -0.2, ymax = 2.4,
  colormap/jet,
]
%projection to xy plane
\addplot3[
    mesh,draw=gray,thin,dashed,mesh/rows=3,
    z filter/.code={\def\pgfmathresult{0}}
  ] 
  table {\datatable};

%specific polyhedron marks
\addplot3[black!70,thick,mark=*] coordinates{(3,1,0.7) (3,1,0.0)};
\addplot3[black!70,thick,mark=*] coordinates{(2.7,1.7,0.9) (2.7,1.7,0.0)};
\addplot3[black!70,thick,mark=*] coordinates{(2.3,0.7,0.7) (2.3,0.7,0.0)};
\addplot3[black!70,thick,mark=*] coordinates{(1.8,2.3,0.6) (1.8,2.3,0.0)};

\addplot3[draw=black!70,thin,mesh/rows=3] 
  coordinates {(3,1,0.0) (2.7,1.7,0.0) (1.8,2.3,0.0) (2.3,0.7,0.0) (3,1,0.0)};

%the real mesh and its outline
\addplot3[surf,opacity = 0.8,shader=interp,mesh/rows=3] table {\datatable};
\addplot3[mesh,draw=black,thick,mesh/rows=3] table {\datatable};

% \node at (axis cs:2.3,1.8,.6) [above] {$P_\sigma$};
\node at (axis cs:2.5,1,0) [above] {$\sigma$};
\end{axis}
\end{tikzpicture}

 
\end{document}

\caption{Polyhedrale Subdivision}
\label{fig:polyhedralSubdivision}
\end{figure}
Die Berührungskanten aus Punkt 4, bzw. ihrere höherdimensionalen Äquivalente werden als \textit{Kinks} bezeichnet. 


\subsection{Abs-Normal-Form}
Eine weitere Art der Darstellung ergibt sich aus der Max-Min Repräsentation \eqref{eq:maxMinRepresentiation} einer stückweisen linearen Funktion.
Da sich $max$ und $min$ als
\[
\max(a,b) = \frac{1}{2}(a+b + |a-b|)\quad \text{und} \quad \min(a,b) = \frac{1}{2}(a+b - |a-b|)
\]
darstellen lassen, können alle $max$ und $min$ - Funktionsaufrufe als $s\geq 0$ Aufrufe der $|z_i|$ Funktion behandelt werden. Die Argumente $z_i$ heißen hierbei \textit{switching-Variablen}. Jede switching Variable $z_i$ beschreibt für sich genommen bereits eine affine Funktion, welche wiederrum die Funktionen $|z_j|, ~j<i$ aufrufen. Desweiteren existieren Variablen $x_k, ~k\leq n$, welche unabhängig von $abs$ sind. Diese Abhängigkeiten lassen sich in der \textit{Abs Normal Form} beschreiben, welche von Griewank et. al. in \cite{plan} eingeführt wurde:
\begin{definition}[Abs Normal Form]
 Eine stückweise lineare Funktion $F:\R^n\to \R^m$ mit $F(x) = y$ ist in \textit{Abs Normal Form}, falls sie die folgende Form besitzt
 \[
  \begin{bmatrix}
   z\\y
  \end{bmatrix}
  =
  \begin{bmatrix}
   c\\b
  \end{bmatrix}
  +
  \begin{bmatrix}
   Z & L\\
   J & Y
  \end{bmatrix}
  \begin{bmatrix}
   x\\|z|
  \end{bmatrix}
 \]
wobei 
\[
c\in \R^s, ~ Z\in \R^{s\times n},~ L\in\R^{s\times s}, ~ b\in\R^m, ~ J\in\R^{m\times n}, ~ Y\in \R^{m\times s} 
\]
$L$ muss offensichtlich die Form eine streng unteren Dreiecksmatrix besitzen, da sonst im Auswertungsgraph der Funktion $F$ Zyklen entstehen könnten.
\end{definition}
Die kleinste ganze Zahl $\nu \leq s$ für die 
\[
 L^\nu =0
\]
gilt, wird Verschachtelungstiefe oder \textit{switching depth} genannt. 
In der Numerischen Praxis kommen Beispiele vor, bei denen eine hohe \textit{switching depth} auftreten kann, beispielsweise bei Flux Limitern bei ortsdiskretisierten PDEs, welche wir im Beispiel (siehe TODO:REFERENZ ZU MINMOD) zur Shallow Water Equation benutzen. Griewank erläutert in \cite{monster}, dass eine große switching depth zu numerischen Instabilitäten führen kann und man daher versuchen soll, diese möglichst gering zu halten, eben um auch die kombinatorische Komplexität der Matrizen $Z$, $L$ und $Y$ zu minimieren. Tatsächlich lässt sich die switching depth $\nu$ der Abs-Normal Repräsentation für eine stückweise lineare Abbildung $F:\R^n\to\R^m$ durch 
\[
\nu \leq \bar \nu(n) = 2n-1                                                                                                                                                                                                                                                                                                                                                                                                                                                                                                                                                                                                                                                            \]
beschränken (\cite[S.3]{plan})


Unsere Funktion $F$ ist linear, falls $\nu=s = 0$, falls sie also keine $abs$ Terme besitzt. Für den Fall $\nu=1$ heißt $F$ \textit{simply switched}

Ziel dieser Arbeit ist es, die Eigenschaften der Matrizen der Abs-Normal Formulierung für unsere Berechnungen zu nutzen.
Dazu wollen wir später die Klasse der \textit{stückweise glatten Funktionen} durch stückweise lineare Funktionen approximieren.
Die Klasse der stückweisen glatten Funktionen kann als Erweiterung der stückweisen affinen Funktionen betrachtet werden. Sie bestehen aus $C^{1,1}$ Auswahlfunktionen, welche durch die unglatten Operationen $max$, $min$ und $abs$ verknüpft sind.
Die grundsätzliche Idee besteht darin, die Auswahlfunktionen durch Tangenten anzunähern und sie wieder durch Anwendung der $abs$ Funktion zu verbinden, sodass eine stückweise Linearisierung der Ursprungsfunktion entsteht. Auf dieser gelten besondere Eigenschaften, welche in den nachfolgenden Kapiteln sukzessive angewendet werden.

Um die stückweise Linearisierung einer stückweise glatten Funktion $F$ automatisiert zu berechnen, wollen wir die Möglichkeiten des Automatischen Differenzierens nutzen und erweitern.

\section{Automatisches Differenzieren}
Automatisches oder Algorithmisches Differenzieren ist ein Verfahren, um exakte, im Rahmen der Rechengenauigkeit \cite[S.51]{griewank2008evaluating}, Gradienten von Funktionen zu berechnen. Um dies zu erreichen wird weder numerisch noch algebraisch die Ableitung berechnet, sondern durch wiederholte Anwendung der Kettenregel. Dadurch wird eine verschachtelte Funktion auf einzelne Funktionen heruntergebrochen, von denen die Ableitung bekannt ist. Das bedeutet, dass wir durch einen Aufwand, der durch ein vielfaches einer Funktionsauswertung beschränkt ist, den Gradienten einer Funktion an einer Stelle berechnen können (\cite[S. 43, S.83]{griewank2008evaluating}). 

Um eine vektorwertige Funktion $F:\mathcal D \subset \R^n\to \R^m$ mittels Automatischem Differenzieren abzuleiten, muss sie bereits in einer algebraischen Form vorliegen, sie soll also aus Verknüpfung der Relationen $+, -, \cdot, /$ oder polynomiellen arithmetischen Operationen sin, cos, sqrt, exp, log usw. bestehen, also jenen, von denen eine Ableitung bekannt ist.  

Zum Auswerten einer programmierten Funktion führt der Compiler eine sogenannte \textit{Three-Part Evaluation Procedure} durch, d.h. er speichert die unabhängigen Eingangsvariablen $x\in \R^n$ in internen Variablen $(v_{1-n},\ldots,v_{-1})$ ab, führt $l-m$ Berechnungsschritte $(v_1,\ldots, v_{l-m})$ aus und gibt die abhängigen Variablen $(v_{l-m+1},\ldots,v_l)$ zurück. Diese Berechnungen lassen sich durch einen \textit{Computational Graph} visualisieren, wie dies in Tabelle \ref{fig:computationalGraphTable} für das Beispiel $F(x) = \sin(|x_1\cdot x_2|+x_1)$ durchgeführt wurde. 
Für das Ausführen werden diese \textit{Elementarfunktionen} $\lbrace \phi_i \rbrace_{i\in I}$ nacheinander in der richtigen Reihenfolge ausgeführt, welche durch die Halbordnung $\prec$ definiert ist, sodass wir für jedes $v_i$ mit $i>0$ und Argumenten $v_j$, $j<i$ schreiben können
\[
 v_i = \phi_i(v_j)_{j\prec i}
\]

Um jetzt die erste Ableitung einer Funktion zu bestimmen, müssen wir jede ihrer Elementarfunktion ableiten und sie mit der Kettenregel wieder zusammenfügen. Falls am Punkt $\xo$ die Ableitung ausgewertet werden soll in Richtung des Inkrements $\Delta x = x- \xo $, dann erhalten eine tangentiale Approximation der Ableitung unter Nutzung des Funktionswertes $\vo_i = v_i(\xo)$ mittels 
\[
 v_i(\xo - \Delta x) -\vo \approx \Delta v_i \equiv \Delta v_i(\Delta x) 
\]
Für alle $\phi_i$ wird angenommen, dass sie differenzierbar in einer hinreichenden Umgebung sind. Dann nutzen wir die Tangentiale Linearisierung
\begin{align}
\Delta v_i &= \Delta v_j \pm \Delta v_k & \text{für } v_i = v_j \pm v_k\\
\Delta v_i &= \vo_j* \Delta v_k +\Delta v_j*\vo_k& \text{für } v_i = v_j * v_k\\
\Delta v_i &= \mathring c_{ij} * \Delta v_j  & \text{für } v_i = \phi_i(v_j) \not \equiv \abs(v_j)
\end{align}
wobei $\mathring c_{ij} = \phi'_i(\vo_j)$
% und als \textit{tangent functions} bezeichnet werden 
und es ergibt sich durch die Kettenregel
\[
 \Delta y = \Delta F(\xo,\Delta x) \equiv \nabla F(\xo)\Delta x
\]
wobei $ \nabla F(\xo)\in \R^{m\times m}$ die Jacobimatrix bezeichnet.

Diese Methode wird "Vorwärtsmodus" genannt, bei welchem jedoch insbesondere für Ableitungen höherer Dimensionen $d$ der Aufwand linear mit $d$ wächst. AD-Programme wie ADOL-C nutzen den sogenannten "Rückwärtsmodus", welcher die Zielfunktion auswertet, Parameter abspeichert und die Ableitung mit weniger Aufwand berechnen kann. Diese Möglichkeit wird an dieser Stelle nur erwähnt, da ihre Erläuterung den Rahmen der Arbeit sprengen würde.



% \begin{figure}[H]
%  \centering
%  \documentclass{standalone}
\IfStandalone{
	\usepackage{pgfplots,pgfplotstable}
	\usetikzlibrary{external}
	\newcommand{\fromRoot}[1]{../#1}
	\usetikzlibrary{positioning,shapes}
	\newcommand{\abs}{abs}
}{%
}
\begin{document}

\tikzsetnextfilename{computational_graph}
\begin{tikzpicture}[
	state/.style={draw,ellipse,minimum width=6em},
% 	operation/.style={above,sloped,font={\tiny}},
	pl/.style={above,sloped},
% 	every node/.style={draw,ellipse,font={\tiny}},
	every edge/.style={thick,draw},
% 	>=triangle 90,
]
\node[state,double] (x1) {$v_{-2} = x_1$};
\node[state,double] (x2) [below= of x1]{$v_{-1} = x_2$};

\node (v0) [right= of x1] {$v_0$};
\node (v1) [right= of v0] {$v_1$};

\node (v2) [below right= of v1] {$v_2$};

\node [state,double] (y1) [below right= of v2] {$v_3$};

\path[->] 
		(x1) edge (v0)
		(x2) edge (v0) 
		(v0) edge (v1);
		
\path[->]
		(v1) edge (v2) 
		(x2) edge (v2) 
		(v2) edge (y1);
% 		(v1) edge (v3)
% 		(z1) edge node[pl] {$\abs(z_1)$} (v4)
% 		(z2) edge node[pl] {$\abs(z_2)$} (v5)
% 		(v4) edge (v6)
% 		(v5) edge (v6)
% 		(v6) edge (v7)
% 		(v7) edge (y1);
\end{tikzpicture} 
\end{document}

%  \caption{Computational Graph für $F(x) = \sin(|x_1\cdot x_2|+x_1)$}
%  
%  \begin{tabular}{|c|c|}
%  \hline
%  $v_{-2} =  x_1$ 	\\
%  $v_{-1} =  x_2$ 	\\
%  \hline
%  $v_{0} = v_{-2}\cdot v_{-1}$ 	\\
%  $v_{1} =  \abs(v_{0})$ 	\\
%  $v_{2} =  v_1+v_{-1}$ 	\\
%  $v_{3} =  \sin(v_{2})$ 	\\
%  \hline
%  $y =  v_{3}$ 	\\
%  \hline
% \end{tabular}
% \caption{Three Part Evaluation Procedure für $F(x) = \sin(|x_1\cdot x_2|+x_1)$}
% \label{fig:computationalGraphTable}
% \end{figure}


\section{Stückweise Linearisierung}

Ersetzt man nun diese stückweise glatten Funktionen durch stückweise lineare Funktionen, so erhält man eine Approximation der Funktion zweiter Ordnung im Abstand zum Ausgangspunkt (\cite[Prop.1]{monster}). Man definiert dafür für einen Punkt $\xo$ und eine Richtung $\Delta x$ die \textit{Inkrementfunktion} $\Delta F(\xo,\Delta x):\mathcal D \times  \R^n \to \R^m$ für welche gilt:
\[
 F(\xo + \Delta x) = F(\xo) + \Delta F(\xo,\Delta x) + \mathcal O (\|\Delta x\|^2)
\]
% \begin{figure}
%   \input{computational_graph _gen.tikz}
% \end{figure}
\section{Alternative Methoden?}
\subsection{Event Handling?}
\subsection{Schrittweitensteuerung?}