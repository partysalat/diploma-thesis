\chapter{Behandlung von Unglattheiten}
Sämtliche Betrachtungen, welche wir bis jetzt geführt haben, setzen voraus, dass die rechte Seite $F$ unseres Modells \ref{eq:odemodel} hinreichend glatt definiert ist, damit wir sie anwenden können. In diesem Kapitel werden Lösungen geführt, welche es ermöglichen, vorherige Betrachtungen auf stückweise lineare
\footnote{In dieser Arbeit nutzen wir "linear" synonym zu "affin" oder "polyhedral", obwohl letztere eventuell präziser wären. Sie entsprechen den Begriffen bei Griewank \cite{monster} und Scholtes \cite{scholtes2012introduction}}
Funktionen, bzw. lokal Lipschitzstetige Funktionen $F$ anzuwenden. 

\section{Stückweise Linearisierung}
Wir betrachten nun die Klasse der lokal Lipschitzstetigen Funktionen. Eine Funktion $F:U\subseteq \R^n\to \R^m$ heißt Lipschitzstetig, wenn eine Konstante $L$ existiert, sodass für alle $x,y\in U$ gilt
\[
 \|F(y) - F(x)\| \leq \|y-x\|
\]
, wobei $L$ als \textit{Lipschitzkonstante} bezeichnet wird. 



\subsection{Stückweise lineare Funktionen}

\subsection{Abs Normal Form}


\section{Alternative Methoden}
\subsection{Event Handling}
\subsection{Schrittweitensteuerung}