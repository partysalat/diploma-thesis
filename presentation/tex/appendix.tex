% All your regular slides
% After your last numbered slide
\appendix
\newcounter{finalframe}
\setcounter{finalframe}{\value{framenumber}}

\begin{frame}
\frametitle{3D und 4D Var. data assimilation}

\begin{block}{3D Var}
 \[J_{3DVar} = \frac{1}{2}(x-x_b)^\tr B^{-1} (x-x_b) +  \frac{1}{2} (C(x) -y)^\tr R^{-1}(C(x) -y)\]
\end{block}
\begin{block}{4D Var}
\[J_{4DVar} = \frac{1}{2}(x-x_b)^\tr B^{-1} (x-x_b) + \frac{1}{2} (C(M_i(x)) -y_i)^\tr R^{-1}(C(M_i(x)) -y_i)\]
\end{block}
\begin{itemize}
\item $x_b$ Hintergrund (vorherige Berechnung, Penalty Parameter)
 \item $B$ Fehler-Kovarianzmatrix des Hintergrundfeldes
 \item $R$ Fehler-Kovarianzmatrix der Observierungen
 \item $M_i$ Modelloperator (überführt $x_0$ nach $x_i$)
\end{itemize}
3D Var betrachtet nur einen Zeitpunkt, 4D eine Zeitspanne (rechenintensiver, \cite{huang2003introduction})
\end{frame}

\begin{frame}
\frametitle{Beispiel: Abs Normal Form Berechnung}
  \begin{block}{Beispiel}
\centering
  $
   \begin{aligned}
    F(x_1,x_2) = \begin{pmatrix}
                  x_1 + |z_1| + |z_3|\\
                  x_2
                 \end{pmatrix}
&& \text{mit} &&\begin{aligned}
        z_1 &= x_1-x_2 \\
        z_2 &= x_2 \\
        z_3 &= x_1-z_2        
       \end{aligned}
   \end{aligned}$
  \end{block}
\[
 \begin{lbmatrix}{1}
  z_1\\
  z_2\\
  z_3\\
  \hline
  y_1\\
  y_2
 \end{lbmatrix}
 =
 \begin{lbmatrix}{1}
  0\\
  0\\
  0\\
  \hline
  0\\
  0
 \end{lbmatrix}
 +
 \begin{lbmatrix}{6}
  1&-1 &\vline&0 &0 &0 \\
  0& 1 &\vline&0 &0 &0 \\
  1&0 &\vline&0 &-1 &0 \\
  \hline
  1&0 &\vline&1 &0 &1 \\
  0&1 &\vline&0 &0 &0 \\
 \end{lbmatrix}
 \begin{lbmatrix}{1}
  x_1\\
  x_2\\
  \hline
  |z_1|\\
  |z_2|\\
  |z_3|
 \end{lbmatrix}
\]
\end{frame}

% \begin{frame}[<+->]
% \frametitle{GIMP Algorithmus}
%  ODE SOLVER ALG
% \end{frame}
\begin{frame}[<+->]
\frametitle{Berechnung der GIMP}
\begin{align}
x_n - x_a &= h \int_{-\frac{1}{2}}^{\frac{1}{2}} F(\xo) + \Delta F(\xo,t(x_n-x_a))dt\\
\iff x_n - x_a -h F(\xo)&= \underbrace{h \int_{-\frac{1}{2}}^{\frac{1}{2}} \Delta F(\xo,t(x_n-x_a))dt}_{r(x_n,x_a)}\\
x_n - x_a -h F(\xo) &= r(x_n,x_a)
\end{align}
\begin{align}
  x_n^{\text{(n)}} - x_a - h (F(\xoold) +\Delta F(\xoold,\xonew - \xoold)) &= r(x_n^{\text{(a)}},x_a)\\
 \iff 2\xonew - 2x_a - h (F(\xoold) +\Delta F(\xoold,\xonew - \xoold)) &= r(x_n^{\text{(a)}},x_a)\\
 \iff  2\xonew -  h (F(\xoold) +\Delta F(\xoold,\xonew - \xoold)) &= r(x_n^{\text{(a)}},x_a) + 2x_a
\end{align}
\[x_n = 2\xo - x_a\]
\end{frame}

\begin{frame}
 \frametitle{Implizite SSP Methoden und CFL}
 \centering
\begin{tabular*}{0.75\textwidth}{  c | c | c | r  }
\hline 
family & stages & order & CFL \\ 
\hline 
Gauß-Legendre & 1 & 2 & 2 \\
\hline 
Radau IA& 1 & 1 & $\infty$ \\
\hline 
Radau IIA & 1 & 1 & $\infty$ \\
\hline 
Lobatto IIIA & 2 & 2 & 2 \\
\hline 
Lobatto IIIB & 2 & 2 & 2 \\
\hline 
\end{tabular*}
\end{frame}

\begin{frame}[fragile]
\frametitle{Adjoint Algorithmus}
\algrenewcommand{\algorithmiccomment}[1]{\hfill{\scriptsize #1}}
\begin{algorithmic}[1]
 \Function{calc\_kink\_partial}{$\cx,\hx,\Delta x$}
 	\State $\hat{\tau} \gets 0$,$\check{\tau} \gets 0$, $x_{kink} \gets \cx$
 	\State $\bar{A} \gets 0_{n\times n} $, $\hat{A} \gets 0_{n\times n}$
 	\Repeat
 	  \State $\check{\tau} \gets \check{\tau} + \hat{\tau}, ~ x_{kink} \gets x_{kink} +\check{\tau}\Delta x$
 	  \State $\hat{\tau} \gets \check{\tau} + \Call{critMult}{ x_{kink},\Delta x}$ 		 \Comment{Berechne kritischen Multiplikator}
		\If{$\hat{\tau}>1$} $\hat{\tau}\gets 1$ \EndIf 
 		\State $\xo \gets x_{kink}+0.5\cdot \hat{\tau} \Delta x$	\Comment{Berechne Mittelpunkt zwischen den Knicken}
 	  \State $\frac{\partial F(\rx)}{\partial x} \gets $ gen\_jac($\rx,\Delta x$) \Comment{Berechne $\partial F$ aus der Abs-Normalform}
 	  \State $\bar{A} \gets \bar{A} +  \frac{\partial F(\xo)}{\partial x} \cdot (\hat{\tau} - \check{\tau})$ 
 		\State $\hat{A} \gets \hat{A} +  \frac{\partial F(\rx)}{\partial x} \cdot  \left(\frac{1}{2}(\check{\tau} + \hat{\tau})-0.5\right)$ \Comment{Verschiebe $\tau$ um $-0.5$}
 \Until{$\hat{\tau} \geq 1$	}
 \State \Return $[\bar{A}, \hat{A}]$;
 \EndFunction
 \end{algorithmic}
\end{frame}

\begin{frame}[fragile]
\frametitle{Adjoint Algorithmus}
\algrenewcommand{\algorithmiccomment}[1]{\hfill{\scriptsize #1}}
\begin{algorithmic}[1]
%  \Require $x_{0},t_0,T, h,x_{Obs}, TOL$
 \Function{jac\_data\_assimilation}{$x_{0},t_0,T, h,x_{\text{obs}}, TOL$}
 \State $N = \ceilS{\frac{t_0 - T}{h}}$, $\hat{\bar{x}} \gets 0$ \Comment{Setze Anfangswert}
 \State $x \gets  \Call{solveODE}{x_0,t_0, T,h, TOL};$\Comment{Löse ODE in Vorwärtsrichtung}
 \For{$k\gets$ N-1 to $1$} \Comment{Zeitschritt rückwärts}
 	%\State $\rx \gets \cx - \frac h2 F(\cx)$ \Comment{initialization by half Euler}
 	\State $\rx \gets 0.5(x_k + x_{k-1})$ \Comment{Berechne Mittelpunkt}
 	\State $F(\rx) +\Delta F(\rx;\cdot) \gets$ \Call{Update}{} \Comment{Berechne neue Linearisierung an $\rx$}
 	\State $\Delta x \gets x_{k-1}-x_k$\Comment{Berechne neue Richtung}
 	\State $[\bar{A},\hat{A}] \gets \Call{calc\_kink\_partials}{x_k,x_{k-1},\Delta x}$  \Comment{Berechne $\partial F$}
 	%\Until{$\|\hx - \cx - r - h F(\rx)\|$} < TOL
 	\State{\begin{varwidth}[t]{\linewidth}$\check{\bar{x}} \gets$ Solve( \par
		    \hskip\algorithmicindent $I-\frac{h}{2}\bar{A}^\tr + h \hat{A}^\tr)\check{\bar{x}}=$\par 
		    \hskip\algorithmicindent $ (I+\frac{h}{2}\bar{A}^\tr + h\hat{A}^\tr)\hat{\bar{x}}- hC^\tr(C\rx-\rx_{\text{obs}}))$
		    \end{varwidth}
		    }
 \EndFor
 \State \Return $-\check{\bar{x}}$
 \EndFunction
  \end{algorithmic}
\end{frame}
\begin{frame}[allowframebreaks,fragile]
\frametitle{Polynomial Escape}
\algrenewcommand{\algorithmiccomment}[1]{\hfill{\scriptsize #1}}
\begin{algorithmic}[1]
 \Function{gen\_jac}{$x,\Delta x$}
 	\State $k \gets \max_i\lbrace i~|~\Delta x_i\rbrace$
 	\State $z \gets $calculate\_z($x$)
 	\For{$i=1\ldots s$}\Comment{Berechnung von $\nabla z$}
	  \State $\sigma_i\gets \sign(z_i)$ 
	  \State $\nabla z_{i,1} = Z.row(i)\cdot \Delta x$\Comment{Berechne $Z\Delta x$}
	  \State $\nabla z_{i,j+1} = Z.row(i)\cdot e_j$, für $j=1,\ldots,k-1$
	  \State $\nabla z_{i,j} = Z.row(i)\cdot e_j$, für $j=k+1,\ldots, n$
	  \State $\nabla z.row(i) \mathrel{+}= L_{i,j}\cdot \sigma_j\cdot \nabla z.row(j)$ für $j=1,\ldots, i$\Comment{Berechne $L\Sigma \nabla z$}
	  \State $j\gets 2$
	  \While{$\sigma_i==0$ \&\& $j<s$ }\Comment{Firstsign}
	    \State $\sigma_i \gets \sign(\nabla z_{i,j})$
	    \State $j++$
	  \EndWhile
	\EndFor
	\For{$i=1\ldots n$}\Comment{Berechnung von $\nabla y$}
	  \State $\nabla y_{i,1} = J.row(i)\cdot \Delta x$\Comment{Berechne $J\Delta x$}
	  \State $\nabla y_{i,j+1} = J.row(i)\cdot e_j$, für $j=1,\ldots,k-1$
	  \State $\nabla y_{i,j} = J.row(i)\cdot e_j$, für $j=k+1,\ldots, n$
	  \State $\nabla y.row(i) \mathrel{+}= Y_{i,j}\cdot \sigma_j\cdot \nabla z.row(j)$ für $j=1,\ldots, s$\Comment{Berechne $Y\nabla z$}
	\EndFor
	\For{$j=1\ldots n$}\Comment{Berechnung von $J_\sigma^E\cdot E^{-1}$}
	  \State $\tilde J_{i,j} =\nabla y_{j,i+1}$für $i=1,\ldots,k-1$
	  \State $\tilde J_{i,j} = \nabla y_{j,i} $ für $i=k+1,\ldots,n$
	  \State $\tilde J_{k,j} = \nabla y_{j,1}$
	  \State $\tilde J_{k,j} \mathrel{+}= (\tilde J_{i,j}\Delta x_i) / \Delta x_k$ für $i=1,\ldots,k-1,k+1,\ldots,n$
	\EndFor
	\State $J_\sigma = \tilde J$
 \EndFunction
 \end{algorithmic}
\end{frame}

\begin{frame}
\frametitle{LC Diode Adjungierte Gleichung I}
\begin{minipage}[c]{0.45\textwidth}
\centering
\scalebox{0.9}{\documentclass{standalone}
\IfStandalone{
	\usepackage{pgfplots,pgfplotstable}
	\usetikzlibrary{external}
	\newcommand{\fromRoot}[1]{../#1}
}{%
}

\begin{document}
\tikzsetnextfilename{lc_adjoint_sol3}
\begin{tikzpicture}
\begin{axis}[
	width=\linewidth,
	xlabel=Zeit $t$,
	ylabel=$\dot{\overline{\alpha}}$,
	legend entries ={
	$\overline{\alpha}$,
% 	$\dot{\bar x_2}$
	}
% 	legend style={at={(0.5,1.0)},anchor=south},
% 	legend columns=2
]
  	\addplot[mark=none,green,very thin] table[x index=0,y index=4] {\fromRoot img/data/lc_adjoint_sol.dat};%expl midpoint
%    	\addplot[mark=none,blue,very thin] table[x index=0,y index=5] {\fromRoot img/data/lc_adjoint_sol_grad.dat};%expl midpoint
\end{axis}
\end{tikzpicture}
\end{document}}
\scalebox{0.9}{\documentclass{standalone}
\IfStandalone{
	\usepackage{pgfplots,pgfplotstable}
	\usetikzlibrary{external}
	\newcommand{\fromRoot}[1]{../#1}
}{%
}

\begin{document}
\tikzsetnextfilename{lc_adjoint_sol4}
\begin{tikzpicture}
\begin{axis}[
	width=\linewidth,
	xlabel=Zeit $t$,
	ylabel=$\dot{\overline{\beta}}$,
	legend entries ={
	$\overline \beta$,
% 	$\dot{\bar x_2}$
	}
% 	legend style={at={(0.5,1.0)},anchor=south},
% 	legend columns=2
]
  	\addplot[mark=none,green,very thin] table[x index=0,y index=5] {\fromRoot img/data/lc_adjoint_sol.dat};%expl midpoint
%    	\addplot[mark=none,blue,very thin] table[x index=0,y index=5] {\fromRoot img/data/lc_adjoint_sol_grad.dat};%expl midpoint
\end{axis}
\end{tikzpicture}
\end{document}}
\end{minipage}
\begin{minipage}[c]{0.45\textwidth}
\centering
\scalebox{0.9}{\documentclass{standalone}
\IfStandalone{
	\usepackage{pgfplots,pgfplotstable}
	\usetikzlibrary{external}
	\newcommand{\fromRoot}[1]{../#1}
}{%
}

\begin{document}
\tikzsetnextfilename{lc_adjoint_sol3g}
\begin{tikzpicture}
\begin{axis}[
	width=\linewidth,
	xlabel=Zeit $t$,
	ylabel=$\dot{\overline{\alpha}}$,
	legend entries ={
% 	$\bar x_2$,
	$\dot{\overline{\alpha}}$
	}
% 	legend style={at={(0.5,1.0)},anchor=south},
% 	legend columns=2
]
%   	\addplot[mark=none,green,very thin] table[x index=0,y index=5] {\fromRoot img/data/lc_adjoint_sol.dat};%expl midpoint
   	\addplot[mark=none,blue,very thin] table[x index=0,y index=4] {\fromRoot img/data/lc_adjoint_sol_grad.dat};%expl midpoint
\end{axis}
\end{tikzpicture}
\end{document}}
\scalebox{0.9}{\documentclass{standalone}
\IfStandalone{
	\usepackage{pgfplots,pgfplotstable}
	\usetikzlibrary{external}
	\newcommand{\fromRoot}[1]{../#1}
}{%
}

\begin{document}
\tikzsetnextfilename{lc_adjoint_sol4g}
\begin{tikzpicture}
\begin{axis}[
	width=\linewidth,
	xlabel=Zeit $t$,
	ylabel=$\dot{\overline{\beta}}$,
	legend entries ={
% 	$\bar x_2$,
	$\dot{\overline{\beta}}$
	}
% 	legend style={at={(0.5,1.0)},anchor=south},
% 	legend columns=2
]
%   	\addplot[mark=none,green,very thin] table[x index=0,y index=5] {\fromRoot img/data/lc_adjoint_sol.dat};%expl midpoint
   	\addplot[mark=none,blue,very thin] table[x index=0,y index=5] {\fromRoot img/data/lc_adjoint_sol_grad.dat};%expl midpoint
\end{axis}
\end{tikzpicture}
\end{document}}
\end{minipage}
\end{frame}

\begin{frame}
\frametitle{LC Diode Adjungierte Gleichung II}
\begin{minipage}[c]{0.45\textwidth}
\centering
\scalebox{0.9}{\documentclass{standalone}
\IfStandalone{
	\usepackage{pgfplots,pgfplotstable}
	\usetikzlibrary{external}
	\newcommand{\fromRoot}[1]{../#1}
}{%
}

\begin{document}
\tikzsetnextfilename{lc_adjoint_sol1}
\begin{tikzpicture}
\begin{axis}[
	width=\linewidth,
	xlabel=Zeit $t$,
	ylabel=$\dot{\overline{x_2}}$,
	legend entries ={
	$\overline x_2$
	}
% 	legend style={at={(0.5,1.0)},anchor=south},
% 	legend columns=2
]
  	\addplot[mark=none,green,very thin] table[x index=0,y index=2] {\fromRoot img/data/lc_adjoint_sol.dat};%expl midpoint
%    	\addplot[mark=none,blue,very thin] table[x index=0,y index=5] {\fromRoot img/data/lc_adjoint_sol_grad.dat};%expl midpoint
\end{axis}
\end{tikzpicture}
\end{document}}
\scalebox{0.9}{\documentclass{standalone}
\IfStandalone{
	\usepackage{pgfplots,pgfplotstable}
	\usetikzlibrary{external}
	\newcommand{\fromRoot}[1]{../#1}
}{%
}

\begin{document}
\tikzsetnextfilename{lc_adjoint_sol2}
\begin{tikzpicture}
\begin{axis}[
	width=\linewidth,
	xlabel=Zeit $t$,
	ylabel=$\dot{\overline{x_3}}$,
	legend entries ={
	$\overline{x_3}$,
% 	$\dot{\bar x_2}$
	}
% 	legend style={at={(0.5,1.0)},anchor=south},
% 	legend columns=2
]
  	\addplot[mark=none,green,very thin] table[x index=0,y index=3] {\fromRoot img/data/lc_adjoint_sol.dat};%expl midpoint
%    	\addplot[mark=none,blue,very thin] table[x index=0,y index=5] {\fromRoot img/data/lc_adjoint_sol_grad.dat};%expl midpoint
\end{axis}
\end{tikzpicture}
\end{document}}
\end{minipage}
\begin{minipage}[c]{0.45\textwidth}
\centering
\scalebox{0.9}{\documentclass{standalone}
\IfStandalone{
	\usepackage{pgfplots,pgfplotstable}
	\usetikzlibrary{external}
	\newcommand{\fromRoot}[1]{../#1}
}{%
}

\begin{document}
\tikzsetnextfilename{lc_adjoint_sol1g}
\begin{tikzpicture}
\begin{axis}[
	width=\linewidth,
	xlabel=Zeit $t$,
	ylabel=$\dot{\overline{x_2}}$,
	legend entries ={
% 	$\bar x_2$,
	$\dot{\overline{x_2}}$
	}
% 	legend style={at={(0.5,1.0)},anchor=south},
% 	legend columns=2
]
%   	\addplot[mark=none,green,very thin] table[x index=0,y index=5] {\fromRoot img/data/lc_adjoint_sol.dat};%expl midpoint
   	\addplot[mark=none,blue,very thin] table[x index=0,y index=2] {\fromRoot img/data/lc_adjoint_sol_grad.dat};%expl midpoint
\end{axis}
\end{tikzpicture}
\end{document}}
\scalebox{0.9}{\documentclass{standalone}
\IfStandalone{
	\usepackage{pgfplots,pgfplotstable}
	\usetikzlibrary{external}
	\newcommand{\fromRoot}[1]{../#1}
}{%
}

\begin{document}
\tikzsetnextfilename{lc_adjoint_sol2g}
\begin{tikzpicture}
\begin{axis}[
	width=\linewidth,
	xlabel=Zeit $t$,
	ylabel=$\dot{\overline{x_3}}$,
	legend entries ={
% 	$\bar x_3$,
	$\dot{\overline{x_3}}$
	}
% 	legend style={at={(0.5,1.0)},anchor=south},
% 	legend columns=2
]
%   	\addplot[mark=none,green,very thin] table[x index=0,y index=5] {\fromRoot img/data/lc_adjoint_sol.dat};%expl midpoint
   	\addplot[mark=none,blue,very thin] table[x index=0,y index=3] {\fromRoot img/data/lc_adjoint_sol_grad.dat};%expl midpoint
\end{axis}
\end{tikzpicture}
\end{document}}
\end{minipage}
\end{frame}


\begin{frame}
\frametitle{Adjungierte Gleichung Shallow Water Equation}
\begin{minipage}[c]{0.45\textwidth}
\centering
\scalebox{0.9}{\documentclass{standalone}
\IfStandalone{
	\usepackage{pgfplots,pgfplotstable}
	\usetikzlibrary{external}
	\newcommand{\fromRoot}[1]{../#1}
}{%
}

\begin{document}
\tikzsetnextfilename{swe_adjoint_sol1}
\begin{tikzpicture}
\begin{axis}[
	width=\linewidth,
	xlabel=Zeit $t$,
	ylabel=$\overline{h}$,
	legend entries ={
	$\overline{h}$
	},
% 	legend style={at={(0.5,1.0)},anchor=south},
% 	legend columns=2
]
\foreach \ind in {1,2,...,11}{
  	\addplot[mark=none,green,very thin] table[x index=0,y index=\ind] {\fromRoot img/data/swe_adjoint_sol.dat};%expl midpoint
 }
\end{axis}
\end{tikzpicture}
\end{document}}
\scalebox{0.9}{\documentclass{standalone}
\IfStandalone{
	\usepackage{pgfplots,pgfplotstable}
	\usetikzlibrary{external}
	\newcommand{\fromRoot}[1]{../#1}
}{%
}

\begin{document}
\tikzsetnextfilename{swe_adjoint_sol2}
\begin{tikzpicture}
\begin{axis}[
	width=\linewidth,
	xlabel=Zeit $t$,
	ylabel=$\overline{hu}$,
	legend entries ={
	$\overline{hu}$
	},
% 	legend style={at={(0.5,1.0)},anchor=south},
% 	legend columns=2
]
\foreach \ind in {1,2,...,11}{
  	\addplot[mark=none,green,very thin] table[x index=0,y index=\ind] {\fromRoot img/data/swe_adjoint_sol1.dat};%expl midpoint
 }
\end{axis}
\end{tikzpicture}
\end{document}}
\end{minipage}
\begin{minipage}[c]{0.45\textwidth}
\centering
\scalebox{0.9}{\documentclass{standalone}
\IfStandalone{
	\usepackage{pgfplots,pgfplotstable}
	\usetikzlibrary{external}
	\newcommand{\fromRoot}[1]{../#1}
}{%
}

\begin{document}
\tikzsetnextfilename{swe_adjoint_sol1_grad}
\begin{tikzpicture}
\begin{axis}[
	width=\linewidth,
	xlabel=Zeit $t$,
	ylabel=$\dot{\overline{h}}$,
	legend entries ={
	$\dot{\overline{h}}$
	},
% 	legend style={at={(0.5,1.0)},anchor=south},
% 	legend columns=2
]
\foreach \ind in {1,2,...,11}{
  	\addplot[mark=none,blue,very thin] table[x index=0,y index=\ind] {\fromRoot img/data/swe_adjoint_sol_grad.dat};%expl midpoint
 }
\end{axis}
\end{tikzpicture}
\end{document}}
\scalebox{0.9}{\documentclass{standalone}
\IfStandalone{
	\usepackage{pgfplots,pgfplotstable}
	\usetikzlibrary{external}
	\newcommand{\fromRoot}[1]{../#1}
}{%
}

\begin{document}
\tikzsetnextfilename{swe_adjoint_sol2_grad}
\begin{tikzpicture}
\begin{axis}[
	width=\linewidth,
	xlabel=Zeit $t$,
	ylabel=$\dot{\overline{hu}}$,
	legend entries ={
	$\dot{\overline{hu}}$
	},
% 	legend style={at={(0.5,1.0)},anchor=south},
% 	legend columns=2
]
\foreach \ind in {1,2,...,11}{
  	\addplot[mark=none,blue,very thin] table[x index=0,y index=\ind] {\fromRoot img/data/swe_adjoint_sol1_grad.dat};%expl midpoint
 }
\end{axis}
\end{tikzpicture}
\end{document}}
\end{minipage}
\end{frame}





\begin{frame}
\frametitle{Adjungierte Gleichung Rolling Stones}
\begin{minipage}[c]{0.45\textwidth}
\centering
\scalebox{1.1}{\documentclass{standalone}
\IfStandalone{
	\usepackage{pgfplots,pgfplotstable}
	\usetikzlibrary{external}
	\newcommand{\fromRoot}[1]{../#1}
}{%
}

\begin{document}
\tikzsetnextfilename{rolling_adjoint_sol1}
\begin{tikzpicture}
\begin{axis}[
	width=\linewidth,
	xlabel=Zeit $t$,
	ylabel=$\dot{\bar x}$,
	legend entries ={
	$\bar x_1$,
	$\dot{\bar x}_1$
	},
% 	legend style={at={(0.5,1.0)},anchor=south},
% 	legend columns=2
]
  	
  
  \addplot[mark=none,green,very thin] table[x index=0,y index=1] {\fromRoot img/data/rolling_adjoint_solution.dat};%impl midpoint
  \addplot[mark=none,blue,very thin] table[x index=0,y index=1] {\fromRoot img/data/rolling_adjoint_sol_grad.dat};%expl midpoint
  
\end{axis}
\end{tikzpicture}
\end{document}}
\end{minipage}
\begin{minipage}[c]{0.45\textwidth}
\centering
\scalebox{1.1}{\documentclass{standalone}
\IfStandalone{
	\usepackage{pgfplots,pgfplotstable}
	\pgfplotsset{compat=1.10}
	\usetikzlibrary{external}
	\newcommand{\fromRoot}[1]{../#1}
}{%
}

\begin{document}
\tikzsetnextfilename{rolling_adjoint_sol2}
\begin{tikzpicture}
\begin{axis}[
	width=\linewidth,
	xlabel=Zeit $t$,
	ylabel=$\dot{\bar x}$,
	legend entries ={
	$\bar x_2$,
	$\dot{\bar x}_2$
	},
% 	legend style={at={(0.5,1.0)},anchor=south},
% 	legend columns=2
]
      
	
  \addplot[mark=none,green,very thin] table[x index=0,y index=2] {\fromRoot img/data/rolling_adjoint_solution.dat};%impl midpoint	
%   \addplot[mark=none,blue,very thin] table[x index=0,y index=1] {\fromRoot img/data/rolling_adjoint_sol_grad.dat};%expl midpoint
  \addplot[mark=none,blue,very thin] table[x index=0,y index=2] {\fromRoot img/data/rolling_adjoint_sol_grad.dat};%expl midpoint
  
\end{axis}
\end{tikzpicture}
\end{document}}
\end{minipage}
\end{frame}


% \begin{frame}[<+->]
% \frametitle{Experimente}
% CFL IMAGE
% \end{frame}


\begin{frame}[fragile]
\frametitle{Implementierung}
\cite{openblas}
\cite{armadillo}
\cite{ADOLCmanual}
\cite{boeck14}
\end{frame}

% Backup frames
\setcounter{framenumber}{\value{finalframe}}
