\section[Adjungierte Gleichung]{Adjungierte Gleichung}

\begin{frame}[<+->]
\frametitle{Berechnung von $\frac{\partial F(x)}{\partial x}$}
Nutze die erste Zeile der Abs Normal Form
\begin{align*}
z &= c+Zx + L\Sigma z \\
\iff (I-L\Sigma)z &= c+Zx  \\
\iff z &= (I-L\Sigma)^{-1}(c+Zx) 
\end{align*}
\pause
um sie in die zweite Zeile einzusetzen
\begin{align*}
y &= b+ Jx + Y|z| \\
y &= b+ Jx + Y\Sigma z \\
y &= b+ Jx + Y\Sigma (I-L\Sigma)^{-1}(c+Zx) \\
y &= b+  Y\Sigma (I-L\Sigma)^{-1}c + \underbrace{(J + Y\Sigma (I-L\Sigma)^{-1}Z)}_{J_\sigma} x \\
\end{align*}
wobei $(I-L\Sigma)^{-1} = I +L\Sigma + (L\Sigma)^2+\ldots$ eine Neumannreihe ist.
\end{frame}

\begin{frame}[<+->]
\frametitle{Differential Inclusion}
% GRADIENT NICHT NUR STETIGE UNGLÄTTEN SONDERN AUCH SPRÜNGE -> ABS EXAMPLE
\centering
\begin{figure}
  \begin{minipage}{0.45\textwidth} 
	\includegraphics[width=\linewidth]{../dipl_tex/img/tikz/adj_valley_tracing.pdf}
	\end{minipage}
	\hfill
	\begin{minipage}{0.45\textwidth}
	\includegraphics[width=\linewidth]{../dipl_tex/img/tikz/adj_valley_tracing1.pdf}	
	\end{minipage}
	
\end{figure}
Differential Inclusion und Valley Tracing Mode \\
\pause
Mögliche Lösungsansätze:\hfill
\begin{itemize}
 \item Polynomial Escape
 \item Einbeziehung der Kinks
\end{itemize}
\end{frame}

\begin{frame}[<+->]
\frametitle{Polynomial Escape}
POLYNOMIAL ESCAPE
\end{frame}

\begin{frame}[<+->]
\frametitle{Adjungierte Gleichung}
\begin{figure}
\centering
\includegraphics[width=0.65\linewidth]{../dipl_tex/img/tikz/multiple_kinks_adjoint.pdf}
\end{figure}
\begin{figure}
\centering
% \resizebox{.9\linewidth}{!}{ \documentclass{standalone}
\IfStandalone{
	\usepackage{pgfplots,pgfplotstable}
	\usetikzlibrary{external}
	\newcommand{\fromRoot}[1]{../#1}
	\newcommand{\D}{\Delta}
	\pgfplotsset{compat=1.9}
}{%
}

\begin{document}
\tikzsetnextfilename{piecewise_linearization}
\begin{tikzpicture}
\def\xo{-1};
\pgfmathdeclarefunction{f1}{1}{%
	\pgfmathparse{-(6/10)*(#1+2)^2}%
}
\pgfmathdeclarefunction{f2}{1}{%
	\pgfmathparse{3*( #1+sin((11/10)*deg(#1)) )}%
}
\pgfmathdeclarefunction{tf1}{1}{%
	\pgfmathparse{-1.2*(#1+1)-0.6}%
}
\pgfmathdeclarefunction{tf2}{1}{%
	\pgfmathparse{4.49687*(#1+1)-5.67362}%
}
\begin{axis}[
	height=0.45\textheight,
	axis y line = none,
	axis x line = bottom,
	xmin=-4,xmax=4,
	xtick = \empty,
	ytick = \empty,
	extra x ticks = {\xo},
	extra x tick labels={\(\mathring x\)},
	extra x tick style = {grid=major},
	domain=-4:4,
	samples=500,
]
\addplot[red!60!white] {f1(x)} 
	[yshift=2pt] 
	node[anchor=south,pos=0.8] {$F_1$};
\addplot[blue!60!white] {f2(x)} 
	node[anchor=south,pos=0.1] {$F_2$};
\addplot[black,yshift=1pt] {max(f1(x),f2(x)} 
	node[anchor=south,pos=0.2] {$F = \max(F_1,F_2)$};

\addplot[red!60!white,very thin] {tf1(x)} 
	node[anchor=south,pos=0.75,sloped] {$\mathring F_1 + F_1'(\mathring x)\D x$};
\addplot[blue!60!white,very thin] {tf2(x)} 
	node[anchor=north,pos=0.15,sloped] {$\mathring F_2 + F_2'(\mathring x)\D x$};
\addplot[black,very thin,yshift=1pt] {max(tf1(x),tf2(x)} 
	node[anchor=south,pos=0.75,sloped] {$\mathring F + \D F(\mathring x;\D x)$};
%\addplot [color=black,only marks,mark=*] coordinates { (-1,-0.6) };
%\addplot [color=black,only marks,mark=*] coordinates { (-1,-5.67362) };
\end{axis}
\end{tikzpicture}

 
\end{document}
 }
\includegraphics[width=0.65\linewidth]{../dipl_tex/img/tikz/multiple_kinks_adjoint_new.pdf}
% \caption{Punkte $\xo$ der Gradientberechnung in der adjungierten Gleichung mit $\xo^{(i)} = \xo + 0.5(\tau_i+\tau_{i-1})\Delta x$}
\end{figure}
\centering
Gradientberechnung der adjungierten Gleichung mit $\xo^{(i)} = \xo + 0.5(\tau_i+\tau_{i-1})\Delta x$
\end{frame}

\begin{frame}[<+->]
\frametitle{Adjungierte Gleichung}
GLEICHUNG HERLEITEN -> ALGORITHMUS?
\end{frame}
