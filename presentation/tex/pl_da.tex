\section[Adjungierte Gleichung]{Adjungierte Gleichung}

\begin{frame}[<+->]
\frametitle{Berechnung von $\frac{\partial F(x)}{\partial x}$}
Nutze die erste Zeile der Abs Normal Form
\begin{align*}
z &= c+Zx + L\Sigma z \\
\iff (I-L\Sigma)z &= c+Zx  \\
\iff z &= (I-L\Sigma)^{-1}(c+Zx) 
\end{align*}
\pause
um sie in die zweite Zeile einzusetzen
\begin{align*}
y &= b+ Jx + Y|z| \\
\iff y &= b+ Jx + Y\Sigma z \\
\iff y &= b+ Jx + Y\Sigma (I-L\Sigma)^{-1}(c+Zx) \\
\iff y &= b+  Y\Sigma (I-L\Sigma)^{-1}c + \underbrace{(J + Y\Sigma (I-L\Sigma)^{-1}Z)}_{J_\sigma} x \\
\end{align*}
\vspace*{1cm}
wobei $(I-L\Sigma)^{-1} = I +L\Sigma + (L\Sigma)^2+\ldots$ eine Neumannreihe ist.
\end{frame}

\begin{frame}[<+->]
\frametitle{Differential Inclusion}
% GRADIENT NICHT NUR STETIGE UNGLÄTTEN SONDERN AUCH SPRÜNGE -> ABS EXAMPLE
\centering
\begin{figure}
  \begin{minipage}{0.45\textwidth} 
	\includegraphics[width=\linewidth]{../dipl_tex/img/tikz/adj_valley_tracing.pdf}
	\end{minipage}
	\hfill
	\begin{minipage}{0.45\textwidth}
	\includegraphics[width=\linewidth]{../dipl_tex/img/tikz/adj_valley_tracing1.pdf}	
	\end{minipage}
	
\end{figure}
Differential Inclusion und Valley Tracing Mode \\
\pause
Mögliche Lösungsansätze:\hfill
\begin{itemize}
 \item Polynomial Escape
 \item Einbeziehung der Knicke
\end{itemize}
\end{frame}
% \subsection{Polynomial Escape}
\begin{frame}[<+->]
\frametitle{Polynomial Escape I}
\begin{block}{Polynomial Escape \cite[Prop. 6]{monster}}
 Es kann kein Pfad der Form
 \vspace*{-0.3cm}
 \[\Delta \hat x = \sum_{j=1}^n e_j t^j, ~ 0<t<\bar t, e_j\in R^n,~ \det(e_1,\ldots,e_n) \neq 0\]
 \vspace*{0.3cm}
 für ein $\bar t$ auf einem Knick liegen (\cite[Proposition 6]{monster},\cite[S.11]{plan}).
%Nahc vorraussetzung gibts nur endlich viele hyperflächen (Kinks) -> darum geht das
\end{block}
\begin{block}{Firstsign}
 Def. $\nabla x = (\nabla x_i^\tr)_{i=1,\ldots,n}$, $\det(\nabla x)\neq 0$, $\Delta x_k$ betragsmäßig größtes Element von $\Delta x$: 
 \vspace*{-0.3cm}
 \[
 \nabla x = E =
\begin{pmatrix}
  \nabla x_1^\tr\\
  \vdots\\
  \nabla x_n^\tr\\
 \end{pmatrix}
 =
  \begin{pmatrix}
   I_{n-k} & \vdots &0\\
  0 & \Delta x_k & 0\\
    0 & \vdots&I_{k-1}
 \end{pmatrix}
\]

% , ansonsten 0.
\end{block}

\end{frame}

\begin{frame}[<+->]
\frametitle{Polynomial Escape II}
\begin{block}{Firstsign}
 Berechne $z$ und Ableitung der Abs Normal Form
 \vspace*{-0.3cm}
 \begin{equation}\label{eq:polynomialEscape}
 \begin{aligned}
  \nabla z &= Z \nabla x + L \Sigma \nabla z\\
  \nabla y &= J\nabla x + Y\Sigma \nabla z
 \end{aligned}
 \end{equation}
 und nutze $
       \sigma_i = \firstsign(z_i,\nabla z_i^\tr \Delta x) =\sign (z_i + \Delta z_i)
      $ für $\Sigma$
wobei $\firstsign(z)$ das Vorzeichen des ersten nichtverschwindende Eintrag von $z$ ist
% , ansonsten 0.
\end{block}
\begin{block}{Polynomial Escape \cite[Prop. 8]{monster}}
Die Auswertung von \eqref{eq:polynomialEscape} ergibt eine Matrix $J_\sigma^E = (\nabla y_{i})_{i=1,\ldots,m}$ dessen Rücktransformation
\[
 J_\sigma\equiv J_\sigma^E E^{-1} \in \partial^L \Delta F(\xo;0)
\]
eine limiting Jacobian von $F'(\xo;\Delta x)$ am Punkt $\xo$ ist.
\end{block}

\end{frame}


% \subsection{Adjungierte Gleichung}
\begin{frame}[<+->]
\frametitle{Adjungierte Gleichung}
\begin{figure}
\centering
\includegraphics[width=0.65\linewidth]{../dipl_tex/img/tikz/multiple_kinks_adjoint.pdf}
\end{figure}
\begin{figure}
\centering
% \resizebox{.9\linewidth}{!}{ \documentclass{standalone}
\IfStandalone{
	\usepackage{pgfplots,pgfplotstable}
	\usetikzlibrary{external}
	\newcommand{\fromRoot}[1]{../#1}
	\newcommand{\D}{\Delta}
	\pgfplotsset{compat=1.9}
}{%
}

\begin{document}
\tikzsetnextfilename{piecewise_linearization}
\begin{tikzpicture}
\def\xo{-1};
\pgfmathdeclarefunction{f1}{1}{%
	\pgfmathparse{-(6/10)*(#1+2)^2}%
}
\pgfmathdeclarefunction{f2}{1}{%
	\pgfmathparse{3*( #1+sin((11/10)*deg(#1)) )}%
}
\pgfmathdeclarefunction{tf1}{1}{%
	\pgfmathparse{-1.2*(#1+1)-0.6}%
}
\pgfmathdeclarefunction{tf2}{1}{%
	\pgfmathparse{4.49687*(#1+1)-5.67362}%
}
\begin{axis}[
	height=0.45\textheight,
	axis y line = none,
	axis x line = bottom,
	xmin=-4,xmax=4,
	xtick = \empty,
	ytick = \empty,
	extra x ticks = {\xo},
	extra x tick labels={\(\mathring x\)},
	extra x tick style = {grid=major},
	domain=-4:4,
	samples=500,
]
\addplot[red!60!white] {f1(x)} 
	[yshift=2pt] 
	node[anchor=south,pos=0.8] {$F_1$};
\addplot[blue!60!white] {f2(x)} 
	node[anchor=south,pos=0.1] {$F_2$};
\addplot[black,yshift=1pt] {max(f1(x),f2(x)} 
	node[anchor=south,pos=0.2] {$F = \max(F_1,F_2)$};

\addplot[red!60!white,very thin] {tf1(x)} 
	node[anchor=south,pos=0.75,sloped] {$\mathring F_1 + F_1'(\mathring x)\D x$};
\addplot[blue!60!white,very thin] {tf2(x)} 
	node[anchor=north,pos=0.15,sloped] {$\mathring F_2 + F_2'(\mathring x)\D x$};
\addplot[black,very thin,yshift=1pt] {max(tf1(x),tf2(x)} 
	node[anchor=south,pos=0.75,sloped] {$\mathring F + \D F(\mathring x;\D x)$};
%\addplot [color=black,only marks,mark=*] coordinates { (-1,-0.6) };
%\addplot [color=black,only marks,mark=*] coordinates { (-1,-5.67362) };
\end{axis}
\end{tikzpicture}

 
\end{document}
 }
\includegraphics[width=0.65\linewidth]{../dipl_tex/img/tikz/multiple_kinks_adjoint_new.pdf}
% \caption{Punkte $\xo$ der Gradientberechnung in der adjungierten Gleichung mit $\xo^{(i)} = \xo + 0.5(\tau_i+\tau_{i-1})\Delta x$}
\end{figure}
\centering
Gradientberechnung der adjungierten Gleichung mit $\xo^{(i)} = \xo + 0.5(\tau_i+\tau_{i-1})\Delta x$
\end{frame}

\begin{frame}[<+->]
\frametitle{Adjungierte Gleichung}
Wende auf die adjungierte Gleichung
\begin{align}
 \dot \xadj =  - \frac{\partial F(x;\Delta x)}{\partial x}^\tr \xadj + C^\tr (Cx-\xobs) 
\end{align}
\pause
die verallgemeinerte implizite Mittelpunktsregel an:
\begin{align}
%  \xadj_n -\xadj_a = \int_{-\frac{1}{2}}^{\frac{1}{2}}C^\tr
\xadj_n - \xadj_a &= h\cdot \int_{-0.5}^{0.5}C^\tr(C\xo-\rxobs) - \frac{\partial F(\xo,\Delta x)}{\partial x}^\tr \cdot \xadj dt\\
		  &= h\cdot \left[ C^\tr(C\xo-\rxobs) -\int_{-0.5}^{0.5} \frac{\partial F(\xo,\Delta x)}{\partial x}^\tr \cdot \xadj dt\right]
\end{align}
\pause
mit $N$ Knicken, $-0.5 = \tau_0 <\tau_1 <\ldots < \tau_N=0.5$ und $\mathring \tau_i = 0.5 (\tau_i +\tau_{i-1})$ folgt
\begin{align}
\xadj_n - \xadj_a &= h\left[ C^\tr(C\xo-\rxobs) - \sum_{i=1}^N \int_{\tau_{i-1}}^{\tau_{i}}\underbrace{\frac{\partial F(\xo+\mathring \tau_i\Delta x,\Delta x)}{\partial x}^\tr}_{A_i^\tr} \cdot \xadj dt\right]\\
\end{align}
\end{frame}

\begin{frame}[<+->]
\frametitle{Adjungierte Gleichung}
Mit  $\mathring \tau_i = 0.5 (\tau_i +\tau_{i-1})$ und $\Delta \tau_i = \tau_i-\tau_{i-1}$ folgt
\pause
\begin{align}
\xadj_n - \xadj_a &= h\cdot \left[C^\tr(C\xo -\rxobs) - \sum_{i=1}^N A_i^\tr \cdot \int_{\tau_{i-1}}^{\tau_{i}} \rxadj + t\Delta \xadj dt\right]\\
		  &= \ldots\\
		  &= h\cdot \left[C^\tr(C\xo -\rxobs) - \sum_{i=1}^N A_i^\tr \cdot \left( \diff \tau_i\cdot \rxadj +  \diff \tau_i \rtau_i \diff \xadj \right)\right]
\end{align}
\pause
Da $\rxadj = 0.5(\xadj_n + \xadj_a)$ und $\Delta \xadj = \xadj_n-\xadj_a$ gilt, folgt mittels Umsortieren
\[
\begin{aligned}
\left[I +h\sum_{i=1}^N A_i^\tr \left(\frac{1}{2} \diff \tau_i +\diff \tau_i \rtau_i\right) \right]\xadj_n &= 
&\left[I - h\sum_{i=1}^N A_i^\tr  \left(\frac{1}{2}\diff \tau_i-\diff \tau_i \rtau_i\right)\right]\xadj_a \\
&&+h C^\tr(C\xo -\rxobs)
\end{aligned}
\]
\end{frame}

% \subsection{Algorithmus}
% \begin{frame}[fragile]
% \frametitle{Algorithmus}
% \algrenewcommand{\algorithmiccomment}[1]{\hfill{\scriptsize #1}}
% \begin{algorithmic}[1]
%  \Function{calc\_kink\_partial}{$\cx,\hx,\Delta x$}
%  	\State $\hat{\tau} \gets 0$,$\check{\tau} \gets 0$, $x_{kink} \gets \cx$
%  	\State $\bar{A} \gets 0_{n\times n} $, $\hat{A} \gets 0_{n\times n}$
%  	\Repeat
%  	  \State $\check{\tau} \gets \check{\tau} + \hat{\tau}, ~ x_{kink} \gets x_{kink} +\check{\tau}\Delta x$
%  	  \State $\hat{\tau} \gets \check{\tau} + \Call{critMult}{ x_{kink},\Delta x}$ 		 \Comment{Berechne kritischen Multiplikator}
% 		\If{$\hat{\tau}>1$} $\hat{\tau}\gets 1$ \EndIf 
%  		\State $\xo \gets x_{kink}+0.5\cdot \hat{\tau} \Delta x$	\Comment{Berechne Mittelpunkt zwischen den Kinks}
%  	  \State $\frac{\partial F(\rx)}{\partial x} \gets $ gen\_jac($\rx,\Delta x$) \Comment{Berechne $\partial F$ aus der Abs-Normalform}
%  	  \State $\bar{A} \gets \bar{A} +  \frac{\partial F(\xo)}{\partial x} \cdot (\hat{\tau} - \check{\tau})$ 
%  		\State $\hat{A} \gets \hat{A} +  \frac{\partial F(\rx)}{\partial x} \cdot  \left(\frac{1}{2}(\check{\tau} + \hat{\tau})-0.5\right)$ \Comment{Verschiebe $\tau$ um $-0.5$}
%  \Until{$\hat{\tau} \geq 1$	}
%  \State \Return $[\bar{A}, \hat{A}]$;
%  \EndFunction
%  \end{algorithmic}
% \end{frame}
% 
% \begin{frame}[fragile]
% \frametitle{Algorithmus}
% \algrenewcommand{\algorithmiccomment}[1]{\hfill{\scriptsize #1}}
% \begin{algorithmic}[1]
% %  \Require $x_{0},t_0,T, h,x_{Obs}, TOL$
%  \Function{jac\_data\_assimilation}{$x_{0},t_0,T, h,x_{\text{obs}}, TOL$}
%  \State $N = \ceilS{\frac{t_0 - T}{h}}$, $\hat{\bar{x}} \gets 0$ \Comment{Setze Anfangswert}
%  \State $x \gets  \Call{solveODE}{x_0,t_0, T,h, TOL};$\Comment{Löse ODE in Vorwärtsrichtung}
%  \For{$k\gets$ N-1 to $1$} \Comment{Zeitschritt rückwärts}
%  	%\State $\rx \gets \cx - \frac h2 F(\cx)$ \Comment{initialization by half Euler}
%  	\State $\rx \gets 0.5(x_k + x_{k-1})$ \Comment{Berechne Mittelpunkt}
%  	\State $F(\rx) +\Delta F(\rx;\cdot) \gets$ \Call{Update}{} \Comment{Berechne neue Linearisierung an $\rx$}
%  	\State $\Delta x \gets x_{k-1}-x_k$\Comment{Berechne neue Richtung}
%  	\State $[\bar{A},\hat{A}] \gets \Call{calc\_kink\_partials}{x_k,x_{k-1},\Delta x}$  \Comment{Berechne $\partial F$}
%  	%\Until{$\|\hx - \cx - r - h F(\rx)\|$} < TOL
%  	\State{\begin{varwidth}[t]{\linewidth}$\check{\bar{x}} \gets$ Solve( \par
% 		    \hskip\algorithmicindent $I-\frac{h}{2}\bar{A}^\tr + h \hat{A}^\tr)\check{\bar{x}}=$\par 
% 		    \hskip\algorithmicindent $ (I+\frac{h}{2}\bar{A}^\tr + h\hat{A}^\tr)\hat{\bar{x}}- hC^\tr(C\rx-\rx_{\text{obs}}))$
% 		    \end{varwidth}
% 		    }
%  \EndFor
%  \State \Return $-\check{\bar{x}}$
%  \EndFunction
%   \end{algorithmic}
% \end{frame}
