%Dokumentenklasse
\documentclass{article}

%Sprach-Packages
%\usepackage[utf8]{inputenc}%Uni
\usepackage{etex}
\usepackage[utf8]{inputenc}%AT home
\usepackage[german]{babel}
\usepackage[T1]{fontenc}


\include{inc/mathenv}
\include{inc/commands}
%-----
\title{Problemstellung}
\author{Ben Lenser}
%\date{}                                           % Activate to display a given date or no date

\begin{document}
\maketitle



\section{Gegeben}
\begin{itemize}
	\item X(T)  State Variable, beschreibt evolution des Mediums in den Grid Points
	\item $\begin{cases}
						\frac{dX}{dt} = F(X,U)\\
					  X(0) = V
					 \end{cases}$
					Modell, beschreibt Fließ-Evolution; System nicht linearer differentialgleichungen
	\item Kontrollvariablen $(U,V)\in \P$ im Kontrollraum (erstellt aus den Anfangsbedingungen oder interne Variablen des Modells (Parameter oder Boundary Conditions))
	\item $\Xobs$ Diskrete Messdaten (zeitabhängig, Funktionsraum abhängig)
	\item $C:S\to O$: Abbildung vom State Raum $S$ in Observationraum ($O$)
	\item $J:= \int_0^T \frac{1}{2} \|C\cdot X(U,V) - \Xobs \|$ Kostenfunktion (Diskrepanz zwischen Modelllösung und Messwerten)
\end{itemize}

\section{Problem}
\begin{ass}
	Finde Lösungen $U=U^*, V=V^*$ mit
	\[ J(U^*,V^*) = \inf_{(U,V)\in \P}J(U,V)\]
	\end{ass}
Sei 
\begin{itemize}
	\item $(u,v) \in \P$: Richtungen aus dem Steuerungsraum
	\item $\Xhat$: Gateaux Ableitung (Richtungsableitung) von $X$ in Richtung $(u,v)$: Ist die Lösung von Tangent Linear Model (TLM):
	\[
		\begin{cases}
				\frac{d\Xhat}{dt} =\left[ \frac{\partial F}{\partial X}\right]\cdot \Xhat +  \left[ \frac{\partial F}{\partial U}\right] \cdot u\\
				\Xhat(0) = v
		\end{cases}
	\], $\left[ \frac{\partial F}{\partial X}\right]$ ist Jacobimatrix des Modells bezüglich der State variable
\end{itemize}

Einfaches Ableiten von $J$ ergibt:
\begin{align*}
	\hat{J}(U,V,u,v)&= (J(U,V))'\\
									&= (\int_0^T \frac{1}{2} \|C\cdot X(U,V) - \Xobs \|)'\\
									&= \int_0^T (C\cdot X(U,V) - \Xobs ,C\Xhat)dt
\end{align*}
$\hat{J}$ ist linear abhängig von (u,v), es ergibt sich ein expliziter Ausruch des Funktional - Gradienten.\\
Wir definieren $P$ als adjungierte Variable.

 
\end{document}  

















